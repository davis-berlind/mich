% document set up
\documentclass{article}
\usepackage[utf8]{inputenc}
\usepackage{fullpage}
\setlength{\parindent}{0cm}
\setlength{\parskip}{1em}

% math packages
\usepackage{amsfonts}
\usepackage{amsmath}
\usepackage{amssymb}
\usepackage{amsthm}
\usepackage{enumitem}
\usepackage{bbm}

% general packages
\usepackage[colorlinks=true,linkcolor=blue,citecolor=blue]{hyperref}
\usepackage{natbib}

\newcommand\indep{\perp\!\!\!\perp}
\newcommand{\sforall}{\;\forall\;}
\newcommand{\E}{\mathbb{E}}
\newcommand{\Var}{\text{Var}}
\newcommand{\Cov}{\text{Cov}}
\newcommand{\argmin}[1]{\underset{#1}{\text{arg min}}}
\newcommand{\argmax}[1]{\underset{#1}{\text{arg max}}}

\newtheorem{assumption}{Assumption}
\newtheorem{theorem}{Theorem}
\newtheorem{proposition}{Proposition}
\newtheorem{corollary}{Corollary}
\newtheorem{remark}{Remark}
\newtheorem{definition}{Definition}
\newtheorem{lemma}{Lemma}

\begin{document}

\begin{definition}\label{def:a-mixing}
Let $\{X_t\}_{t\in \mathbb{Z}}$ be a stochastic process on the probability space $(\Omega, \mathcal{F}, \mathbb{P})$, then $\{X_t\}_{t\in \mathbb{Z}}$ is said to be $\alpha$-mixing if:
\begin{align*}
    \lim_{k\to\infty} \alpha_k(\{X_t\}_{t\in\mathbb{Z}}) = 0,
\end{align*}
where:
\begin{align*}
    \alpha_k(\{X_t\}_{t\in\mathbb{Z}}) := \sup_{t\in\mathbb{Z}} \; \alpha\left(\sigma(\{X_s\}_{s \leq t}), \; \sigma(\{X_s\}_{s \geq t + k})\right).
\end{align*}
Here $\sigma(Y)$ is the $\sigma$-algebra generated by $Y$ and the strong mixing, or $\alpha$-mixing, coefficient between two $\sigma$-algebras $\mathcal{A}, \mathcal{B} \subset \mathcal{F}$ is defined as:
\begin{align*}
    \alpha(\mathcal{A}, \mathcal{B}) := \sup_{A\in\mathcal{A}, B\in\mathcal{B}} |\mathbb{P}(A \cap B) - \mathbb{P}(A)\mathbb{P}(B)|.
\end{align*}
To simplify notation, we will often write $\alpha_k$ in place of $\alpha_k(\{X_t\}_{t\in\mathbb{Z}})$.
\end{definition}

\begin{lemma}[Lemma 3 in \cite{Padilla23}]
Let $\nu > 0$ be given. Suppose that $\{X_t\}_{t=1}^\infty$ is a stationary $\alpha$-mixing time-series with mixing coefficients $\{\alpha_k\}_{k=0}^K$. Suppose that $\E[X_t] = 0$ and that there exists constants $\delta, \Delta, D_1, D_2 > 0$ such that:
\begin{align*}
    \sup_{t \geq 1} \E\left[\left|X_t\right|^{2+\delta+\Delta}\right] \leq D_1 
\end{align*}
and:
\begin{align*}
    \sum_{k=0}^\infty (k+1)^{\frac{\delta}{2}} \alpha_k^{\frac{\Delta}{2+\delta+\Delta}} \leq D_1.
\end{align*}
Then for any $a \in (0,1)$, there is some constant $C > 0$ such that:
\begin{align*}
    \mathbb{P} \left(\left|\sum_{t=1}^T X_t\right| \leq \frac{C}{a}\sqrt{T}\left[\log (\nu T) + 1\right], \;\sforall T \geq \nu^{-1}\right) 
    \geq 1 - a^2.
\end{align*}
\end{lemma}

\begin{assumption}
Let $t_0$ be the time instance such that $Y_t \sim F_l$ for $t < t_0$ and $Y_t \sim F_r$ for $t \geq t_0$. Suppose that the distributions $F_l$ and $F_r$ are such that:
\begin{align*}
    \E[Y_t] &= 
    \begin{cases}
        0, & t < t_0, \\
        b_0, & t \geq t_0.
    \end{cases} \\
    \Var(Y_t) &= 
    \begin{cases}
        1, & t < t_0, \\
        s_0^2, & t \geq t_0.
    \end{cases}
\end{align*}
and there exists constants $\delta_1, D_1 > 0$ and $\delta_2, D_2 > 0$ such that:
\begin{align*}
    \max\left\{\E\left[|Y_1|^{2+\delta_1}\right], \E\left[|Y_{t_0} - b_0|^{2+\delta_1}\right]\right\} \leq D_1
\end{align*}
and:
\begin{align*}
    \max\left\{\E\left[|Y^2_1 - 1|^{2+\delta_2}\right], \E\left[|(Y_{t_0} - b_0)^2 - s_0^2|^{2+\delta_2}\right]\right\} \leq D_2.
\end{align*}
We further assume that $\{Y_t\}_{t=1}^\infty$ is an $\alpha$-mixing process (see Definition \ref{def:a-mixing}) with coefficients that satisfy $\alpha_k \leq e^{-Ck}$ for some constant $C > 0$.
\end{assumption}

%%%%%%

\begin{proof}

\subsubsection*{Case 1: $t > t_0$ and $\min\{t, T-t\} > cT$.}

From (\ref{eq:thm3-delta_t}) we have:
\begin{align*}
    \log \frac{\Delta_{t_0}}{\Delta_t} &= \log \left(\frac{T - t + 1}{T - t_0 +1} \right) + \frac{1}{2} \sum_{t' = t_0}^{t - 1} y_{t'}^2 \\
    &\quad\: + \left(\frac{T - t_0 + 1}{2}\right) \log\left(\frac{T-t_0+1}{2}\right) - \left(\frac{T - t + 1}{2}\right) \log\left(\frac{T-t+1}{2}\right) - \frac{t-t_0}{2} \\
    &\quad\: + \left(u_0 + \frac{T - t +1}{2}\right)\log\left[ \frac{1}{2}\sum_{t'=t}^T (y_{t'} - \overline{y}_{t:T})^2 \right]  - \left(u_0 + \frac{T - t_0 +1}{2}\right)\log\left[\frac{1}{2}\sum_{t'=t_0}^T (y_{t'} - \overline{y}_{t_0:T})^2 \right] \\
    &\quad\: + \mathcal{O}(T^{-1}) .
\end{align*}
Since $y_{t'} \sim \mathcal{N}(b_0,s_0^2)$ for each $t \geq t_0$, we can standardize the random terms to get:
\begin{align*}
    \sum_{t' = t_0}^{t - 1} y_{t'}^2 &= \sum_{t' = t_0}^{t - 1} (s_0z_{t'} + b_0)^2 \\
    &= s_0^2 \sum_{t' = t_0}^{t - 1} z_{t'}^2 + 2 b_0 s_0 \sum_{t' = t_0}^{t - 1} z_{t'} + (t-t_0)b_0^2
\end{align*}
and for each $t \geq t_0$:
\begin{align*}
    \sum_{t'=t}^T (y_{t'} - \overline{y}_{t:T})^2 &= \sum_{t'=t}^T (y_{t'} - b_0 + b_0 - \overline{y}_{t:T})^2 \\
    &= s_0^2\sum_{t'=t}^T (z_{t'} - \overline{z}_{t:T})^2 
\end{align*}
so we now have:
\begin{align*}
    \log \frac{\Delta_{t_0}}{\Delta_t} &= \log \left(\frac{T - t + 1}{T - t_0 +1} \right) + \frac{s_0^2}{2} \sum_{t' = t_0}^{t - 1} z_{t'}^2 + b_0 s_0 \sum_{t' = t_0}^{t - 1} z_{t'} + \left(\frac{t-t_0}{2}\right)b_0^2 \\
    &\quad\: + \left(\frac{T - t_0 + 1}{2}\right) \log\left(\frac{T-t_0+1}{2}\right) - \left(\frac{T - t + 1}{2}\right) \log\left(\frac{T-t+1}{2}\right) - \frac{t-t_0}{2} \\
    &\quad\: + \left(u_0 + \frac{T - t +1}{2}\right)\log\left[ \frac{s_0^2}{2}\sum_{t'=t}^T (z_{t'} - \overline{z}_{t:T})^2 \right]  - \left(u_0 + \frac{T - t_0 +1}{2}\right)\log\left[\frac{s_0^2}{2}\sum_{t'=t_0}^T (z_{t'} - \overline{z}_{t_0:T})^2 \right] \\
    &\quad\: + \mathcal{O}(T^{-1}).
\end{align*}
Next, since $T-t+1 > cT$ and $T-t_0+1 < T$, we have:
\begin{align*}
    \log \left(\frac{T - t_0 + 1}{T - t+1} \right) < \log \frac{1}{c} 
\end{align*}
which gives: 
\begin{align}
    \log \frac{\Delta_{t_0}}{\Delta_t} &> \log c + \frac{s_0^2}{2} \sum_{t' = t_0}^{t - 1} z_{t'}^2 + b_0 s_0 \sum_{t' = t_0}^{t - 1} z_{t'} + \left(\frac{t-t_0}{2}\right)b_0^2 \notag \\
    &\quad\: + \left(\frac{T - t_0+1}{2}\right) \log\left(\frac{T-t_0+1}{2}\right) - \left(\frac{T - t+1}{2}\right) \log\left(\frac{T-t+1}{2}\right) - \frac{t-t_0}{2} \notag\\
    &\quad\: + \left(u_0 + \frac{T - t +1}{2}\right)\log\left[ \frac{s_0^2}{2}\sum_{t'=t}^T (z_{t'} - \overline{z}_{t:T})^2 \right]  - \left(u_0 + \frac{T - t_0 +1}{2}\right)\log\left[\frac{s_0^2}{2}\sum_{t'=t_0}^T (z_{t'} - \overline{z}_{t_0:T})^2 \right] \notag\\
    &\quad\: + \mathcal{O}(T^{-1}). \label{eq:thm3-cs1-bd1}
\end{align}
Note that since $z_t \overset{\text{i.i.d.}}{\sim} \mathcal{N}(0,1)$, then we have:
\begin{align*}
    \sum_{t'=t_0}^{t-1} z^2_{t'} &\sim \chi^2_{t-t_0} \\
    \sum_{t'=t_0}^T (z_{t'} - \overline{z}_{t_0:T})^2 &\sim \chi^2_{T-t_0} \\
    \sum_{t'=t}^T (z_{t'} - \overline{z}_{t:T})^2 &\sim \chi^2_{T-t}. 
\end{align*}
By the Chernoff bound we get:
\begin{align}
    \mathbb{P}\left(\left|\sum_{t'=t_0}^{t-1} z_{t'}\right| \geq 2\sqrt{T \log T}\right) &\leq 2\exp\left[-\frac{2 T\log T}{t-t_0}\right] \notag \\
    &<  \frac{2}{T^2} \label{thm3-cs1-E1-inter}
\end{align}
where (\ref{thm3-cs1-E1-inter}) follows from the fact that $T > t-t_0$. Now define the event:
\begin{align*}
    \mathcal{E}_1 &:= \left\{\left|\sum_{t'=t_0}^{t-1} z_{t'}\right| < 2\sqrt{T \log T}, \; \sforall t \text{ s.t. } t > t_0 + \sqrt{T\log^{1+\varepsilon}T} \text{ and } t < (1-c)T\right\}.
\end{align*}
Then:
\begin{align*}
    \mathbb{P}(\mathcal{E}^c_1) &= \mathbb{P}\left(\bigcup_{t\;:\; t_0 + \sqrt{T\log^{1+\varepsilon}T} < t < (1-c)T} \left\{\left|\sum_{t'=t_0}^{t-1} z_{t'}\right| \geq 2\sqrt{T \log T}\right\}\right) \\
    &\leq \sum_{t\;:\; t_0 + \sqrt{T\log^{1+\varepsilon}T} < t < (1-c)T} \mathbb{P}\left(\left|\sum_{t'=t_0}^{t-1} z_{t'}\right| \geq 2\sqrt{T \log T}\right) \tag{union bound}\\
    &\leq \sum_{t\;:\; t_0 + \sqrt{T\log^{1+\varepsilon}T} < t < (1-c)T} \frac{2}{T^2} \tag{\ref{thm3-cs1-E1-inter}} \\
    &\leq \frac{2(1-2c)}{T}
\end{align*}
where in the last line we have used the fact that $(1-c)T > t > t_0 > cT$ to bound the number of terms in the sum. Similarly, by the $\chi^2$-concentration inequality (\ref{eq:chi2-ineq}):
\begin{align}
    \mathbb{P}\left(\left|\sum_{t'=t_0}^{t-1} z^2_{t'} - (t -t_0)\right| \geq 8\sqrt{T \log T}\right) &< 2 \exp\left[- 
    \frac{8 T \log T}{t-t_0} \right] \notag\\
    &<  \frac{2}{T^8} \label{thm3-cs1-E2-inter}\\
    \mathbb{P}\left(\left|\sum_{t'=t}^T (z_{t'} - \overline{z}_{t:T})^2 - (T-t)\right| \geq 8\sqrt{T \log T}\right) &\leq 2 \exp\left[-\frac{8 T \log T}{T-t} \right] \notag \\
    &<  \frac{2}{T^8} \label{thm3-cs1-E3-inter}
\end{align}
where again (\ref{thm3-cs1-E2-inter}) and (\ref{thm3-cs1-E3-inter}) follow from the fact that $T > T - t$ and $T > t-t_0$. So, if we define the events:
\begin{align*}
    \mathcal{E}_2 &:= \left\{\left|\sum_{t'=t_0}^{t-1} z^2_{t'} - (t -t_0)\right| < 8\sqrt{T \log T}, \; \sforall t \text{ s.t. } t > t_0 + \sqrt{T\log^{1+\varepsilon}T} \text{ and } t < (1-c)T\right\} \\
    \mathcal{E}_3 &:= \left\{\left|\sum_{t'=t}^T (z_{t'} - \overline{z}_{t:T})^2 - (T-t)\right| < 8\sqrt{T \log T}, \; \sforall t \text{ s.t. } t \geq t_0 \text{ and } t < (1-c)T\right\}
\end{align*}
then:
\begin{align*}
    \mathbb{P}(\mathcal{E}^c_2) &= \mathbb{P}\left(\bigcup_{t\;:\; t_0 + \sqrt{T\log^{1+\varepsilon}T} < t < (1-c)T} \left\{\left|\sum_{t'=t_0}^{t-1} z^2_{t'} - (t -t_0)\right| \geq 8\sqrt{T \log T}\right\}\right) \\
    &\leq \sum_{t\;:\; t_0 + \sqrt{T\log^{1+\varepsilon}T} < t < (1-c)T} \mathbb{P}\left(\left|\sum_{t'=t_0}^{t-1} z^2_{t'} - (t -t_0)\right| \geq 8\sqrt{T \log T}\right) \tag{union bound}\\
    &\leq \sum_{t\;:\; t_0 + \sqrt{T\log^{1+\varepsilon}T} < t < (1-c)T} \frac{2}{T^8} \tag{\ref{thm3-cs1-E2-inter}} \\
    &\leq \frac{2(1-2c)}{T^7}
\end{align*}
and:
\begin{align*}
    \mathbb{P}(\mathcal{E}^c_3) &= \mathbb{P}\left(\bigcup_{t\;:\; t_0 \leq t < (1-c)T} \left\{\left|\sum_{t'=t}^T (z_{t'} - \overline{z}_{t:T})^2 - (T-t)\right| \geq 8\sqrt{T \log T}\right\}\right) \\
    &\leq \sum_{t\;:\; t_0 \leq t < (1-c)T} \mathbb{P}\left(\left|\sum_{t'=t}^T (z_{t'} - \overline{z}_{t:T})^2 - (T-t)\right| \geq 8\sqrt{T \log T}\right) \tag{union bound}\\
    &\leq \sum_{t\;:\; t_0 \leq t < (1-c)T} \frac{2}{T^8} \tag{\ref{thm3-cs1-E3-inter}}\\
    &\leq \frac{2(1-2c)}{T^7}.
\end{align*}
If we now define the joint event $\mathcal{E} := \mathcal{E}_1 \cap \mathcal{E}_2 \cap \mathcal{E}_3$, then:
\begin{align*}
    \mathbb{P}(\mathcal{E}) &= 1 - \mathbb{P}(\mathcal{E}^c_1 \cup \mathcal{E}^c_2 \cup \mathcal{E}^c_3) \\
    &\geq 1 - \mathbb{P}(\mathcal{E}^c_1 ) - \mathbb{P}(\mathcal{E}^c_2) - \mathbb{P}(\mathcal{E}^c_3)  \tag{union bound} \\
    &> 1 - 2(1-2c)\left(\frac{1}{T} + \frac{2}{T^7}\right).
\end{align*}
So $\lim_{T\to\infty} \mathbb{P}(\mathcal{E})  = 1$. Now suppose that we are on the event $\mathcal{E}$, then we have:
\begin{align*}
    \sum_{t'=t}^T (z_{t'} - \overline{z}_{t:T})^2 &= \sum_{t'=t}^T (z_{t'} - \overline{z}_{t:T})^2 - (T-t) + (T-t) \\
    &> T-t - 8 \sqrt{T \log T}. 
\end{align*}
Since $T-t > cT$, for large enough $T$ we have $T-t - 8 \sqrt{T \log T} > 0$, so taking the log is well-defined for large $T$, and thus:
\begin{align}
    \left(u_0 + \frac{T - t +1}{2}\right)\log\left[ \frac{s_0^2}{2}\sum_{t'=t}^T (z_{t'} - \overline{z}_{t:T})^2 \right]  > \left(u_0 + \frac{T - t +1}{2}\right)\log\left[ \frac{s_0^2}{2} \left(T - t - 8\sqrt{T \log T}\right)\right]. \label{eq:thm3-cs1-bd2}
\end{align}
Similarly, on $\mathcal{E}$ we have:
\begin{align*}
    \sum_{t'=t_0}^T (z_{t'} - \overline{z}_{t_0:T})^2 &= \sum_{t'=t_0}^T (z_{t'} - \overline{z}_{t_0:T})^2 - (T-t_0) + (T-t_0) \\
    &< T-t_0 + 8 \sqrt{T \log T}  
\end{align*}
and thus:
\begin{align}
    -\left(u_0 + \frac{T - t_0 +1}{2}\right)\log\left[ \frac{s_0^2}{2}\sum_{t'=t_0}^T (z_{t'} - \overline{z}_{t_0:T})^2 \right]  > - \left(u_0 + \frac{T - t_0 +1}{2}\right)\log\left[ \frac{s_0^2}{2} \left(T - t_0 + 8\sqrt{T \log T}\right)\right]. \label{eq:thm3-cs1-bd3}
\end{align}
Combining the bounds (\ref{eq:thm3-cs1-bd2}) and (\ref{eq:thm3-cs1-bd3}) with (\ref{eq:thm3-cs1-bd1}), we get the deterministic bound:
\begin{align*}
    \log \frac{\Delta_{t_0}}{\Delta_t} &> \left(\frac{t-t_0}{2}\right)b_0^2\\
    &\quad\: + \log c + \frac{s_0^2}{2}\left(t - t_0 - 8 \sqrt{T\log T}\right) - 2 |b_0| s_0 \sqrt{T \log T} \\
    &\quad\: + \left(\frac{T - t_0 + 1}{2}\right) \log\left(\frac{T-t_0+1}{2}\right) - \left(\frac{T - t+1}{2}\right) \log\left(\frac{T-t+1}{2}\right) - \frac{t-t_0}{2} \\
    &\quad\: + \left(u_0 + \frac{T - t +1}{2}\right)\log\left[ \frac{s_0^2}{2} \left(T - t - 8\sqrt{T \log T}\right)\right] \\
    &\quad\: - \left(u_0 + \frac{T - t_0 +1}{2}\right)\log\left[\frac{s_0^2}{2}\left(T-t_0 + 8\sqrt{T \log T}\right) \right] \\
    &\quad\: + \mathcal{O}(T^{-1}).
\end{align*}
Collecting like terms, we have:
\begin{align*}
    \log \frac{\Delta_{t_0}}{\Delta_t} &> \left(\frac{t-t_0}{2}\right)(b_0^2 + s_0^2 - \log s_0^2 - 1) \\
    &\quad\: + \log c - 2 (|b_0|s_0 +2s_0^2) \sqrt{T \log T} \\
    &\quad\: + \left(\frac{T - t+1}{2}\right)\log\left(1 - \frac{1 + 8\sqrt{T \log T}}{T-t+1}\right)\\
    &\quad\: - \left(\frac{T - t_0+1}{2}\right)\log\left(1 + \frac{-1 + 8\sqrt{T \log T}}{T-t_0+1}\right)  \\
    &\quad\: + u_0\log\left(T - t - 8\sqrt{T \log T}\right) \\
    &\quad\: - u_0\log\left(T-t_0 + 8\sqrt{T \log T}\right) \\
    &\quad\: + \mathcal{O}(T^{-1}).
\end{align*}
Using the bounds $T - t_0 < (1-c)T$ and $T-t > cT$, we get the bounds:
\begin{align*}
    T-t - 8 \sqrt{T\log T} &> cT - 8 \sqrt{T\log T} \\
    T-t_0 + 8 \sqrt{T\log T} &< (1-c)T + 8 \sqrt{T\log T} 
\end{align*}
and thus:
\begin{align*}
    \log \frac{\Delta_{t_0}}{\Delta_t} &> \left(\frac{t-t_0}{2}\right)(b_0^2 + s_0^2 - \log s_0^2 - 1) \\
    &\quad\: + \log c - 2 (|b_0|s_0 +2s_0^2) \sqrt{T \log T} \\
    &\quad\: + \left(\frac{T - t+1}{2}\right)\log\left(1 - \frac{1 + 8\sqrt{T \log T}}{T-t+1}\right)\\
    &\quad\: - \left(\frac{T - t_0+1}{2}\right)\log\left(1 + \frac{-1 + 8\sqrt{T \log T}}{T-t_0+1}\right)  \\
    &\quad\: + u_0\log \frac{cT - 8\sqrt{T \log T}}{(1-c)T + 8\sqrt{T \log T}} \\
    &\quad\: + \mathcal{O}(T^{-1}).
\end{align*}
Since $\lim_{T\to\infty} T^{-1} \sqrt{T\log T} = 0$, for large enough $T$ we have: 
\begin{align*}
    \left|u_0\log \frac{cT - 8\sqrt{T \log T}}{(1-c)T + 8\sqrt{T \log T}} - u_0\log \frac{c}{(1-c)}\right| < \frac{1}{2}
\end{align*}
and thus:
\begin{align*}
    \log \frac{\Delta_{t_0}}{\Delta_t} &> \left(\frac{t-t_0}{2}\right)(b_0^2 + s_0^2 - \log s_0^2 - 1) \\
    &\quad\: + \log c - 2 (|b_0|s_0 +2s_0^2) \sqrt{T \log T} \\
    &\quad\: + \left(\frac{T - t+1}{2}\right)\log\left(1 - \frac{1 + 8\sqrt{T \log T}}{T-t+1}\right)\\
    &\quad\: - \left(\frac{T - t_0+1}{2}\right)\log\left(1 + \frac{-1 + 8\sqrt{T \log T}}{T-t_0+1}\right)  \\
    &\quad\: + u_0\log \frac{c}{(1-c)} - \frac{1}{2} \\
    &\quad\: + \mathcal{O}(T^{-1}).
\end{align*}
Next, since $T-t_0 > 0$, we have:
\begin{align*}
    \frac{-1 + 8\sqrt{T \log T}}{T-t_0+1} > -1
\end{align*}
and since $\log(1+x) \leq x$ for $x > -1$, we have:
\begin{align*}
    \left(\frac{T - t_0 + 1}{2}\right)\log\left(1 + \frac{-1 + 8\sqrt{T \log T}}{T-t_0+1}\right)  &\leq \left(\frac{T - t_0+1}{2}\right) \left(\frac{-1 + 8\sqrt{T \log T}}{T-t_0+1}\right) \\
    &= \frac{-1 + 8\sqrt{T \log T}}{2}.
\end{align*}
And thus:
\begin{align}
    \log \frac{\Delta_{t_0}}{\Delta_t} &> \left(\frac{t-t_0}{2}\right)(b_0^2 + s_0^2 - \log s_0^2 - 1) \notag \\
    &\quad\: - 2 (|b_0|s_0 +2s_0^2 + 2) \sqrt{T \log T}  \notag\\
    &\quad\: + \left(\frac{T - t+1}{2}\right)\log\left(1 - \frac{1 + 8\sqrt{T \log T}}{T-t+1}\right)  \notag\\
    &\quad\: + u_0\log \frac{c}{(1-c)} + \log c \notag \\
    &\quad\: + \mathcal{O}(T^{-1}). \label{eq:thm3:cs1-bd4}
\end{align}
Next, since $T-t + 1 > cT$, for large $T$ we have:
\begin{align*}
    \lim_{T\to \infty} \frac{1 + 8\sqrt{T \log T}}{T-t+1} = 0.
\end{align*}
Using a first order Taylor expansion of $\log(1-x)$ around zero we have $\log(1-x) = -x + \mathcal{O}(x^2)$, and thus:
\begin{align*}
    \left(\frac{T - t+1}{2}\right)\log\left(1 - \frac{1 + 8\sqrt{T \log T}}{T-t+1}\right) &= \left(\frac{T - t+1}{2}\right)\left[-\frac{1 + 8\sqrt{T \log T}}{T-t+1} + \mathcal{O} \left(\frac{\log T}{T}\right)\right] \\
    &= -\frac{1}{2} - 4 \sqrt{T\log T} + \mathcal{O}(\log T) \tag{$T-t+1 = \mathcal{O}(T)$}
\end{align*}
where the $\mathcal{O} \left(T^{-1}\log T\right)$ term in the first line comes from the fact that:
\begin{align*}
    \frac{(1 + 8\sqrt{T \log T})^2}{(T-t+1)^2} &\leq \frac{1 + 16\sqrt{T\log T} + 64 T\log T}{c^2T^2} \tag{$T-t+1 > cT$} \\
    &= \mathcal{O} \left(\frac{\log T}{T}\right).
\end{align*}
The lower bound in (\ref{eq:thm3:cs1-bd4}) now simplifies to: 
\begin{align*}
    \log \frac{\Delta_{t_0}}{\Delta_t} &> \left(\frac{t-t_0}{2}\right)(b_0^2 + s_0^2 - \log s_0^2 - 1) \\
    &\quad\: - 2 \left(|b_0|s_0 +2s_0^2 + 4\right) \sqrt{T \log T} \\
    &\quad\: + u_0\log \frac{c}{(1-c)} + \log c - \frac{1}{2} \\
    &\quad\: + \mathcal{O}(\log T) + \mathcal{O}(T^{-1}) \\
    &= \left(\frac{t-t_0}{2}\right)(b_0^2 + s_0^2 - \log s_0^2 - 1) \\
    &\quad\: - 2 \left(|b_0|s_0 +2s_0^2 + 4\right) \sqrt{T \log T} \\
    &\quad\: + \mathcal{O}(\log T).
\end{align*}
Define the upper bound terms:
\begin{align*}
    M_b &:= \mathbbm{1}\{b_0 = 0\} + \overline{b}\mathbbm{1}\{|b_0| > 0\} \\
    M_v &:= \mathbbm{1}\{s_0 = 1\} + \overline{s}\mathbbm{1}\{s_0 \neq 1\} 
\end{align*}
where $\overline{b}$ is the upper bound on $|b_0|$ from Assumption \ref{assumption:1} (ii) and $\overline{s}^2 = \sup I_1$, where $I_1 \subset (1, \infty)$ is the interval defined in Assumption \ref{assumption:1} (iii). Then, if there is no mean change and $b_0 =0$ or if there is a mean change and Assumption \ref{assumption:1} (ii) holds, we have $|b_0| \leq M_b$, and similarly if there is no variance change and $s_0 =1$ or if there is a variance change and Assumption \ref{assumption:1} (iii) holds, we have $s_0 \leq M_v$. Using these bounds we get:
\begin{align*}
    \log \frac{\Delta_{t_0}}{\Delta_t} &> \left(\frac{t-t_0}{2}\right)(b_0^2 + s_0^2 - \log s_0^2 - 1) \\
    &\quad\: - 2 \left(M_bM_v +2M_v^2 + 4\right) \sqrt{T \log T} \\
    &\quad\: + \mathcal{O}(\log T).
\end{align*}
Now, using the assumption that there is some $\varepsilon >0$ so that $t-t_0 > \sqrt{T\log^{1+\varepsilon} T}$, as well as the fact that $x - \log x \geq 1$ for all $x > 0$, we get: 
\begin{align*}
    \log \frac{\Delta_{t_0}}{\Delta_t} &>  \left(\frac{b_0^2 + s_0^2 - \log s_0^2 - 1}{2}\right)\sqrt{T\log^{1+\varepsilon} T} \\
    &\quad\: - 2 \left(M_bM_v +2M_v^2 + 4\right) \sqrt{T \log T} \\
    &\quad\: + \mathcal{O}(\log T).
\end{align*}
If Assumption \ref{assumption:1} (ii) holds, then for some $\varphi \in (0, \varepsilon / 2)$, we have:
\begin{align*}
    b_0^2 \geq \log^{-\varphi} T
\end{align*}
and thus: 
\begin{align*}
    \log \frac{\Delta_{t_0}}{\Delta_t} &>  \frac{1}{2}\sqrt{T\log^{1+\varepsilon - 2\varphi} T} \\
    &\quad\: - 2 \left(M_bM_v +2M_v^2 + 4 \right) \sqrt{T \log T} \\
    &\quad\: + \mathcal{O}(\log T).
\end{align*}
So if we set $\delta = \varepsilon - 2\varphi > 0$, we get the desired result:
\begin{align*}
    \log \frac{\Delta_{t_0}}{\Delta_t} \gtrsim \sqrt{T\log^{1+\delta} T}.
\end{align*}
Similarly, if if Assumption \ref{assumption:1} (iii) holds, then for some intervals $I_1 \subset (1,\infty)$ and $I_2 \subset (0,1)$, we either have:
\begin{align*}
   s_0^2 > \underline{s}^2_1 = \inf I_1  > 1
\end{align*}
or:
\begin{align*}
   s_0^2 < \overline{s}^2_2 = \sup I_2 < 1.
\end{align*}
Define:
\begin{align*}
    \alpha := \min\{\underline{s}_1^2 - \log \underline{s}_1^2 - 1, \overline{s}_2^2 - \log \overline{s}_2^2 - 1\}
\end{align*}
then noting that $x - \log x - 1$ is a convex function with a unique minimum at $x = 1$, we have
\begin{align*}
    s_0^2 - \log s_0^2 - 1 &\geq \alpha > 0.
\end{align*}
So we have:
\begin{align*}
    \log \frac{\Delta_{t_0}}{\Delta_t} &>  \frac{\alpha}{2}\sqrt{T\log^{1+\varepsilon} T} \\
    &\quad\: - 2 \left(M_bM_v +2M_v^2 + 2 \right) \sqrt{T \log T} \\
    &\quad\: + \mathcal{O}(\log T)
\end{align*}
and thus: 
\begin{align*}
    \log \frac{\Delta_{t_0}}{\Delta_t} \gtrsim \sqrt{T\log^{1+\varepsilon} T}.
\end{align*}

\subsubsection*{Case 2: $t < t_0$ and $\min\{t, T-t\} > cT$.}

From (\ref{eq:thm3-delta_t}) we again have:
\begin{align*}
    \log \frac{\Delta_{t_0}}{\Delta_t} &= \log \left(\frac{T - t + 1}{T - t_0 +1} \right) + \frac{1}{2} \sum_{t' = t}^{t_0 - 1} y_{t'}^2 \\
    &\quad\: + \left(\frac{T - t_0 + 1}{2}\right) \log\left(\frac{T-t_0+1}{2}\right) - \left(\frac{T - t + 1}{2}\right) \log\left(\frac{T-t+1}{2}\right) + \frac{t_0-t}{2} \\
    &\quad\: + \left(u_0 + \frac{T - t +1}{2}\right)\log\left[ \frac{1}{2}\sum_{t'=t}^T (y_{t'} - \overline{y}_{t:T})^2 \right]  - \left(u_0 + \frac{T - t_0 +1}{2}\right)\log\left[\frac{1}{2}\sum_{t'=t_0}^T (y_{t'} - \overline{y}_{t_0:T})^2 \right] \\
    &\quad\: + \mathcal{O}(T^{-1}).
\end{align*}
Standardizing the random terms gives:
\begin{align*}
    \sum_{t' = t}^{t_0 - 1} y_{t'}^2 &= \sum_{t' = t}^{t_0 - 1} z_{t'}^2 \\
    \sum_{t'=t_0}^T (y_{t'} - \overline{y}_{t_0:T})^2 &= s_0^2 \sum_{t'=t_0}^T  (z_{t'} - \overline{z}_{t_0:T})^2
\end{align*}
and thus:
\begin{align*}
    \log \frac{\Delta_{t_0}}{\Delta_t} &= \log \left(\frac{T - t + 1}{T - t_0 + 1} \right) - \frac{1}{2} \sum_{t' = t}^{t_0 - 1} z_{t'}^2 \\
    &\quad\: + \left(\frac{T - t_0 + 1}{2}\right) \log\left(\frac{T-t_0+1}{2}\right) - \left(\frac{T - t + 1}{2}\right) \log\left(\frac{T-t+1}{2}\right) + \frac{(t_0-t)}{2} \\
    &\quad\: + \left(u_0 + \frac{T - t +1}{2}\right)\log\left[ \frac{1}{2}\sum_{t'=t}^T (y_{t'} - \overline{y}_{t:T})^2 \right]- \left(u_0 + \frac{T - t_0 +1}{2}\right)\log\left[\frac{s_0^2}{2}\sum_{t'=t_0}^T  (z_{t'} - \overline{z}_{t_0:T})^2 \right] \\
    &\quad\: +  \mathcal{O}(T^{-1}) 
\end{align*}
Since $t < t_0$ implies $T-t + 1> T-t_0 +1$, then:
\begin{align*}
    \log\frac{T-t+1}{T-t_0+1} > 0
\end{align*}
so we have:
\begin{align*}
    \log \frac{\Delta_{t_0}}{\Delta_t} &> - \frac{1}{2} \sum_{t' = t}^{t_0 - 1} z_{t'}^2  + \left(\frac{T - t_0+1}{2}\right) \log\left(\frac{T-t_0+1}{2}\right) - \left(\frac{T - t+1}{2}\right) \log\left(\frac{T-t+1}{2}\right) + \frac{(t_0-t)}{2} \\
    &\quad\: + \left(u_0 + \frac{T - t +1}{2}\right)\log\left[ \frac{1}{2}\sum_{t'=t}^T (y_{t'} - \overline{y}_{t:T})^2 \right] - \left(u_0 + \frac{T - t_0 +1}{2}\right)\log\left[\frac{s_0^2}{2}\sum_{t'=t_0}^T  (z_{t'} - \overline{z}_{t_0:T})^2 \right] \\
    &\quad\: +  \mathcal{O}(T^{-1}).
\end{align*}
If we factor out a $\log s_0^2$ term from the second line of this lower bound, then we have:
\begin{align}
    \log \frac{\Delta_{t_0}}{\Delta_t} &> - \frac{1}{2} \sum_{t' = t}^{t_0 - 1} z_{t'}^2 + \frac{(t_0-t)}{2} + \frac{(t_0-t)}{2}\log s_0^2 \notag \\
    &\quad\: + \left(\frac{T - t_0 + 1}{2}\right) \log\left(\frac{T-t_0+1}{2}\right) - \left(\frac{T - t + 1}{2}\right) \log\left(\frac{T-t+1}{2}\right) \notag\\
    &\quad\: + \left(u_0 + \frac{T - t +1}{2}\right)\log\left[ \frac{1}{2s_0^2}\sum_{t'=t}^T (y_{t'} - \overline{y}_{t:T})^2 \right] \notag\\
    &\quad\: - \left(u_0 + \frac{T - t_0 +1}{2}\right)\log\left[\frac{1}{2}\sum_{t'=t_0}^T  (z_{t'} - \overline{z}_{t_0:T})^2 \right] \notag\\
    &\quad\: +  \mathcal{O}(T^{-1}). \label{eq:thm3-cs2-bd1}
\end{align}
Again, since $z_{t} \overset{\text{i.i.d.}}{\sim} \mathcal{N}(0,1)$, we have:
\begin{align*}
    \sum_{t'=t}^{t_0-1} z^2_{t'} &\sim \chi^2_{t_0-t} \\
    \sum_{t'=t}^{t_0-1} (z_{t'} - \overline{z}_{t:(t_0-1)})^2 &\sim \chi^2_{t_0-t-1} \\
    \sum_{t'=t_0}^T (z_{t'} - \overline{z}_{t_0:T})^2 &\sim \chi^2_{T-t_0}.
\end{align*}
Then by the $\chi^2$-concentration inequality (\ref{eq:chi2-ineq}):
\begin{align}
    \mathbb{P}\left(\left|\sum_{t'=t}^{t_0-1} z^2_{t'} - (t_0 -t)\right| \geq 8\sqrt{T \log T}\right) &< 2 \exp\left[- 
    \frac{8 T \log T}{t-t_0} \right] \notag \\
    &<  \frac{2}{T^8} \label{eq:thm3-O1-inter} \\
    \mathbb{P}\left(\left|\sum_{t'=t}^{t_0-1} (z_{t'} - \overline{z}_{t:(t_0-1)})^2- (t_0 -t - 1)\right| \geq 8\sqrt{T \log T}\right) &< 2 \exp\left[- 
    \frac{8 T \log T}{t-t_0-1} \right] \notag \\
    &< \frac{2}{T^8} \label{eq:thm3-O2-inter} \\
    \mathbb{P}\left(\left|\sum_{t'=t_0}^T (z_{t'} - \overline{z}_{t:T})^2 - (T-t)\right| \geq 8\sqrt{T\log T}\right)&< 2 \exp\left[- \frac{8 T \log T}{T-t} \right]\notag \\
    &<  \frac{2}{T^8}. \label{eq:thm3-O3-inter}
\end{align}
Where in each case above we have used the fact that $T > T -t$ and $T > t - t_0$ to get the upper bounds in (\ref{eq:thm3-O1-inter})-(\ref{eq:thm3-O3-inter}). So, if we define the events:
\begin{align*}
    \Omega_1 &:= \left\{\left|\sum_{t'=t}^{t_0-1} z^2_{t'} - (t_0 - t)\right| < 8\sqrt{T \log T}, \; \sforall t \text{ s.t. } t < t_0 - \sqrt{T\log^{1+\varepsilon}T} \text{ and } t > cT\right\} \\
    \Omega_2 &:= \left\{\left|\sum_{t'=t}^{t_0-1} (z_{t'} - \overline{z}_{t:(t_0-1)})^2- (t_0 -t - 1)\right| < 8\sqrt{T \log T}, \; \sforall t \text{ s.t. } t < t_0 - \sqrt{T\log^{1+\varepsilon}T} \text{ and } t > cT\right\} \\
    \Omega_3 &:= \left\{\left|\sum_{t'=t}^T (z_{t'} - \overline{z}_{t:T})^2 - (T-t)\right| < 8\sqrt{T \log T}, \sforall t \leq t_0\right\}.
\end{align*}
Then we have:
\begin{align*}
    \mathbb{P}(\Omega^c_1) &= \mathbb{P}\left(\bigcup_{t\;:\; cT < t < t_0 - \sqrt{T\log^{1+\varepsilon}T}} \left\{\left|\sum_{t'=t}^{t_0-1} z^2_{t'} - (t_0 -t)\right| \geq 8\sqrt{T \log T}\right\}\right) \\
    &\leq \sum_{t\;:\; cT < t < t_0 - \sqrt{T\log^{1+\varepsilon}T}} \mathbb{P}\left(\left|\sum_{t'=t}^{t_0-1} z^2_{t'} - (t_0 -t)\right| \geq 8\sqrt{T \log T}\right) \tag{union bound}\\
    &\leq \sum_{t\;:\; cT < t < t_0 - \sqrt{T\log^{1+\varepsilon}T}} \frac{2}{T^8} \tag{\ref{eq:thm3-O1-inter}} \\
    &\leq \frac{2(1-2c)}{T^7}
\end{align*}
where in the last line we have used $cT< t < t_0 <(1-c)T$ to bound the number of terms in the sum. An identical argument using (\ref{eq:thm3-O2-inter}) gives $P(\Omega_2^c) \leq 2(1-2c)T^{-7}$, and:
\begin{align*}
    \mathbb{P}(\Omega^c_3) &= \mathbb{P}\left(\bigcup_{t=1}^{t_0} \left\{\left|\sum_{t'=t}^T (z_{t'} - \overline{z}_{t:T})^2 - (T-t)\right| < 8\sqrt{T \log T}\right\}\right) \\
    &\leq \sum_{t=1}^{t_0} \mathbb{P}\left(\left|\sum_{t'=t}^T (z_{t'} - \overline{z}_{t:T})^2 - (T-t)\right| < 8\sqrt{T \log T}\right) \tag{union bound}\\
    &\leq \sum_{t=1}^{t_0} \frac{2}{T^8} \tag{\ref{eq:thm3-O3-inter}} \\
    &\leq \frac{2(1-c)}{T^7} \tag{$t < t_0 < (1-c)T$}
\end{align*}
So if we define the joint event $\Omega := \bigcap_{i=1}^3 \Omega_i$, then:
\begin{align*}
    \mathbb{P}(\Omega) &= 1 - \mathbb{P}(\Omega^c_1 \cup \Omega^c_2 \cup \Omega^c_3) \\
    &\geq 1 - \mathbb{P}(\Omega^c_1) - \mathbb{P}(\Omega^c_2) - \mathbb{P}(\Omega^c_3) \tag{union bound} \\
    &> 1 - \frac{4(1-2c)}{T^7} - \frac{2(1-c)}{T^7}.
\end{align*}
So $\lim_{T\to\infty} \mathbb{P}(\Omega)  = 1$. Suppose that we are on $\Omega$, then:
\begin{align}
     - \frac{1}{2} \sum_{t' = t}^{t_0 - 1} z_{t'}^2 &= - \frac{(t_0 - t)}{2} + \frac{(t_0 - t)}{2} - \frac{1}{2} \sum_{t' = t}^{t_0 - 1} z_{t'}^2 \notag \\
     &> - \frac{(t_0 - t)}{2} - 4\sqrt{T\log T} \label{eq:thm3-cs2-bd2}
\end{align}
and:
\begin{align}
    \sum_{t'=t_0}^T  (z_{t'} - \overline{z}_{t_0:T})^2 &= \sum_{t'=t_0}^T  (z_{t'} - \overline{z}_{t_0:T})^2 - (T - t_0) + (T-t_0) \notag \\
    &< 8\sqrt{T \log T} + (T - t_0). \label{eq:thm3-cs2-bd3}
\end{align}
Combining the bounds (\ref{eq:thm3-cs2-bd2}) and (\ref{eq:thm3-cs2-bd3}) with (\ref{eq:thm3-cs2-bd1}), we have:
\begin{align*}
    \log \frac{\Delta_{t_0}}{\Delta_t} &> -4\sqrt{T \log T} + \frac{(t_0 - t)\log s_0^2}{2} \\
    &\quad\: + \left(\frac{T - t_0 + 1}{2}\right) \log\left(\frac{T-t_0+1}{2}\right) - \left(\frac{T - t + 1}{2}\right) \log\left(\frac{T-t+1}{2}\right) \\
    &\quad\: + \left(u_0 + \frac{T - t +1}{2}\right)\log\left[ \frac{1}{2s_0^2}\sum_{t'=t}^T (y_{t'} - \overline{y}_{t:T})^2 \right] \\
    &\quad\: - \left(u_0 + \frac{T - t_0 +1}{2}\right)\log\left[\frac{1}{2}(T-t_0 +8\sqrt{T \log T})\right] \\
    &\quad\: +  \mathcal{O}(T^{-1}). 
\end{align*}
Focusing on just the terms:
\begin{align*}
    \left(\frac{T - t_0 + 1}{2}\right) \log\left(\frac{T-t_0+1}{2}\right) - \left(u_0 + \frac{T - t_0 +1}{2}\right)\log\left[\frac{1}{2}(T-t_0 +8\sqrt{T \log T})\right]
\end{align*}
we can rewrite these as we did in the case of $t > t_0$ to get:
\begin{align*}
    - \left(\frac{T - t_0 + 1}{2}\right)\log\left(1 + \frac{-1 + 8\sqrt{T \log T}}{T-t_0+1}\right) - u_0\log\left[\frac{1}{2}(T-t_0 +8\sqrt{T \log T})\right].
\end{align*}
Again, since $T- t_0 + 1 > 0$, for all $T$ we have:
\begin{align*}
    \frac{-1 + 8\sqrt{T \log T}}{T-t_0+1} > -1
\end{align*}
and since $\log(1+x) \leq x$ for $x > -1$, we have:
\begin{align*}
    \left(\frac{T - t_0 + 1}{2}\right)\log\left(1 + \frac{-1 + 8\sqrt{T \log T}}{T-t_0+1}\right)  &\leq \left(\frac{T - t_0 + 1}{2}\right) \left(\frac{-1 + 8\sqrt{T \log T}}{T-t_0+1}\right) \\
    &= \frac{-1 + 8\sqrt{T \log T}}{2}.
\end{align*}
Combining this bound with the fact that $T-t_0 < (1-c)T$, we get:
\begin{align}
    \log \frac{\Delta_{t_0}}{\Delta_t} &> -4\sqrt{T \log T} + \frac{(t_0 - t)\log s_0^2}{2} \notag \\
    &\quad\: - \left(\frac{T - t + 1}{2}\right) \log\left(\frac{T-t+1}{2}\right) + \left(u_0 + \frac{T - t +1}{2}\right)\log\left[ \frac{1}{2s_0^2}\sum_{t'=t}^T (y_{t'} - \overline{y}_{t:T})^2 \right] \notag \\
    &\quad\: + \frac{1}{2} - 4 \sqrt{T \log T} - u_0\log\left[\frac{(1-c)T}{2} + 4\sqrt{T\log T}\right] \notag \\
    &\quad\: +  \mathcal{O}(T^{-1}) \label{eq:thm3-cs2-bd4}. 
\end{align}
We now proceed in two steps. First, we show that if Assumption \ref{assumption:1} (iii) holds (in addition to Assumption \ref{assumption:1} (i), (iv), and (v) (b)), then for any value of $b_0$, there is some $\delta > 0$ such that: $$\log \frac{\Delta_{t_0}}{\Delta_t} \gtrsim \sqrt{T\log^{1+\delta} T}.$$ Second, if $s_0^2$ is a fixed constant and Assumption \ref{assumption:1} (iii) does not hold, then it must be the case that $s_0^2 = 1$ since since otherwise we would be able to construct some intervals $I_1\subset(1,\infty)$ and $I_2\subset(0,1)$ such that $s_0^2 \in I_1\cup I_2$. Therefore, the next step is to set $s_0^2 = 1$ and show that if Assumption \ref{assumption:1} (ii) holds (in addition to in addition to Assumption \ref{assumption:1} (i), (iv), and (v) (b)), then there is some $\delta > 0$ such that: $$\log \frac{\Delta_{t_0}}{\Delta_t} \gtrsim \sqrt{T\log^{1+\delta} T}.$$

Focusing now on the remaining random term in the lower bound (\ref{eq:thm3-cs2-bd4}), we have:
\begin{align*}
    \sum_{t'=t}^T (y_{t'} - \overline{y}_{t:T})^2 &= \sum_{t'=t}^{t_0-1 } (y_{t'} - \overline{y}_{t:T})^2 + \sum_{t'=t_0}^T (y_{t'} - \overline{y}_{t:T})^2 \\
    &= \sum_{t'=t}^{t_0-1 } (y_{t'} - \overline{y}_{t:(t_0-1)} + \overline{y}_{t:(t_0-1)} - \overline{y}_{t:T})^2 + \sum_{t'=t_0}^T (y_{t'} - \overline{y}_{t_0:T} + \overline{y}_{t_0:T} - \overline{y}_{t:T})^2 \\
    &= \sum_{t'=t}^{t_0-1 } (y_{t'} - \overline{y}_{t:(t_0-1)})^2 + \sum_{t'=t_0}^T (y_{t'} - \overline{y}_{t_0:T})^2 \\
    &\quad + (t_0-t)(\overline{y}_{t:(t_0-1)} - \overline{y}_{t:T})^2  +  (T-t_0+1)(\overline{y}_{t:T} - \overline{y}_{t_0:T})^2 \\
    &> \sum_{t'=t}^{t_0-1 } (y_{t'} - \overline{y}_{t:(t_0-1)})^2 + \sum_{t'=t_0}^T (y_{t'} - \overline{y}_{t_0:T})^2 \\
    &= \sum_{t'=t}^{t_0-1 } (z_{t'} - \overline{z}_{t:(t_0-1)})^2 + s_0^2\sum_{t'=t_0}^T (z_{t'} - \overline{z}_{t_0:T})^2 
\end{align*}
Furthermore, on $\Omega$ we have:
\begin{align*}
    \sum_{t'=t}^{t_0-1 } (z_{t'} - \overline{z}_{t:(t_0-1)})^2 + s_0^2\sum_{t'=t_0}^T (z_{t'} - \overline{z}_{t_0:T})^2 &=  \sum_{t'=t}^{t_0-1 } (z_{t'} - \overline{z}_{t:(t_0-1)})^2 -(t_0 - t -1) + (t_0 - t -1)  \\
    &\quad + s_0^2\sum_{t'=t_0}^T (z_{t'} - \overline{z}_{t_0:T})^2 -s_0^2(T-t_0) + s_0^2(T-t_0) \\
    &> t_0 - t -1 - 8 \sqrt{T\log T} + s_0^2(T-t_0) - 8 s_0^2 \sqrt{T \log T} \\
    &= (1 - s_0^2)(t_0 - t) -1 - 8 \sqrt{T\log T} + s_0^2(T-t) - 8 s_0^2 \sqrt{T \log T}.
\end{align*}
Therefore, on $\Omega$:
\begin{align*}
    \log\left[ \frac{1}{s_0^2(T-t+1)}\sum_{t'=t}^T (y_{t'} - \overline{y}_{t:T})^2 \right] &> \log\left[1 + \frac{\left(s_0^{-2} -1\right)(t_0 - t) - \left(s_0^{-2} + 1\right)\left(8\sqrt{T\log T} +1\right)}{T - t + 1}  \right].
\end{align*}
Let $\underline{s}^2 = \inf I_2$, where $I_2 \subset (0,1)$ is the interval defined in Assumption \ref{assumption:1} (iii). Then $s_0^2 > \underline{s}^2$, and thus:
\begin{align*}
     0 < \frac{\left(s_0^{-2} + 1\right)\left(8\sqrt{T\log T} +1\right)}{T - t + 1} &<  \frac{(\underline{s}^{-2} + 1)\left(8\sqrt{T\log T} +1\right)}{T - t + 1} \\
     &< \frac{(\underline{s}^{-2} + 1)\left(8\sqrt{T\log T} +1\right)}{cT}. \tag{$T-t+1 > cT$}  
\end{align*}
Returning to the lower bound in (\ref{eq:thm3-cs2-bd4}), we now have:
\begin{align*}
    \log \frac{\Delta_{t_0}}{\Delta_t} &> \frac{1}{2}-8\sqrt{T \log T} + \frac{(t_0 - t)\log s_0^2}{2} \\
    &\quad\: + u_0 \log\left[ (s_0^{-2}-1)\frac{(t_0 - t)}{2} -\frac{1}{2s_0^2} - \frac{4}{s_0^2} \sqrt{T\log T} + \frac{cT}{s_0^2} - 4 \sqrt{T \log T}\right] \\
    &\quad\: + \left(\frac{T - t + 1}{2}\right)\log\left[1 + \frac{\left(s_0^{-2} -1\right)(t_0 - t)}{T - t + 1} -  \frac{(\underline{s}^{-2} + 1)\left(8\sqrt{T\log T} +1\right)}{cT}\right]  \\
    &\quad\: - u_0\log\left[\frac{(1-c)T}{2} + 4\sqrt{T\log T}\right] \\
    &\quad\: +  \mathcal{O}(T^{-1}). 
\end{align*}
Note that in the second and fourth lines, the terms inside the logarithm are dominated by the $cT$ and $(1-c)T$ terms respectively, so the second and fourth line are both $\mathcal{O}(\log T)$, and we can absorb these terms into the $\sqrt{T \log T}$ terms to get: 
\begin{align}
    \log \frac{\Delta_{t_0}}{\Delta_t} &> \frac{(t_0 - t)\log s_0^2}{2} + \left(\frac{T - t + 1}{2}\right)\log\left[1 + \frac{\left(s_0^{-2} -1\right)(t_0 - t)}{T - t + 1} -  \frac{(\underline{s}^{-2} + 1)\left(8\sqrt{T\log T} +1\right)}{cT}\right] \notag \\
    &\quad\: + \mathcal{O}(\sqrt{T \log T}). \label{eq:thm3-cs2-bd5} 
\end{align}
We now proceed to show that for any $\xi > 0$, for large enough $T$ we have:
\begin{align}
    \log \frac{\Delta_{t_0}}{\Delta_t} &> \frac{(t_0 - t)}{2} \left(\log s_0^2 + \frac{\log[1 +  (s_0^{-2} -1)(1-c)]}{1-c} - \xi\right)  + \mathcal{O}(\sqrt{T \log T}). \label{eq:thm3-cs2-bd6} 
\end{align}
To establish (\ref{eq:thm3-cs2-bd6}), we consider the cases of $s_0^2 > 1$ and $s_0^2 < 1$ separately:

\textbf{Case 2.a: $s_0^2 < 1$.}

For some $\alpha > 0$, consider the function $\log(1 + z)$ on the interval $[\alpha c/2, \alpha (1-c)]$, where $c$ is the same constant from Assumption $\ref{assumption:1}$ (i). Because $\log(1 + z)$ is continuous for $z > -1$, it is uniformly continuous on the closed and bounded interval $[\alpha c/2, \alpha (1-c)]$. Therefore, for any $\xi > 0$, there is an $\epsilon > 0$ such that:
\begin{align*}
     \log(1 + z - \epsilon) > \log(1 + z) - \frac{c \xi}{2}
\end{align*}
Making the substitution $z = \alpha x$ with $x \in [c/2, (1-c)]$ and by dividing by $x$, we get:
\begin{align*}
     \frac{\log(1 + \alpha x - \epsilon)}{x} &> \frac{\log(1 + \alpha x) - \frac{c \xi}{2}}{x} \\
     &> \frac{\log(1 + \alpha x)}{x} - \xi. \tag{$x \geq c/2$}
\end{align*}
Consider the function $f: (0,1) \to \mathbb{R}^+$:
\begin{align*}
    f(x) := \frac{\log(1 + \alpha x)}{x}.
\end{align*}
Using the inequality $\log(1+x) > \frac{x}{1+x}$ for all $x > - 1$, we get:
\begin{align*}
    f'(x) &= -\frac{\log(1 + \alpha x)}{x^2} + \frac{\alpha}{x(1 + \alpha x)} \\
    &< -\frac{\alpha x }{x^2(1+\alpha x)} + \frac{\alpha}{x(1 + \alpha x)} \\
    &= 0.
\end{align*}
So $f$ is strictly decreasing on $[c/2, (1-c)]$, and we have:
\begin{align*}
     \frac{\log(1 + \alpha x - \epsilon)}{x} &> \frac{\log(1 + \alpha x)}{x} - \xi \\
     &\geq \frac{\log[1 + \alpha (1-c)]}{1-c} - \xi.
\end{align*}
Consider the term in (\ref{eq:thm3-cs2-bd5}): 
\begin{align*}
     \frac{T-t+1}{t-t_0}\log\left[1 + \frac{\left(s_0^{-2} -1\right)(t_0 - t)}{T - t + 1} -  \frac{(\underline{s}^{-2} + 1)\left(8\sqrt{T\log T} +1\right)}{cT}\right].
\end{align*}
Because we have:
\begin{align*}
    \lim_{T \to \infty}  \frac{(\underline{s}^{-2} + 1)\left(8\sqrt{T\log T} +1\right)}{cT} = 0
\end{align*}
then using the same $\epsilon$ implied by our choice of $\xi$ above, for large enough $T$ we have:
\begin{align*}
     \log\left[1 + \frac{\left(s_0^{-2} -1\right)(t_0 - t)}{T - t + 1} -  \frac{(\underline{s}^{-2} + 1)\left(8\sqrt{T\log T} +1\right)}{cT}\right] > \log\left[1 + \frac{\left(s_0^{-2} -1\right)(t_0 - t)}{T - t + 1} - \epsilon\right].
\end{align*}
Furthermore, we have that $\min\{t_0,T-t_0\} > cT$ implies: 
\begin{align*}
    \frac{t_0 - t}{T - t + 1} \in [c/2, (1-c)].
\end{align*}
So if we let $x = \frac{t_0 - t}{T - t + 1}$ and $\alpha = s_0^{-2} -1$, noting that $s_0^{2}  \in (0,1)$ implies $\alpha > 0$, then by our previous analysis of the function $f$, for large enough $T$ we have:
\begin{align*}
    \frac{T-t+1}{t-t_0}\log\left[1 + \frac{\left(s_0^{-2} -1\right)(t_0 - t)}{T - t + 1} - \epsilon\right] &> \frac{\log[1 +  (s_0^{-2} -1)(1-c)]}{1-c} - \xi.
\end{align*}
Returning to our lower bound in (\ref{eq:thm3-cs2-bd5}), we have established (\ref{eq:thm3-cs2-bd6}):
\begin{align*}
    \log \frac{\Delta_{t_0}}{\Delta_t} &> \frac{(t_0 - t)\log s_0^2}{2} + \left(\frac{T - t + 1}{2}\right)\log\left[1 + \frac{\left(s_0^{-2} -1\right)(t_0 - t)}{T - t + 1} - \epsilon\right] + \mathcal{O}(\sqrt{T \log T}) \\
    &> \frac{(t_0 - t)}{2} \left(\log s_0^2 + \frac{\log[1 +  (s_0^{-2} -1)(1-c)]}{1-c} - \xi\right)  + \mathcal{O}(\sqrt{T \log T}).
\end{align*}

\textbf{Case 2.b: $s_0^2 > 1$.}

For some $\alpha \in (0,1)$, consider the function $\log(1 - z)$ on the interval $[\alpha c/2, \alpha (1-c)]$, where $c$ is the same constant from Assumption $\ref{assumption:1}$ (i). Because $\log(1 - z)$ is continuous for $z < 1$, it is uniformly continuous on the closed and bounded interval $[\alpha c/2, \alpha (1-c)]$. Therefore, for any $\xi > 0$, there is an $\epsilon > 0$ such that:
\begin{align*}
     \log(1 - z - \epsilon) > \log(1 - z) - \frac{c \xi}{2}.
\end{align*}
Making the substitution $z = \alpha x$ with $x \in [c/2, (1-c)]$ and by dividing by $x$, we get:
\begin{align*}
     \frac{\log(1 - \alpha x - \epsilon)}{x} &> \frac{\log(1 - \alpha x) -\frac{c \xi}{2}}{x} \\
     &> \frac{\log(1 - \alpha x)}{x} - \xi. \tag{$x \geq c/2$}
\end{align*}
Consider the function $f: (0,1) \to \mathbb{R}^-$:
\begin{align*}
    f(x) := \frac{\log(1 - \alpha x)}{x}.
\end{align*}
Using the inequality $\log(1+x) > \frac{x}{1+x}$ for all $x > - 1$ and the fact that $-\alpha x > -1$, we get:
\begin{align*}
    f'(x) &= -\frac{\log(1 - \alpha x)}{x^2} - \frac{\alpha}{x(1 - \alpha x)} \\
    &< \frac{\alpha x }{x^2(1+\alpha x)} - \frac{\alpha}{x(1 + \alpha x)} \\
    &= 0.
\end{align*}
So $f$ is strictly decreasing on $[c/2, (1-c)]$, and we have:
\begin{align*}
     \frac{\log(1 - \alpha x - \epsilon)}{x} &> \frac{\log(1 - \alpha x)}{x} - \xi \\
     &\geq \frac{\log[1 - \alpha (1-c)]}{1-c} - \xi.
\end{align*}
Using the same argument as in Case 2.a above, for large enough $T$ we have:
\begin{align*}
     \log\left[1 - \frac{\left(1- s_0^{-2}\right)(t_0 - t)}{T - t + 1} -  \frac{(\underline{s}^{-2} + 1)\left(8\sqrt{T\log T} +1\right)}{cT}\right] > \log\left[1 + \frac{\left(s_0^{-2} -1\right)(t_0 - t)}{T - t + 1} - \epsilon\right].
\end{align*}
Once again $\min\{t_0,T-t_0\} > cT$ implies that: 
\begin{align*}
    \frac{t_0 - t}{T - t + 1} \in [c/2, (1-c)].
\end{align*}
So if we let $x = \frac{t_0 - t}{T - t + 1}$ and $\alpha = 1 - s_0^{-2}$, noting that $s_0^{2} \in (1,\infty)$ implies $\alpha \in (0,1)$, then by our previous analysis of the function $f$, for large enough $T$ we have:
\begin{align*}
    \frac{T-t+1}{t-t_0}\log\left[1 - \frac{\left(1- s_0^{-2}\right)(t_0 - t)}{T - t + 1} - \epsilon\right] &> \frac{\log[1 -  (1 -s_0^{-2})(1-c)]}{1-c} - \xi.
\end{align*}
Returning to our lower bound in (\ref{eq:thm3-cs2-bd5}), we have established (\ref{eq:thm3-cs2-bd6}):
\begin{align*}
    \log \frac{\Delta_{t_0}}{\Delta_t} &> \frac{(t_0 - t)\log s_0^2}{2} + \left(\frac{T - t + 1}{2}\right)\log\left[1 + \frac{\left(s_0^{-2} -1\right)(t_0 - t)}{T - t + 1} - \epsilon\right] + \mathcal{O}(\sqrt{T \log T}) \\
    &> \frac{(t_0 - t)}{2} \left(\log s_0^2 + \frac{\log[1 -  (1 - s_0^{-2})(1-c)]}{1-c} - \xi\right)  + \mathcal{O}(\sqrt{T \log T}).
\end{align*}

With (\ref{eq:thm3-cs2-bd6}) established for both $s_0^2 < 1$ and $s_0^2 > 1$, examining this bound we see that if we can now show:
\begin{align}
    \log s_0^2 + \frac{\log[1 +  (s_0^{-2} -1)(1-c)]}{1-c} - \xi > 0 \label{eq:thm3-cs2-finalbd}
\end{align}
then because $|t_0-t| \geq \sqrt{T \log^{1+\varepsilon} T}$ for some $\varepsilon > 0$, we will have shown the desired result:
\begin{align*}
    \log \frac{\Delta_{t_0}}{\Delta_t} \gtrsim \sqrt{T \log^{1+\varepsilon} T}.
\end{align*}
To prove (\ref{eq:thm3-cs2-finalbd}), consider the function $g: (0,1) \to \mathbb{R}^+$:
\begin{align*}
    g(x) := (1-\alpha)x^\alpha + \alpha x^{\alpha - 1}.
\end{align*}
where $\alpha \in (1/2,1)$. Note that:
\begin{align*}
    g'(x) &= \alpha(1-\alpha)x^{\alpha-1}\left(1 - \frac{1}{x}
    \right).
\end{align*}
Because $1 - x^{-1} < 0$ for all $x \in (0,1)$, $g$ is strictly decreasing on $(0,1)$. Similarly, $1 - x^{-1} > 0$ for all $x > 1$, so $g$ is strictly increasing on $(1,\infty)$. Then because $g(1) = 1$, we get that $g(x) > 1$ for all $x \in (0,1)\cup(1,\infty)$. Furthermore, if we set $\alpha = 1 - c$, then:
\begin{align*}
    \log s_0^2 + \frac{\log[1 +  (s_0^{-2} -1)(1-c)]}{1-c} = \alpha^{-1}\log g(s_0^2).
\end{align*}
Now let $\underline{s}^2_2 = \inf I_1$ and $\overline{s}^2_2 = \sup I_2$ and $\overline{s}^2_2 = \sup I_2$ $\overline{s}^2_2 = \sup I_2$, where $I_1 \subset (1,\infty)$ and $I_2 \subset (0,1)$ are the intervals defined in Assumption (\ref{assumption:1}) (iii). Then by our previous analysis of $g$ we have:
\begin{align*}
    \alpha^{-1}\log g(s_0^2) &> \alpha^{-1}\min\{\log g(\underline{s}^2_1), \log g(\overline{s}^2_2)\} > 0.
\end{align*}
Our original choice of $\xi$ was arbitrary, so we can establish (\ref{eq:thm3-cs2-finalbd}) by picking $\xi < \alpha^{-1}\min\{\log g(\underline{s}^2_1), \log g(\overline{s}^2_2)\}.$

We have now shown that if Assumption \ref{assumption:1} (iii) holds, then:
\begin{align*}
    \log \frac{\Delta_{t_0}}{\Delta_t} \gtrsim \sqrt{T \log^{1+\varepsilon} T}
\end{align*}
regardless of whether or not Assumption \ref{assumption:1} (ii) holds. Assume now that $s_0^2 = 1$ and that Assumption \ref{assumption:1} (ii) does in fact hold. 

When $s_0^2 = 1$, we can simplify the lower bound in (\ref{eq:thm3-cs2-bd4}) to:
\begin{align}
    \log \frac{\Delta_{t_0}}{\Delta_t} &> - \left(\frac{T - t + 1}{2}\right) \log\left(\frac{T-t+1}{2}\right) + \left(u_0 + \frac{T - t +1}{2}\right)\log\left[ \frac{1}{2}\sum_{t'=t}^T (y_{t'} - \overline{y}_{t:T})^2 \right] \notag \\
    &\quad\: +  \mathcal{O}(\sqrt{T \log T}). \label{eq:thm3-cs2-bd7}
\end{align}
Focusing again on the remaining random term, when $s_0 = 1$, we can write:
\small
\begin{align}
    \sum_{t'=t}^T (y_{t'} - \overline{y}_{t:T})^2 &= \sum_{t'=t}^T y^2_{t'} - \frac{1}{T-t+1}\left(\sum_{t'=t}^T y_{t'} \right)^2 \notag \\
    &= \sum_{t'=t}^{t_0-1} z^2_{t'} + \sum_{t'=t_0}^T (y_{t'} - b_0 + b_0)^2 - \frac{1}{T-t+1}\left(\sum_{t'=t}^{t_0-1} z_{t'} + \sum_{t'=t_0}^T (y_{t'} - b_0 + b_0)\right)^2 \notag \\
    &= \sum_{t'=t}^T z^2_{t'} + 2b_0\sum_{t'=t_0}^T z_{t'} + (T - t_0 + 1)b_0^2 \notag \\
    &\quad - \frac{1}{T-t+1}\left(\sum_{t'=t}^T z_{t'} \right)^2  - \frac{2b_0(T-t_0+1)}{T-t+1}\sum_{t'=t}^T z_{t'} - \frac{(T-t_0+1)^2b_0^2}{T-t+1} \notag \\
    &=  \sum_{t'=t}^T (z_{t'} - \overline{z}_{t:T})^2 + 2b_0(t_0-t)\left(\frac{1}{T-t+1} \sum_{t'=t}^T z_{t'} - \frac{1}{t_0-t}\sum_{t' = t}^{t_0 - 1} z_{t'}\right) + \frac{(T-t_0+1)(t_0-t)b_0^2}{T-t+1} \notag \\
    &> \sum_{t'=t}^T (z_{t'} - \overline{z}_{t:T})^2 - 2|b_0|(t_0-t)\left[\frac{1}{T-t+1} \left|\sum_{t'=t}^T z_{t'}\right| + \frac{1}{t_0-t}\left|\sum_{t' = t}^{t_0 - 1} z_{t'}\right|\right] + (t_0-t)cb_0^2 \label{eq:thm3-cs2-bd8}
\end{align}
\normalsize
where in the last line we have bounded the last term by using the fact that $T-t+1 < T$ and $T-t_0 + 1 > cT$. Next, if Assumption \ref{assumption:1} (ii) holds, then there is some $\varphi \in (0, \varepsilon/2)$ so that $|b_0| \geq \log^{-\varphi/2}T$, and by the Chernoff bound we have:
\begin{align*}
    \mathbb{P}\left(\left|\sum_{t'=t}^{t_0-1} z_{t'}\right| > \frac{(t_0-t) c|b_0|}{8}\right) &\leq \mathbb{P}\left(\left|\sum_{t'=t_0}^{t-1} z_{t'}\right| > \frac{(t_0-t) c\log^{-\varphi/2}T}{8}\right) \tag{$|b_0| \geq \log^{-\varphi/2}T$} \\
    &\leq 2 \exp\left[- \frac{(t_0-t)c^2\log^{-\varphi}T}{128}\right]. \tag{Chernoff Bound}
\end{align*}
If we assume that $t_0-t > \frac{256}{c^2}\sqrt{T\log^{1+\varepsilon}T}$, then since $\frac{256}{c^2}\sqrt{T\log^{1+\varepsilon-2\varphi}T} > \frac{256}{c^2}\log T$ the last line implies that:
\begin{align}
    \mathbb{P}\left(\left|\sum_{t'=t}^{t_0-1} z_{t'}\right| > \frac{(t_0-t) c|b_0|}{8}\right) &\leq \frac{2}{T^2}. \label{eq:thm3-O4-inter}
\end{align}
An identical argument plus the fact that $T - t + 1 > t_0 - t$ gives:
\begin{align}
    \mathbb{P}\left(\left|\sum_{t'=t}^T z_{t'}\right| > \frac{(T-t + 1) c|b_0|}{8}\right) &\leq \frac{2}{T^2}. \label{eq:thm3-O5-inter}
\end{align}
So if we define two additional events:
\begin{align*}
    \Omega_4 &= \left\{\frac{1}{t_0-t}\left|\sum_{t'=t}^{t_0-1} z_{t'}\right| < \frac{c|b_0|}{8}, \; \sforall t \text{ s.t. } t < t_0 - c^{-2}256\sqrt{T\log^{1+\varepsilon}T} \text{ and } t > cT\right\}\\
    \Omega_5 &= \left\{\frac{1}{T-t+1}\left|\sum_{t'=t}^T z_{t'}\right| < \frac{c|b_0|}{8}, \; \sforall t \text{ s.t. } t < t_0 - c^{-2}256\sqrt{T\log^{1+\varepsilon}T} \text{ and } t > cT\right\}
\end{align*}
then we have:
\begin{align*}
    \mathbb{P}(\Omega^c_4) &= \mathbb{P}\left(\bigcup_{t\;:\; cT < t < t_0 - c^{-2}256\sqrt{T\log^{1+\varepsilon}T}} \left\{\left|\sum_{t'=t}^{t_0-1} z^2_{t'}\right| \geq \frac{c|b_0|}{8}\right\}\right) \\
    &\leq \sum_{t\;:\; cT < t < t_0 - c^{-2}256\sqrt{T\log^{1+\varepsilon}T}} \mathbb{P}\left(\left|\sum_{t'=t}^{t_0-1} z^2_{t'}\right| \geq \frac{c|b_0|}{8}\right) \tag{union bound}\\
    &\leq \sum_{t\;:\; cT < t < t_0 - c^{-2}256\sqrt{T\log^{1+\varepsilon}T}} \frac{2}{T^2} \tag{\ref{eq:thm3-O4-inter}} \\
    &\leq \frac{2(1-2c)}{T}. \tag{$cT < t < t_0 < (1-c)T$}
\end{align*}
Therefore, $\lim_{T\to \infty} \mathbb{P}(\Omega^c_4) = 0$, and an identical argument using (\ref{eq:thm3-O5-inter}) gives $\lim_{T\to \infty} \mathbb{P}(\Omega^c_5) = 0$, so, if we redefine $\Omega = \cap_{i=1}^5 \Omega_i$, then by the union bound we still have $\lim_{T \to \infty} \mathbb{P}(\Omega) = 1$. Since we still have $\Omega_3 \subset \Omega$, then: 
\begin{align*}
    \sum_{t'=t}^T (z_{t'} - \overline{z}_{t:T})^2 > T-t +1 - (1 + 8\sqrt{T \log T})
\end{align*}
and thus on $\Omega$ we can use the lower bound in (\ref{eq:thm3-cs2-bd7}) to write:
\begin{align*}
    \sum_{t'=t}^T (y_{t'} - \overline{y}_{t:T})^2 &> \sum_{t'=t}^T (z_{t'} - \overline{z}_{t:T})^2 - 2|b_0|(t_0-t)\left[\frac{1}{T-t+1} \left|\sum_{t'=t}^T z_{t'}\right| + \frac{1}{t_0-t}\left|\sum_{t' = t}^{t_0 - 1} z_{t'}\right|\right] + (t_0-t)cb_0^2 \\
    &> T-t +1 - (1 + 8\sqrt{T \log T}) - 2|b_0|(t_0-t)\left[\frac{c|b_0|}{8} + \frac{c|b_0|}{8}\right] + (t_0-t)cb_0^2 \\
    &= T-t +1 - (1 + 8\sqrt{T \log T}) + (t_0-t)\frac{cb_0^2}{2}.
\end{align*}
Therefore, on $\Omega$ the lower bound (\ref{eq:thm3-cs2-bd8}) can be rewritten as:
\begin{align*}
    \log \frac{\Delta_{t_0}}{\Delta_t} &> - \left(\frac{T - t + 1}{2}\right) \log\left(\frac{T-t+1}{2}\right) + \left(u_0 + \frac{T - t +1}{2}\right)\log\left[ \frac{1}{2}\sum_{t'=t}^T (y_{t'} - \overline{y}_{t:T})^2 \right] \\
    &\quad\: +  \mathcal{O}(\sqrt{T \log T}) \\
    &> \left(\frac{T - t + 1}{2}\right) \log\left[1 + \frac{(t_0-t)\frac{cb_0^2}{2} - (1 + 8\sqrt{T \log T})}{T-t+1}\right] \\
    &\quad\: + u_0 \log \left[T-t +1 - (1 + 8\sqrt{T \log T}) + (t_0-t)\frac{cb_0^2}{2}\right] \\
    &\quad\: +  \mathcal{O}(\sqrt{T \log T}).
\end{align*}
Since $T-t > cT$ and $b_0^2$ is bounded by Assumption \ref{assumption:1} (ii), we have:
\begin{align*}
    u_0 \log \left[T-t +1 - (1 + 8\sqrt{T \log T}) + (t_0-t)\frac{cb_0^2}{2}\right] = \mathcal{O} (\log T). 
\end{align*}
Absorbing this term into the $\mathcal{O}(\sqrt{T \log T})$ term and using the fact that under Assumption $\ref{assumption:1}$ (ii) we have $(t_0 - t) b_0^2 > \sqrt{T \log^{1 +\varepsilon-2\varphi} T}$, we get:
\begin{align*}
    \log \frac{\Delta_{t_0}}{\Delta_t} &> \left(\frac{T - t + 1}{2}\right) \log\left[1 + \frac{\frac{c}{2}\sqrt{T \log^{1 +\varepsilon-2\varphi} T} - (1 + 8\sqrt{T \log T})}{T-t+1}\right] \\
    &\quad\: +  \mathcal{O}(\sqrt{T \log T}).
\end{align*}
Since $T-t > cT$, then: 
\begin{align*}
    \frac{\frac{c}{2}\sqrt{T \log^{1 +\varepsilon-2\varphi} T} - (1 + 8\sqrt{T \log T})}{T-t+1} = \mathcal{O}\left(\sqrt{\frac{\log^{1 + \varepsilon-2\varphi} T}{T}}\right).
\end{align*}
Therefore, using a first order Taylor expansion around zero to get $\log x = x + \mathcal{O}(x^2)$, we get:
\begin{align*}
     \left(\frac{T - t + 1}{2}\right) \log\left[1 + \frac{\frac{c}{2}\sqrt{T \log^{1 +\varepsilon-2\varphi} T} - (1 + 8\sqrt{T \log T})}{T-t+1}\right] &= \frac{c}{4}\sqrt{T \log^{1 +\varepsilon-2\varphi} T} - \left(\frac{1}{2} + 4\sqrt{T \log T}\right)\\
     & \quad + \left(\frac{T - t + 1}{2}\right)\mathcal{O}\left(\frac{\log^{1 + \varepsilon-\varphi} T}{T}\right) \\
     &= \frac{c}{4}\sqrt{T \log^{1 +\varepsilon-2\varphi} T} + \mathcal{O}(\sqrt{T \log T})
\end{align*}
and thus 
\begin{align*}
    \log \frac{\Delta_{t_0}}{\Delta_t} &> \frac{c}{4}\sqrt{T \log^{1 +\varepsilon-2\varphi} T} + \mathcal{O}(\sqrt{T \log T}).
\end{align*}
So if we set $\delta = \varepsilon - 2\varphi > 0$, we get the desired result:
\begin{align*}
    \log \frac{\Delta_{t_0}}{\Delta_t} \gtrsim \sqrt{T\log^{1+\delta} T}.
\end{align*}

Finally, combining the case of $t > t_0$ and $t < t_0$, again by the union bound have have:
\begin{align*}
    \mathbb{P}(\mathcal{E} \cap \Omega) = 1 - \mathbb{P}(\mathcal{E}^c) -\mathbb{P}(\Omega^c) \to 1
\end{align*}
and if we let $\kappa = c^{-2}256$, then on $\mathcal{E} \cap \Omega$ we have:
\begin{align*}
    \max_{t \;:\; |t - t_0| > \kappa \sqrt{T\log^{1+\varepsilon} T}} \; \mathbb{P}(\gamma = t  \;|\; \mathbf{y} ; \tau_0, u_0, s_0,\pmb{\pi}) < \mathbb{P}(\gamma = t_0  \;|\; \mathbf{y} ; \tau_0, u_0, s_0,\pmb{\pi}).
\end{align*}

\end{proof}





\bibliographystyle{chicago}
\bibliography{MICH/references}


\end{document}