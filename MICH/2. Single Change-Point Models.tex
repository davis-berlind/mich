\section{Single Change-Point (SCP) Models}
\label{sec:scp}

Suppose that we have $T$ observations of a univariate time-series $\mathbf{y} = \{y_t\}_{t=1}^{T}$ that is generated by the following process:
\begin{align}\label{eq:dgp}
    y_t \:|\: \mu_t, \lambda_t \overset{\text{ind.}}{\sim} \mathcal{N}\left(\mu_t, \lambda^{-1}_t\right), \;\sforall t \in [T].
\end{align}
We assume that either $\pmb{\mu} = \{\mu_t\}_{t=1}^{T}$, $\pmb{\lambda} = \{\lambda_t\}^{T}_{t=1}$, or both exhibit piece-wise constant structures with a single change-point occurring at some unknown time $\gamma \in [T]$.\footnote{In some cases it may make sense to restrict $\gamma$ to a subset of $[T]$, e.g. it is common in the CPD literature to assume that there is a buffer at the start and end of $\mathbf{y}$ within which no change $\pmb{\mu}$ or $\pmb{\lambda}$ can occur. We show how to include such a buffer in our model in Appendix \ref{app:buffer}.} In this section we introduce three single change-point (SCP) models for the random processes that generate $\gamma$ along with the jumps in $\pmb{\mu}$ and $\pmb{\lambda}$. In Section \ref{sec:smcp} and Section \ref{sec:sscp} we introduce models that handle the respective cases of either a single change-point in $\pmb{\mu}$ or a single change-point in $\pmb{\lambda}$, while the model in Section \ref{sec:smscp} addresses the setting of a simultaneous change in both $\pmb{\mu}$ and $\pmb{\lambda}$. In each case, by modeling $\gamma$ as a categorical random variable, and by choosing conditionally conjugate prior distributions for the jumps in $\pmb{\mu}$ and $\pmb{\lambda}$, we arrive at simple closed form posterior distributions for the location of $\gamma$. 

\subsection{Single Mean Change-Point (SMCP)}
\label{sec:smcp}

Suppose that we have $T$ observations from (\ref{eq:dgp}) and that $\pmb{\mu}$ is a random sequence with an underlying piece-wise constant structure generated by the following Single Mean Change-Point (SMCP) model:
\begin{align} \label{eq:smcp-start}
    \mu_t &= b\mathbbm{1}{\left\{t\geq \gamma \right\}} \\
    b &\sim \mathcal{N}(0,\tau_0^{-1}) \\
    \gamma &\sim \text{Categorical}(\pmb{\pi}),\; \pmb{\pi} \in \mathcal{S}^T \\
    b &\indep \gamma
    \label{eq:smcp-end}
\end{align}
where $\mathcal{S}^T$ is the $T$-dimensional probability simplex\footnote{I.e. $\pmb{\pi} \in \mathcal{S}^T$ implies $\pi_t \geq 0$ and $\sum_{t=1}^T \pi_t = 1$.} and $\pmb{\lambda}$, $\pmb{\pi}$, and $\tau_0$, are known constants. In words, $\mathbf{y}$ starts centered at zero and jumps by $b\in\mathbb{R}$ at some time $\gamma \in [T]$. For each $t \in [T]$, the posterior distribution $p(b, \gamma \:|\: \mathbf{y})$ for the SMCP model is given by:
\begin{align}
    b \:|\: \gamma = t, \: \mathbf{y} &\sim \mathcal{N}\left(\overline{b}_{t}, \overline{\tau}_{t}^{-1}\right) \label{eq:b-smcp} \\
    \gamma \:|\: \mathbf{y} &\sim \text{Categorical}(\overline{\pmb{\pi}}) \label{eq:gamma-smcp} \\
    \overline{\tau}_t &= \tau_0 + \sum_{t'=t}^{T} \lambda_{t'} \\
    \overline{b}_t &= \frac{1}{\overline{\tau}_t}\sum_{t'=t}^{T} \lambda_{t'} y_{t'} \\
    \overline{\pi}_t &\propto \pi_t\overline{\tau}^{-\frac{1}{2}}_t\exp\left(\overline{\tau}_t\overline{b}^2_t / 2\right).
\end{align}
Even though $b$ and $\gamma$ are generated independently, we see above that these parameters can exhibit arbitrary levels of correlation in the posterior distribution. We also note that the model described in (\ref{eq:smcp-start})-(\ref{eq:smcp-end}) is identical to the SER model introduced in \cite{Wang20} when the covariate matrix $\mathbf{X}$ is lower-triangular with the non-zero entries equal to one. Motivated by this connection, we define a function \texttt{SMCP} that is analogous to the \texttt{SER} function in \cite{Wang20}. \texttt{SMCP} takes $\mathbf{y}$ and the model parameters as inputs and returns the posterior parameters of $p(b, \gamma\:|\:\mathbf{y})$ as its output: 
\begin{align}\label{eq:smcp}
    \texttt{SMCP}\left(\mathbf{y} \:;\: \pmb{\lambda}, \tau_0, \pmb{\pi}\right) := \{\overline{b}_t, \overline{\tau}_t, \overline{\pi}_t\}_{t=1}^T.
\end{align}
When appropriate, we will also use the notation:
\begin{align}
    \{b,\gamma\} \sim \text{SMCP}(\{\overline{b}_t, \overline{\tau}_t, \overline{\pi}_t\}_{t=1}^T)
\end{align}
to mean that $b$ and $\gamma$ follow the distribution specified in (\ref{eq:b-smcp})-(\ref{eq:gamma-smcp}).

\subsection{Single Scale Change-Point (SSCP)}
\label{sec:sscp}

The scale change-point model in this section is precisely the same model introduced by \cite{Cappello22} with the addition a known sequence of precision parameters $\pmb{\tau}=\{\tau_t\}_{t=1}^{T}$. Again, we have $T$ observations from (\ref{eq:dgp}), but now we assume that $\pmb{\mu} \equiv \mathbf{0}$ and that the sequence of precision parameters $\pmb{\lambda}$ is random with an underlying piece-wise constant structure generated by the following Single Scale Change-Point (SSCP) model:
\begin{align}\label{eq:sscp-start}
    \lambda_t &= s^{\mathbbm{1}\{t \geq \gamma\}}\tau_t  \\
    s &\sim \text{Gamma}(u_0,v_0) \\
    \gamma &\sim \text{Categorical}(\pmb{\pi}),\; \pmb{\pi} \in \mathcal{S}^T \\
    s &\indep \gamma.
    \label{eq:sscp-end}
\end{align}
In words, the model in (\ref{eq:sscp-start})-(\ref{eq:sscp-end}) independently draws the location of the change-point $\gamma$ and the jump in the precision $s$, then scales each precision parameter $\tau_t$ by $s$ if $t \geq \gamma$. To see this, suppose that $\gamma > t$, i.e. $t$ is a time before the change-point occurs, then we will have $\mathbbm{1}\{t \geq \gamma\} = 0$ and $s^{\mathbbm{1}\{t \geq \gamma\}} = 1$. On the other hand, if $\gamma \leq t$, then we have $s^{\mathbbm{1}\{t \geq \gamma\}} = s$, so multiplying $\tau_t$ by $s^{\mathbbm{1}\{t \geq \gamma\}}$ gives us the desired scaling effect. \cite{Cappello22} show that posterior distribution $p(s, \gamma\:|\:\mathbf{y})$ for the SSCP model is given by:
\begin{align}
    s \:|\: \gamma = t, \: \mathbf{y} &\sim \text{Gamma}\left(\overline{u}_{t}, \overline{v}_{t}\right) \label{eq:s-sscp} \\
    \gamma \:|\: \mathbf{y}&\sim \text{Categorical}(\overline{\pmb{\pi}}) \label{eq:gamma-sscp}\\
    \overline{u}_{t} &= u_0 + \frac{T - t + 1}{2} \\
    \overline{v}_{t} &= v_0 + \frac{1}{2} \sum_{t'=t}^{T} \tau_{t'}y_{t'}^2 \\
    \overline{\pi}_t &\propto \pi_t \Gamma(\overline{u}_{t}) \overline{v}_{t}^{-\overline{u}_{t}}\exp\left(- \frac{1}{2}\sum_{t'=1}^{t-1} \tau_{t'}y_{t'}^2\right). \label{eq:pi-sscp}
\end{align}
Note that in (\ref{eq:pi-sscp}) the summation term $\sum_{t'=1}^{t-1} \tau_{t'}y_{t'}^2$ may be ill-defined if $t = 1$. Here and throughout the rest of this article we use the convention that a sum is just equal to zero if the indexing set is empty. We can also define a function \texttt{SSCP} that takes the response $\mathbf{y}$ and the model parameters as inputs and returns the posterior parameters of $p(s, \gamma\:|\:\mathbf{y})$ as its output: 
\begin{align}
    \texttt{SSCP}\left(\mathbf{y} \:;\: \pmb{\tau}, u_0, v_0, \pmb{\pi}\right) := \{\overline{u}_t, \overline{v}_t, \overline{\pi}_t\}_{t=1}^T.
\end{align}
As with the SMCP model, we introduce a shorthand for the distribution (\ref{eq:s-sscp})-(\ref{eq:gamma-sscp}): 
\begin{align}
    \{s,\gamma\} \sim \text{SSCP}(\{\overline{u}_t, \overline{v}_t, \overline{\pi}_t\}_{t=1}^T).
\end{align}

\subsection{Single Mean-Scale Change-Point (SMSCP)}
\label{sec:smscp}

We now merge the SMCP and SSCP settings by allowing $\mathbf{y}$ to have a single change-point where both the mean and scale shift simultaneously. We again assume that we have $T$ observations from (\ref{eq:dgp}), only now the concurrent change to the piece-wise constant structure of $\pmb{\mu}$ and $\pmb{\lambda}$ is generated by the following Single Mean-Scale Change-Point (SMSCP) model:
\begin{align}
    \label{eq:smscp-start}
    \mu_{t} &= b \mathbbm{1}{\{t \geq \gamma\}} \\
    \lambda_t &= \tau_t s^{\mathbbm{1}\{t\geq \gamma\}} \\
    b \:|\: s &\sim \mathcal{N}(0,(s \tau_0)^{-1}) \\
    s &\sim \text{Gamma}(u_0, v_0) \\
    \gamma &\sim \text{Categorical}(\pmb{\pi}),\;\pmb{\pi}\in \mathcal{S}^T \\
    \gamma &\indep \{b,s\} 
    \label{eq:smscp-end}
\end{align}
Here again the precision parameters $\pmb{\tau}$ are known constants. The posterior distribution for $p(b, s, \gamma \:|\: \mathbf{y})$ under the SMSCP setting is given by:
\begin{align}
    b \:|\: s, \gamma = t, \mathbf{y} &\sim \mathcal{N}(\overline{b}_t, (\overline{\tau}_t s)^{-1}) \label{eq:b-smscp} \\
    s \:|\: \gamma = t, \mathbf{y} &\sim \text{Gamma}(\overline{u}_t, \overline{v}_t) \\
    \gamma \:|\: \mathbf{y} &\sim \text{Categorical}(\overline{\pmb{\pi}}) \label{eq:gamma-smscp} \\
    \overline{\tau}_t &= \tau_0 + \sum_{t' = t}^{T} \tau_{t'}  \\
    \overline{b}_t &= \frac{1}{\overline{\tau}_t} \sum_{t'=t}^{T} \tau_{t'} y_{t'} \\
    \overline{u}_t &= u_0 + \frac{T - t + 1}{2} \\
    \overline{v}_t &= v_0 - \frac{\overline{\tau}_t\overline{b}^2_t}{2} + \frac{1}{2} \sum_{t'=t}^{T} \tau_{t'}y_{t'}^2 \\
    \overline{\pi}_t &\propto \pi_t\overline{\tau}_t^{-\frac{1}{2}} \Gamma(\overline{u}_t) \overline{v}_t^{-\overline{u}_t}\exp\left(- \frac{1}{2}\sum_{t'=1}^{t-1} \tau_{t'}y^2_{t'}\right).
\end{align}
As before, we define a function \texttt{SMSCP} that takes $\mathbf{y}$ and the model parameters as inputs and returns the posterior parameters of $p(b, s, \gamma\:|\:\mathbf{y})$ as its output:
\begin{align}
    \texttt{SMSCP}\left(\mathbf{y} \:;\: \pmb{\tau}, \tau_0, u_0, v_0, \pmb{\pi}\right) := \{\overline{b}_t, \overline{\tau}_t, \overline{u}_t, \overline{v}_t, \overline{\pi}_t\}_{t=1}^T
\end{align}
as well as a shorthand for the distribution (\ref{eq:b-smscp})-(\ref{eq:gamma-smscp})
\begin{align}
    \{b,s,\gamma\}\sim\text{SMSCP}(\{\overline{b}_t, \overline{\tau}_t, \overline{u}_t, \overline{v}_t, \overline{\pi}_t\}_{t=1}^T).
\end{align}

\subsection{Localization Rates}
\label{sec:localization}

Suppose that $t_0 \in [T]$ is the true location of the change-point for one of the SCP models. Then given $\overline{\pmb{\pi}} := \{\overline{\pi}_t\}_{t=1}^T$ from any of the SCP models, a natural estimator for $t_0$ is the posterior most probable location of $\gamma$, i.e. the maximum \textit{a posteriori} (MAP) estimator:
\begin{align}\label{eq:map}
    \hat{t}_{\text{MAP}} := \argmax{1 \leq t \leq T} \; \overline{\pi}_t.
\end{align}
In this section, we supplement the Bayesian perspective of the SCP models by studying the asymptotic behavior of $\hat{t}_{\text{MAP}}$. In particular, we show that under mild regularity conditions, $\hat{t}_{\text{MAP}}$ is consistent in the sense that there exists some error bound $\epsilon_T$, where:
\begin{align}
    \lim_{T\to\infty} \Pr\left(|t_0 - \hat{t}_{\text{MAP}}| \leq \epsilon_T\right) &= 1 \label{def:consistency} \\
    \lim_{T\to\infty} \frac{\epsilon_T}{T} &= 0. \label{def:loc-rate}
\end{align}
Throughout this article we refer to $\epsilon_T$ as the \textit{localization rate}. Our aim is to find the smallest localization rate that guarantees (\ref{def:consistency}) while satisfying (\ref{def:loc-rate}). We begin by making the following assumption:
\begin{assumption}\label{assumption:1}
    Let $t_0 \in [t]$ be the time such that $y_t \sim F_0$ for $t < t_0$ and $y_t \sim F_1$ for $t_0 \geq t$, where $F_0$ and $F_1$ are distributions such that:\vspace{-10pt}
    \begin{align}
        \E[y_t], \text{\normalfont Var}(y_t) &= 
        \begin{cases}
            (0,1), & \text{if } t < t_0, \\
            (b_0,s_0^2), & \text{if } t \geq t_0.
        \end{cases} 
    \end{align}
    Define the standardized terms:\vspace{-10pt}
    \begin{align} \label{eq:normalized}
    z_t = 
        \begin{cases}
            y_t, & \text{if } t < t_0, \\
            \frac{y_t - b_0}{s_0},& \text{if } t \geq t_0.
        \end{cases}
    \end{align}
    and assume that: \vspace{-10pt}
    \begin{enumerate}[label=(\roman*)]
        \item (Minimum Spacing) There exists a constant $c \in (0,1/2)$ such that $\min\{t_0,T-t_0\} > cT$. 
        \item (Proper Prior) The hyper-parameters in the SCP models are chosen so that $\tau_0, u_0, v_0 > 0$.
        \item (Bounded Prior) The prior distribution for the change-point location, $\pi_t := \Pr(\gamma = t \:; \pmb{\pi})$, is chosen so that for each $t\in[T]$, $\log \frac{\pi_{t_0}}{\pi_{t}}$ is of order: a) $\mathcal{O}(\log T)$, b) $\mathcal{O}(\sqrt{T\log T})$, or c) $\mathcal{O}(\sqrt{T}\log T)$.
    \end{enumerate}
\end{assumption}
In the context of multiple CPD, Assumption \ref{assumption:1} (i) can be interpreted as a minimum spacing condition, i.e. two consecutive change-points must be separated by an interval of minimum length $cT$. This is a standard assumption in Bayesian approaches to CPD, see for example \cite{Cappello22} and \cite{Cappello21}; however, it is notably stronger than other minimum spacing conditions in the literature, e.g. \cite{Wang2020_localization} show that if the minimum spacing condition is known and is of order $\log^{1+\varepsilon} T$ for some $\varepsilon > 0$, then both their $\ell_0$-penalty and WBS methods remain consistent. Assumption \ref{assumption:1} (ii) simply requires that the parameters in the SCP models are chosen so that the priors are non-degenerate, which will ensure the posteriors are proper. Lastly, Assumption \ref{assumption:1} (iii) ensures that our choice of $\pmb{\pi}$ does not end up overwhelming the evidence in the data. Note that this assumption is trivially satisfied if we set $\pi_t = T^{-1}$, i.e. each $t \in [T]$ is equally likely to be the location of the change-point \textit{a priori}. We leave a more detailed discussion of the choice of $\pmb{\pi}$ for Appendix \ref{app:prior}.

The conditions in Assumption \ref{assumption:1} will prove to be sufficient for detecting a change in $\pmb{\mu}$, provided that a change does in fact occur, i.e. $|b_0| > 0$. Assumption \ref{assumption:mean} formalizes the conditions under which the jump size $|b_0|$ is detectable by our model:
\begin{assumption}[Detectable Mean]\label{assumption:mean}  Let $\mathbf{y}$ be generated as in Assumption \ref{assumption:1} so that $b_0$ is the true change in the mean of $\mathbf{y}$ at time $t_0$. Assume that there exist some finite bounds $\underline{b}, \; \overline{b} > 0$ so that $\underline{b}\log^{-\varepsilon/2}T \leq |b_0| \leq \overline{b}$ for some $\varepsilon \geq 0$.
\end{assumption}

\begin{remark}\label{rmk:1}
   Assumption \ref{assumption:mean} allows for the possibility that $\varepsilon = 0$, in which case we will have $\underline{b} \leq |b_0| \leq \overline{b}$, i.e. $|b_0|$ lives in some compact interval that is bounded away from the origin. Alternatively, if $\varepsilon >0$ then $b_0$ is allowed to vanish as $T \to \infty$, so long as it does not happen at a rate faster that $\log^{-\varepsilon/2}T$.\footnote{Here and throughout this article we use the convention $\log^{a}T = (\log T)^{a}$.} In our proof of Theorem \ref{theorem:smcp}, this appears to be the fastest rate at which $b_0$ can approach zero while maintaining a localization rate of order $\log T$ using the SMCP model. 
\end{remark}

\begin{theorem}[SMCP Localization Rate]\label{theorem:smcp}
    Let $\mathbf{y}$ be generated according to (\ref{eq:dgp}) and suppose that Assumption \ref{assumption:1} (i), (ii), and (iii) (a) all hold with $s_0^2 = 1$. Additionally, if Assumption \ref{assumption:mean} holds for the constant $\varepsilon \geq 0$, then there exists some $\kappa > 0$ so that:
    \vspace{-5pt}
    \begin{align*}
        \lim_{T\to\infty}\Pr\left(|t_0 - \hat{t}_{\normalfont \text{MAP}}| \leq \kappa \log^{1+\varepsilon} T\right) = 1  
    \end{align*}
    where $\hat{t}_{\normalfont \text{MAP}}$ is constructed as in (\ref{eq:map}) by fitting the SMCP model in (\ref{eq:smcp-start})-(\ref{eq:smcp-end}). 
\end{theorem}
Theorem \ref{theorem:smcp} establishes that the estimator $\hat{t}_{\text{MAP}}$ constructed based on model (\ref{eq:smcp-start})-(\ref{eq:smcp-end}) achieves a localization rate of order $\log^{1+\varepsilon} T$. In the proof of Theorem \ref{theorem:smcp} (see Appendix \ref{app:localization-smcp}), we are able to show that if $|b_0|$ achieves its lower bound in Assumption \ref{assumption:mean}, i.e. $b_0^2 = \underline{b }^2\log^{-\varepsilon}T$, then $\kappa = Cb_0^{-2}$ for some constant $C > 0$, i.e. the exact localization rate is $Cb_0^{-2} \log T$, which \cite{Wang2020_localization} previously demonstrated is minimax optimal aside from a $\log T$ factor. 

The result in Theorem \ref{theorem:smcp} is unchanged if we relax the normality assumption to each $y_t$ is sub-Gaussian, but we cannot remove the independence assumption without sacrificing the $\log T$ localization rate. In Theorems \ref{theorem:sscp} and \ref{theorem:smscp}, we turn our attention to characterizing the localization rate when there is a change in $\pmb{\lambda}$ and examine the behavior of $\hat{t}_{\text{MAP}}$ under a more general dependence setting. In particular, when $\mathbf{y}$ is an $\alpha$-mixing process (see Definition \ref{def:alpha-mixing} of Appendix \ref{app:notation}), then under mild regularity conditions both the SSCP and SMCP models can identify a change in the variance (or the mean in the case of SMSCP) with a localization rate of order $\sqrt{T}\log^{1+\varepsilon} T$ for some $\varepsilon >0 $. This rate can be reduced to $\sqrt{T\log^{1+\varepsilon} T}$ if we are willing to assume that each $y_t$ is an independent draw from a Gaussian distribution as in (\ref{eq:dgp}). Assumption \ref{assumption:scale} formalizes the conditions under which our models can detect a change in the scale of $\mathbf{y}$ and achieve these localization rates:

\begin{assumption}[Detectable Scale]\label{assumption:scale}
   Let $\mathbf{y}$ be generated as in Assumption \ref{assumption:1} so that $s_0^2$ is the true change in the variance of $\mathbf{y}$ at time $t_0$. Assume that there exist some compact intervals $I_1 \subset (0, 1)$ and $I_2 \subset (1, \infty)$ such that $s_0^2 \in I_1 \cup I_2$. 
\end{assumption}

\begin{theorem}[SSCP Localization Rate]\label{theorem:sscp}
    Suppose that Assumption \ref{assumption:1} (i) and (ii) hold with $b_0=0$. Define $z_t$ as in (\ref{eq:normalized}) and assume that:
    \vspace{-10pt}
    \begin{enumerate}[label=(\roman*)]
        \item $\{y_t\}_{t=1}^T$ is an $\alpha$-mixing process with coefficients that satisfy $\alpha_k \leq e^{-Ck}$ for some $C > 0$.
        \item There exists constants $\delta, \; D > 0$ such that $\max_{1\leq t \leq T} \E\left[|z_t^2 - 1|^{2+\delta}\right]\leq D.$ 
    \end{enumerate}
    \vspace{-5pt}
    For any $\varepsilon > 0$ if Assumption $\ref{assumption:scale}$ holds along with Assumption $\ref{assumption:1}$ (iii) (c), then there exists some $\kappa > 0$ so that: 
    \vspace{-5pt}
    \begin{align*}
        \lim_{T\to\infty}\Pr\left(|t_0 - \hat{t}_{\normalfont \text{MAP}}| \leq \kappa \sqrt{T}\log^{1+\varepsilon} T\right) = 1  
    \end{align*}
    where $\hat{t}_{\normalfont \text{MAP}}$ is constructed as in (\ref{eq:map}) by fitting the SSCP model in (\ref{eq:sscp-start})-(\ref{eq:sscp-end}). Furthermore, if $\mathbf{y}$ is generated according to (\ref{eq:dgp}) and Assumption $\ref{assumption:1}$ (iii) (b) holds, then we have:
    \vspace{-5pt}
    \begin{align*}
        \lim_{T\to\infty}\Pr\left(|t_0 - \hat{t}_{\normalfont \text{MAP}}| \leq \kappa \sqrt{T\log T}\right) = 1.  
    \end{align*}
\end{theorem}

\begin{theorem}[SMSCP Localization Rate]\label{theorem:smscp}
    Suppose that Assumption \ref{assumption:1} (i) and (ii) hold. Define $z_t$ as in (\ref{eq:normalized}) and assume that:
    \vspace{-10pt}
    \begin{enumerate}[label=(\roman*)]
        \item $\{y_t\}_{t=1}^T$ is an $\alpha$-mixing process with coefficients that satisfy $\alpha_k \leq e^{-Ck}$ for some $C > 0$.
        \item There exists constants $\delta_1, \; D_1 > 0$ such that $\max_{1\leq t \leq T} \E\left[|z_t|^{2+\delta_1}\right]\leq D_1.$ 
        \item There exists constants $\delta_2, \; D_2 > 0$ such that $\max_{1\leq t \leq T} \E\left[|z_t^2 - 1|^{2+\delta_2}\right]\leq D_2.$ 
    \end{enumerate}
    \vspace{-5pt}
    For any $\varepsilon > 0$ if either Assumption $\ref{assumption:mean}$ holds for the constant $\xi / 2$ where $\xi \in [0,\varepsilon)$ or Assumption $\ref{assumption:scale}$ holds along with Assumption $\ref{assumption:1}$ (iii) (c), then there exists some $\kappa > 0$ so that: 
    \vspace{-5pt}
    \begin{align*}
        \lim_{T\to\infty}\Pr\left(|t_0 - \hat{t}_{\normalfont \text{MAP}}| \leq \kappa \sqrt{T}\log^{1+\varepsilon}\right) = 1  
    \end{align*}
    where $\hat{t}_{\normalfont \text{MAP}}$ is constructed as in (\ref{eq:map}) by fitting the SMSCP model in (\ref{eq:smscp-start})-(\ref{eq:smscp-end}). Furthermore, if $\mathbf{y}$ is generated according to (\ref{eq:dgp}) and Assumption $\ref{assumption:1}$ (iii) (b) holds, then we have:
        \vspace{-5pt}
    \begin{align*}
        \lim_{T\to\infty}\Pr\left(|t_0 - \hat{t}_{\normalfont \text{MAP}}| \leq \kappa \sqrt{T\log^{1+\xi} T}\right) = 1 .
    \end{align*}
\end{theorem}
Theorem \ref{theorem:sscp} is borrowed directly from \cite{Cappello22}, while Theorem \ref{theorem:smscp} is a new result whose proof is given in Appendix \ref{app:localization-smscp}. Note that in the statement of Theorem \ref{theorem:smscp}, it is possible that $\xi =0$ so that if we have independent Gaussian data and either $|b_0|$ or $|s^2_0-1|$ does not vanish as $T$ gets large, then the localization rate in Theorem \ref{theorem:smscp} reduces to $\kappa \sqrt{T\log T}$, which is precisely the rate achieved by the BSOP method of \cite{Wang21}. Theorem \ref{theorem:smscp} also shows that when we have independent Gaussian data and Assumption \ref{assumption:mean} holds with $s_0^2 = 1$, then the localization rate for $\hat{t}_{\text{MAP}}$ based on the SMSCP model will be a factor $\sqrt{T/\log T}$ larger than the localization rate based on the SMCP model. Therefore, even though the SMSCP model can identify a change in the mean of $\mathbf{y}$ even when the variance is constant, using SMCP model in this case will result in more rapid localization and tighter credible intervals around $t_0$. 

\subsection{Credible Sets}
\label{sec:cred-sets}

In addition to generating the point estimate $\hat{t}_\text{MAP}$, we can use the posterior probabilities $\overline{\pmb{\pi}}$ returned by each of the SCP models to construct $\alpha$-level credible sets around $t_0$ by solving:\footnote{Note that (\ref{eq:cs}) is just an integer program where we solve $\mathbf{z} = \argmin{\mathbf{x}\in\{0,1\}^T} \; \mathbf{1}'\mathbf{x} \text{ s.t. } \overline{\pmb{\pi}}'\mathbf{x} \geq \alpha$, then set $\mathcal{CS}(\alpha, \overline{\pmb{\pi}})$ equal to the elements of $\{1\ldots,T\}$ that correspond to the elements of $\mathbf{z}$ that equal one. This is an example of a knapsack problem.}
\begin{align}\label{eq:cs}
    \mathcal{CS}(\alpha, \overline{\pmb{\pi}}) := \argmin{S \subset [T]} |S| \quad\text{ s.t. } \sum_{t \in S} \overline{\pi}_t \geq \alpha.
\end{align}
In light of the theoretical results in the previous section, a credible set constructed as per (\ref{eq:cs}) will contain a point that is within the localization rate of the true change-point with high probability:
\begin{corollary}
Let $\epsilon_T$ be the localization rate corresponding to one of Theorems \ref{theorem:smcp}, \ref{theorem:sscp}, or \ref{theorem:smscp}, then under the conditions of the these theorems:\vspace{-10pt}
\begin{align*}
    \lim_{T \to \infty} \Pr\left(\min_{t\in \mathcal{CS}(\alpha, \overline{\pmb{\pi}})} \; |t - t_0| \leq \epsilon_T \right) = 1.
\end{align*}
\end{corollary}

\cite{Cappello22} showed that in models where there is no actual change-point, credible sets constructed according to (\ref{eq:cs}) tend to encompass large portions of the series since the elements $\{\overline{\pi}_t\}_{t=1}^T$ tend to be quite diffuse. This observation motivated \cite{Cappello22} to adopt the heuristic of classifying a change-point as detected only if $|\mathcal{CS}(\alpha, \{\overline{\pi}_t\}_{t=1}^T)| \leq T/2$, a convention which we also adopt.