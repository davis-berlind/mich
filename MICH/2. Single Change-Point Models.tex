\section{Single Change-Point (SCP) Models}
\label{sec:scp}

Suppose that we have $T+B_l+B_r$ observations of a univariate time-series $\mathbf{y} = \{y_t\}_{t=1-B_l}^{T+B_r}$ that is generated by the following process,
\begin{align}\label{eq:dgp}
    y_t \:|\: \mu_t, \lambda_t \overset{\text{ind.}}{\sim} \mathcal{N}\left(\mu_t, \lambda^{-1}_t\right), \quad t \in \{1-B_l,\ldots,T+B_r\}.
\end{align}
We assume that either $\pmb{\mu} = \{\mu_t\}_{t=1-B_l}^{T+B_r}$, $\pmb{\lambda} = \{\lambda_t\}^{T+B_r}_{t=1-B_l}$, or both exhibit piece-wise constant structures with a single change-point occurring at some time $t_0 \in \{1,\ldots,T\}$. Of course, the whole point of CPD is that we do not know the location of $t_0$. Instead, we can model the location of the change-point as a scalar categorical random variable $\gamma$ that has support on the set $\{1,\ldots,T\}$. At the same time we can draw the shift in $\pmb{\mu}$ or $\pmb{\lambda}$ from some other distribution independent of $\gamma$. We will see that by choosing conditionally conjugate prior distributions for these shifts, we arrive at a simple closed form posterior distribution for the location of $\gamma$. 

Another common assumption in the CPD literature is that there is a buffer at the start and end of $\mathbf{y}$ within which the signal is constant. We use the parameters $B_l,B_r \geq 0$ to introduce this buffer by assuming that $\mu_t$ and $\lambda_t$ are constant for all $t\in\{1-B_l,\ldots,0,T+1, \ldots, T+B_r\}$. In the remainder of this section we introduce three single change-point (SCP) models for the random processes that generates the changes in $\pmb{\mu}$ and $\pmb{\lambda}$. In Section \ref{sec:smcp} and Section \ref{sec:sscp} we introduce models that handle the respective cases of either a single change-point in $\pmb{\mu}$ or a single change-point in $\pmb{\lambda}$, while the model in Section \ref{sec:smscp} addresses the setting of a simultaneous change in both $\pmb{\mu}$ and $\pmb{\lambda}$. In each case, our choice of conjugate priors leads to simple closed form solutions for the joint posterior distributions over the model parameters.

\subsection{Single Mean Change-Point (SMCP)}
\label{sec:smcp}

Suppose that we have $T+B_l+B_r$ observations from (\ref{eq:dgp}) and that $\pmb{\mu}$ is a random sequence with an underlying piece-wise constant structure generated by the following Single Mean Change-Point (SMCP) model:
\begin{align} \label{eq:smcp-start}
    \mu_t &= b\mathbbm{1}{\left\{t\geq \gamma \right\}} \\
    b &\sim \mathcal{N}(0,\tau_0^{-1}) \\
    \gamma &\sim \text{Categorical}(\pmb{\pi}),\; \pmb{\pi} \in \mathcal{S}^T \\
    b &\indep \gamma
    \label{eq:smcp-end}
\end{align}
where $\mathcal{S}^T$ is the $T$-dimensional probability simplex\footnote{I.e. $\pmb{\pi} \in \mathcal{S}^T$ implies $\pi_t > 0$ for each $t$ and $\sum_{t=1}^T \pi_t = 1$.} and $\pmb{\lambda}$, $\pmb{\pi}$, and $\tau_0$, are known constants. In words, $\mathbf{y}$ starts centered at zero and jumps by $b\in\mathbb{R}$ at some time $\gamma \in \{1,\ldots,T\}$. For each $t \in\{1, \ldots, T\}$, the posterior distribution $p(b, \gamma \:|\: \mathbf{y})$ for the SMCP model is given by:
\begin{align}
    b \:|\: \gamma = t, \: \mathbf{y} &\sim \mathcal{N}\left(\overline{b}_{t}, \overline{\tau}_{t}^{-1}\right) \label{eq:b-smcp} \\
    \gamma \:|\: \mathbf{y} &\sim \text{Categorical}(\overline{\pmb{\pi}}) \label{eq:gamma-smcp} \\
    \overline{\tau}_t &= \tau_0 + \sum_{t'=t}^{T+B_r} \lambda_{t'} \\
    \overline{b}_t &= \overline{\tau}^{-1}_t\sum_{t'=t}^{T+B_r} \lambda_{t'} y_{t'} \\
    \overline{\pi}_t &= \frac{\pi_t\overline{\tau}^{-\frac{1}{2}}_t\exp\left(\overline{\tau}_t\overline{b}^2_t / 2\right)}{\sum_{t'=1}^T \pi_{t'}\overline{\tau}^{-\frac{1}{2}}_{t'}\exp\left(\overline{\tau}_{t'}\overline{b}^2_{t'} / 2\right)}.
\end{align}
Even though $b$ and $\gamma$ are generated independently, we see above that these parameters can exhibit arbitrary levels of correlation in the posterior distribution. We also note that the model described in (\ref{eq:smcp-start})-(\ref{eq:smcp-end}) is identical to the SER model introduced in \cite{Wang20} when the entries in the first $B_r$ rows of the covariate matrix $\mathbf{X}$ are equal to zero, the entries in the last $B_r$ rows are equal to one, and the middle $T$ rows are triangular with all of the lower-triangular entries set equal to one. Motivated by this connection, we define a function \texttt{SMCP} that is analogous to the \texttt{SER} function in \cite{Wang20}. \texttt{SMCP} takes $\mathbf{y}$ as an input and returns the posterior parameters of $p(b, \gamma\:|\:\mathbf{y})$ as its output: 
\begin{align}\label{eq:smcp}
    \texttt{SMCP}\left(\mathbf{y} \:;\: \pmb{\lambda}, \tau_0, \pmb{\pi}, B_l,B_r\right) := \{\overline{b}_t, \overline{\tau}_t, \overline{\pi}_t\}_{t=1}^T.
\end{align}
When appropriate, we will also use the notation:
\begin{align}
    \{b,\gamma\} \sim \text{SMCP}(\{\overline{b}_t, \overline{\tau}_t, \overline{\pi}_t\}_{t=1}^T)
\end{align}
to mean that $b$ and $\gamma$ follow the distribution specified in (\ref{eq:b-smcp})-(\ref{eq:gamma-smcp}).

\subsection{Single Scale Change-Point (SSCP)}
\label{sec:sscp}

The scale change-point model in this section is precisely the same model introduced by \cite{Cappello22} with the addition of buffer parameters $B_l$ and $B_r$, and a known sequence of precision parameters $\pmb{\tau}=\{\tau_t\}_{t={1-B_l}}^{T+B_r}$. Again, we have $T+B_l+B_r$ observations from (\ref{eq:dgp}), but now we assume that $\pmb{\mu} \equiv \mathbf{0}$ and that the sequence of precision parameters $\pmb{\lambda}$ is random with an underlying piece-wise constant structure generated by the following Single Scale Change-Point (SSCP) model:
\begin{align}\label{eq:sscp-start}
    \lambda_t &= s^{\mathbbm{1}\{t \geq \gamma\}}\tau_t  \\
    s &\sim \text{Gamma}(u_0,v_0) \\
    \gamma &\sim \text{Categorical}(\pmb{\pi}),\; \pmb{\pi} \in \mathcal{S}^T \\
    s &\indep \gamma.
    \label{eq:sscp-end}
\end{align}
In words, the model in (\ref{eq:sscp-start})-(\ref{eq:sscp-end}) independently draws the location of the change-point $\gamma$ and the shift in the precision $s$, then scales each precision parameter $\tau_t$ by $s$ if $t \geq \gamma$. To see this, suppose that $\gamma > t$, i.e. $t$ is a time before the change-point occurs, then we will have $\mathbbm{1}\{t \geq \gamma\} = 0$ and $s^{\mathbbm{1}\{t \geq \gamma\}} = 1$. On the other hand, if $\gamma \leq t$, then we have $s^{\mathbbm{1}\{t \geq \gamma\}} = s$, so multiplying $\tau_t$ by $s^{\mathbbm{1}\{t \geq \gamma\}}$ gives us the desired scaling effect. \cite{Cappello22} show that posterior distribution $p(s, \gamma\:|\:\mathbf{y})$ for the SSCP model is given by:
\begin{align}
    s \:|\: \gamma = t, \: \mathbf{y} &\sim \text{Gamma}\left(\overline{u}_{t}, \overline{v}_{t}\right) \label{eq:s-sscp} \\
    \gamma \:|\: \mathbf{y}&\sim \text{Categorical}(\overline{\pmb{\pi}}) \label{eq:gamma-sscp}\\
    \overline{u}_{t} &= u_0 + \frac{T +B_r - t + 1}{2} \\
    \overline{v}_{t} &= v_0 + \frac{1}{2} \sum_{t'=t}^{T+B_r} \tau_{t'}y_{t'}^2 \\
    \overline{\pi}_t &= \frac{\pi_t \Gamma(\overline{u}_{t}) \overline{v}_{t}^{-\overline{u}_{t}}\exp\left(- \frac{1}{2}\sum_{t'=1}^{t-1} \tau_{t'}y_{t'}^2\right)}{\sum_{t'=1}^T \pi_{t'} \Gamma(\overline{u}_{t'}) \overline{v}_{t'}^{-\overline{u}_{t'}}\exp\left(- \frac{1}{2}\sum_{t''=1}^{t'-1} \tau_{t''}y_{t''}^2\right)}. \label{eq:pi-sscp}
\end{align}
Note that in (\ref{eq:pi-sscp}) the summation term $\sum_{t'=1}^{t-1} \tau_{t'}y_{t'}^2$ may be ill-defined if $t = 1$. Here and throughout the rest of this paper we use the convention that a sum is just equal to zero if the indexing set is empty. We can also define a function \texttt{SSCP} that takes the response $\mathbf{y}$ as an input and returns the posterior parameters of $p(s, \gamma\:|\:\mathbf{y})$ as its output: 
\begin{align}
    \texttt{SSCP}\left(\mathbf{y} \:;\: \pmb{\tau}, u_0, v_0, \pmb{\pi}, B_l, B_r\right) := \{\overline{u}_t, \overline{v}_t, \overline{\pi}_t\}_{t=1}^T.
\end{align}
As with the SMCP model, we introduce a shorthand for the distribution (\ref{eq:s-sscp})-(\ref{eq:gamma-sscp}): 
\begin{align}
    \{s,\gamma\} \sim \text{SSCP}(\{\overline{u}_t, \overline{v}_t, \overline{\pi}_t\}_{t=1}^T).
\end{align}

\subsection{Single Mean-Scale Change-Point (SMSCP)}
\label{sec:smscp}

We now merge the SMCP and SSCP settings by allowing $\mathbf{y}$ to have a single change-point where both the mean and scale shift simultaneously. We again assume that we have $T+B_l+B_r$ observations from (\ref{eq:dgp}), only now the concurrent change to the piece-wise constant structure of $\pmb{\mu}$ and $\pmb{\lambda}$ is generated by the following Single Mean-Scale Change-Point (SMSCP) model:
\begin{align}
    \label{eq:smscp-start}
    \mu_{t} &= b \mathbbm{1}{\{t \geq \gamma\}} \\
    \lambda_t &= \tau_t s^{\mathbbm{1}\{t\geq \gamma\}} \\
    b \:|\: s &\sim \mathcal{N}(0,(s \tau_0)^{-1}) \\
    s &\sim \text{Gamma}(u_0, v_0) \\
    \gamma &\sim \text{Categorical}(\pmb{\pi}),\;\pmb{\pi}\in \mathcal{S}^T \\
    \gamma &\indep \{b,s\} 
    \label{eq:smscp-end}
\end{align}
Here again the precision parameters $\pmb{\tau}$ are known constants. The posterior distribution for $p(b, s, \gamma \:|\: \mathbf{y})$ under the SMSCP setting is given by:
\begin{align}
    b \:|\: s, \gamma = t, \mathbf{y} &\sim \mathcal{N}(\overline{b}_t, (\overline{\tau}_t s)^{-1}) \label{eq:b-smscp} \\
    s \:|\: \gamma = t, \mathbf{y} &\sim \text{Gammma}(\overline{u}_t, \overline{v}_t) \\
    \gamma \:|\: \mathbf{y} &\sim \text{Categorical}(\overline{\pmb{\pi}}) \\
    \overline{\tau}_t &= \tau_0 + \sum_{t' = t}^{T+B_r} \tau_{t'} \label{eq:gamma-smscp} \\
    \overline{b}_t &= \overline{\tau}^{-1}_t \sum_{t'=t}^{T+B_r} \tau_{t'} y_{t'} \\
    \overline{u}_t &= u_0 + \frac{T +B_r- t + 1}{2} \\
    \overline{v}_t &= v_0 - \frac{\overline{\tau}_t\overline{b}^2_t}{2} + \frac{1}{2} \sum_{t'=t}^{T+B_r} \tau_{t'}y_{t'}^2 \\
    \overline{\pi}_t &= \frac{\pi_t\overline{\tau}_t^{-\frac{1}{2}} \Gamma(\overline{u}_t) \overline{v}_t^{-\overline{u}_t}\exp\left(- \frac{1}{2}\sum_{t'=1}^{t-1} \tau_{t'}y^2_{t'}\right)}{\sum_{t'=1}^T\pi_{t'}\overline{\tau}_{t'}^{-\frac{1}{2}} \Gamma(\overline{u}_{t'}) \overline{v}_{t'}^{-\overline{u}_{t'}}\exp\left(- \frac{1}{2}\sum_{t''=1}^{t'-1} \tau_{t''}y^2_{t''}\right)}.
\end{align}
As before, we define a function \texttt{SMSCP} that takes $\mathbf{y}$ as an input and returns the posterior parameters of $p(b, s, \gamma\:|\:\mathbf{y})$ as its output:
\begin{align}
    \texttt{SMSCP}\left(\mathbf{y} \:;\: \pmb{\tau}, \tau_0, u_0, v_0, \pmb{\pi}, B_l, B_r\right) := \{\overline{b}_t, \overline{\tau}_t, \overline{u}_t, \overline{v}_t, \overline{\pi}_t\}_{t=1}^T
\end{align}
as well as a shorthand for the distribution (\ref{eq:b-smscp})-(\ref{eq:gamma-smscp})
\begin{align}
    \{b,s,\gamma\}\sim\text{SMSCP}(\{\overline{b}_t, \overline{\tau}_t, \overline{u}_t, \overline{v}_t, \overline{\pi}_t\}_{t=1}^T).
\end{align}

\subsection{Credible Sets and Localization Rate}
\label{sec:localization}

Suppose that $t_0$ is the location of the true change-point for one of the SCP models. Then given $\{\overline{\pi}_t\}_{t=1}^T$ from any of the models, a natural estimator for $t_0$ is the posterior most probable location of $\gamma$, i.e. the maximum \textit{a posteriori} (MAP) estimator:
\begin{align}\label{eq:map}
    \hat{t}_{\text{MAP}} := \argmax{1 \leq t \leq T} \; \overline{\pi}_t.
\end{align}
With $\overline{\pmb{\pi}} = \{\overline{\pi}_t\}_{t=1}^T$, we can also intuitively construct an $\alpha$-level credible set for $t_0$ by solving:\footnote{Note that (\ref{eq:cs}) is just an integer program where we solve $\mathbf{z} = \argmin{\mathbf{x}\in\{0,1\}^T} \; \mathbf{1}'\mathbf{x} \text{ s.t. } \overline{\pmb{\pi}}'\mathbf{x} \geq \alpha$, then set $\mathcal{CS}(\alpha, \overline{\pmb{\pi}})$ equal to the elements of $\{1\ldots,T\}$ that correspond to the elements of $\mathbf{z}$ that equal one. This is an example of a knapsack problem.}
\begin{align}\label{eq:cs}
    \mathcal{CS}(\alpha, \overline{\pmb{\pi}}) := \argmin{S \subset \{1,\ldots,T\}} |S| \quad\text{ s.t. } \sum_{t \in S} \overline{\pi}_t \geq \alpha.
\end{align}
\cite{Cappello22} showed that in models where there is no actual change-point, credible sets constructed according to (\ref{eq:cs}) tend to encompass large portions of the series since the elements $\{\overline{\pi}_t\}_{t=1}^T$ tend to be quite diffuse. This observation motivated \cite{Cappello22} to adopt the heuristic of classifying a change-point as detected only if $|\mathcal{CS}(\alpha, \{\overline{\pi}_t\}_{t=1}^T)| \leq T/2$, a convention which we also adopt.

We now show that under mild regularity conditions, the MAP estimator (\ref{eq:map}) produced by each of the SCP models achieves a localization rate that is minimax optimal up to a $\log(T)$ factor. Formally, we make the following assumption:
\begin{assumption}\label{assumption:1}
    Let $t_0$ be the location of the change-point in either the piece-wise constant mean sequence $\{\mu_t\}_{t=1}^T$, the piece-wise constant scale sequence $\{\lambda_t\}_{t=1}^T$, or both, and let $\{y_t\}_{t=1}^T$ be generated according to (\ref{eq:dgp}). Assume that:
    \begin{enumerate}[label=(\roman*)]
        \item (Minimum Spacing Criterion) There exists a constant $c \in (0,1/2)$ such that $\min\{t_0,T-t_0\} > cT$. 
        \item (Bounded and Non-Negligible Mean Signal) Let $b_0$ be the true change in the mean of $\mathbf{y}$ at $t_0$. Assume that $|b_0|$ is bounded, so that there exists some finite $\overline{b}$ such that $|b_0| < \overline{b}$. Furthermore, assume that for any $\varepsilon > 0$, there is a $\varphi \in (0,\varepsilon/2)$ and a $T^* > 0$ such that for all  $T > T^*$: 
        \begin{align*}
            \log^{-\varphi/2}T \leq |b_0|.
        \end{align*}
        \item (Bounded and Non-Negligible Scale Signal) Let $s^2_0$ be the true change in the variance of $\mathbf{y}$ at $t_0$. Assume that there exists some fixed intervals $I_1 \subset (1, \infty)$ and $I_2 \subset (0, 1)$ such that $s_0^2 \in I_1 \cup I_2$. 
        \item (Proper Prior) The hyper-parameters are chosen so that $\tau_0, u_0, v_0 > 0$.
        \item (Bounded Prior) The prior distribution for the change-point location, $\pi_t := \mathbb{P}(\gamma = t \:; \pmb{\pi})$, is chosen so that either:
        \begin{enumerate}
            \item $$\min_{cT < t < (1-c) T} \log \frac{\pi_{t_0}}{\pi_{t}} = \mathcal{O}(\log T).$$
            \item $$\min_{cT < t < (1-c) T} \log \frac{\pi_{t_0}}{\pi_{t}} = \mathcal{O}(\sqrt{T\log T}).$$
        \end{enumerate}
    \end{enumerate}
\end{assumption}
\begin{remark} \label{rmk:1}
    The rate at which $b_0$ can vanish in Assumption \ref{assumption:1} (ii) was chosen purely for technical reasons. In practice, if $\varphi$ is small, then $\log^{-\varphi}T$ may be approximately equal to one, even for very large $T$. In this case, it may be more reasonable to assume that there is a (small) constant $\underline{b} > 0$ such that $|b_0| > \underline{b}$. The proof of Theorem \ref{theorem:smcp} is exactly the same in this case and depending on the choice of $\underline{b}$, this assumption could return tighter lower-bounds for smaller $T$ on the event described in Theorem \ref{theorem:smcp}.
\end{remark}
Assumption \ref{assumption:1} (i) previously appeared in \cite{Cappello21} and \cite{Cappello22} and intuitively requires that as $T \to \infty$, an infinite number of observations accumulate between $t_0$ and each of the end-points of the sequence. Assumption \ref{assumption:1} (ii) allows $b_0$ to vanish as $T$ gets large, as long as it does not happen at a rate faster than $\log^\varepsilon T$. Similarly, Assumption \ref{assumption:1} (iii) is precisely Assumption 1 (b) from \cite{Cappello22}. Assumption \ref{assumption:1} (iv) simply requires that the priors in the SCP models be proper. We now can state the main localization result for the SMCP model:
\begin{theorem}[SMCP Localization Rate]\label{theorem:smcp}
    Suppose that Assumption \ref{assumption:1} (i), (ii), (iv), and (v) (a) hold. Then, for any $\varepsilon > 0$, there exists a constant $\kappa > 0$ such that, with probability approaching one, we have:
    \begin{align*}
        \max_{t \;:\; |t - t_0| > \kappa \log^{1+\varepsilon} T} \; \mathbb{P}(\gamma = t  \;|\; \mathbf{y} ; \tau_0,\pmb{\pi}) < \mathbb{P}(\gamma = t_0  \;|\; \mathbf{y} ; \tau_0,\pmb{\pi}).
    \end{align*}
\end{theorem}
From Theorem \ref{theorem:smcp}, we see that for $\varepsilon >0$, the estimator $\hat{t}_{\text{MAP}}$ constructed based on model (\ref{eq:smcp-start})-(\ref{eq:smcp-end}) achieves a localization rate of order $\log^{1+\varepsilon}T$.\footnote{Here and throughout this article we use the convention $\log^{1+\varepsilon}T = (\log T)^{1+\varepsilon}$.} \cite{Wang2020_localization} previously demonstrated that this rate is minimax optimal up to a $\log(T)$ term for localizing a change in the mean. In words, for large $T$ we know that $\hat{t}_{\text{MAP}}$ lives in a size $\kappa \log^{1+\varepsilon} T$ neighborhood of $t_0$ with high probability, so if we normalize the index $t$ so that $t\in\{1/T, \ldots, 1\}$, then this neighborhood is converging to the true change-point as $T \to \infty$. The proof of Theorem \ref{theorem:smcp} is given in Appendix \ref{app:localization-smcp}.  

Next, we restate a result from \cite{Cappello22}, which gives the localization rate for the SSCP model: 
\begin{theorem}[SSCP Localization Rate]\label{theorem:sscp}
    Suppose that Assumption \ref{assumption:1} (i), (iii), (iv), and (v) (b) hold. Then, for any $\varepsilon > 0$, there exists a constant $\kappa > 0$ such that, with probability approaching one, we have:
    \begin{align*}
        \max_{t \;:\; |t - t_0| > \kappa \sqrt{T\log^{1+\varepsilon} T},\;\min\{t,T-t\} > cT} \; P(\gamma = t  \;|\; \mathbf{y} ; u_0,v_0,\pmb{\pi}) < P(\gamma = t_0  \;|\; \mathbf{y} ; u_0,v_0,\pmb{\pi}).
    \end{align*}
\end{theorem}
As argued by \cite{Cappello22}, Theorem \ref{theorem:sscp} shows that the estimator $\hat{t}_{\text{MAP}}$ constructed based on model (\ref{eq:sscp-start})-(\ref{eq:sscp-end}) attains a localization rate of order $\sqrt{T\log^{1+\varepsilon} T}$, which \cite{Wang21} previously demonstrated is minimax optimal up to a logarithm factor for localizing a change in the variance.

Lastly, Theorem \ref{theorem:smscp} establishes the localization rate when there is either a change in the mean or a change in the variance in the SMSCP model:
\begin{theorem}[SMSCP Localization Rate]\label{theorem:smscp}
Suppose that Assumption \ref{assumption:1} (i), (iv), and (v) (b) hold. Additionally if either Assumption \ref{assumption:1} (ii), (iii), or both hold, then for any $\varepsilon > 0$, there exists a constant $\kappa > 0$ such that, with probability approaching one, we have:
    \begin{align*}
        \max_{t \;:\; |t - t_0| > \kappa \sqrt{T\log^{1+\varepsilon} T},\;\min\{t,T-t\} > cT} \; P(\gamma = t  \;|\; \mathbf{y} ; \tau_0,u_0,v_0,\pmb{\pi}) < P(\gamma = t_0  \;|\; \mathbf{y} ;\tau_0, u_0,v_0,\pmb{\pi}).
    \end{align*}
\end{theorem}
Theorem \ref{theorem:smscp} shows that the estimator $\hat{t}_{\text{MAP}}$ constructed based on model (\ref{eq:smscp-start})-(\ref{eq:smscp-end}) attains a localization rate of order $\sqrt{T\log^{1+\varepsilon} T}$. As argued above, this rate will be minimax optimal up to a logarithm factor when there truly is a change in the variance (i.e. when Assumption \ref{assumption:1} (iii) holds). However, if there is no change in the variance and just a change in the mean of $\mathbf{y}$ (i.e. only Assumption \ref{assumption:1} (ii) holds), then the localization rate for the SMSCP model will fall short of the optimal $\log^{1+\varepsilon} T$ rate that is attained by the SMCP model, as per Theorem \ref{theorem:smcp}.