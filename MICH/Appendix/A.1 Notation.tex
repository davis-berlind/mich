\subsection{Notation}
\label{app:notation}

\subsubsection{Asymptotic Analysis} 

In the following proofs, for some functions $f$ and $g$ we use the notation $f(T) = \mathcal{O}(g(T))$ to mean that there exists some constant $M > 0$ and some $T^* > 0$ so that for all $T > T^*$ we have: $$|f(T)| \leq M g(T).$$ Similarly, we use $g(T) \gtrsim f(T)$ to indicate that there exists $M > 0$ so that for any $T$ we have $f(T) \leq M g(T).$ We also use $f(T) = o(g(T))$ to mean:$$\lim_{T\to\infty} \frac{f(T)}{g(T)} = 0$$ and $f(T) \sim g(T)$ to mean: $$\lim_{T\to\infty} \frac{f(T)}{g(T)} = 1.$$ 

\subsubsection{Random Variables} 

For a generic real-valued random variable $X$, we use $p(x)$ to denote the density of $X$ when it exists. If $\sigma(X)$ is the $\sigma$-algebra generated by $X$, then we use $\Pr(\cdot)$ to indicate the measure induced by $X$ on the measurable space $(\mathbb{R}),\sigma(X)$.

\subsubsection{Sub-Gaussian Distributions}

We write $X\in\mathcal{SG}(\sigma)$ to denote that the random variable $X$ has a sub-Gaussian distribution with parameter $\sigma$, i.e. for all $\lambda \in \mathbb{R}$: 
\begin{align} 
    \E[\exp(\lambda X)] \leq \exp\left[\frac{\lambda^2\sigma^2}{2}\right]. \label{def:sub-gaussian}
\end{align}
For $t, \lambda > 0$, we have:
\begin{align}
     \Pr(|X| > t) &\leq \Pr(X > t) + \Pr(-X > t) \tag{union bound} \\
     &= \Pr(e^{\lambda X} > e^{\lambda t}) + \Pr(e^{-\lambda X} > e^{\lambda t}) \notag \\
     &\leq \E[\exp(\lambda X - \lambda t)] + \E[\exp(-\lambda X - \lambda t)] \tag{Markov inequality} \\
     &\leq 2\exp\left(\frac{\lambda^2\sigma^2}{2} - \lambda t\right). \tag{by (\ref{def:sub-gaussian})} 
\end{align}
The last bound is minimized be setting $\lambda = \frac{t}{\sigma^2}$, which yields the Chernoff bound:
\begin{align} \label{eq:chernoff}
    \Pr(|X| > t) \leq 2\exp\left[-\frac{t^2}{2\sigma^2}\right]
\end{align}

\subsubsection{Sub-Exponential Distributions}

We write $X\in\mathcal{SE}(\nu, \alpha)$ to denote that the random variable $X$ has a sub-Exponential distribution with parameters $\nu$ and $\alpha$, i.e. for all $|\lambda| \leq \frac{1}{\alpha}$: 
\begin{align*}
    \E[\exp(\lambda X)] \leq \exp\left[\frac{\lambda^2\nu^2}{2}\right].
\end{align*}
If $X\in\mathcal{SE}(\nu, \alpha)$, then by Proposition 2.9 of \cite{Wainwright19}:
\begin{align}\label{eq:wainwright_prop_2.9}
    \Pr(|X - \E[x]| \geq t) \leq 
    \begin{cases}
        2 \exp\left[-\frac{t^2}{2\nu^2}\right], & \text{if } t \in \left(0, \frac{\nu^2}{\alpha}\right], \\
        2 \exp\left[-\frac{t}{2\alpha}\right], & \text{if } t \in \left(\frac{\nu^2}{\alpha}, \infty\right).
    \end{cases}
\end{align}
As per Example 2.11 of \cite{Wainwright19}, if $X \sim \chi^2_n$, then $X\in\mathcal{SE}(2\sqrt{n}, 4)$, and thus: 
\begin{align} \label{eq:chi2-ineq}
    \Pr\left(\frac{|X - n|}{n} \geq t\right) \leq 
    \begin{cases}
        2 \exp\left[-\frac{n t^2}{8}\right], & \text{if } t \in \left(0, 1\right], \\
        2 \exp\left[-\frac{n t}{8}\right], & \text{if } t \in \left(1, \infty\right).
    \end{cases}
\end{align}

\subsubsection{$\alpha$-mixing}

\begin{definition}\label{def:alpha-mixing}
Let $\{X_t\}_{t\in \mathbb{Z}}$ be a stochastic process on the probability space $(\Omega, \mathcal{F}, \Pr)$, then $\{X_t\}_{t\in \mathbb{Z}}$ is said to be $\alpha$-mixing if:
\begin{align*}
    \lim_{k\to\infty} \alpha_k(\{X_t\}_{t\in\mathbb{Z}}) = 0,
\end{align*}
where:
\begin{align*}
    \alpha_k(\{X_t\}_{t\in\mathbb{Z}}) := \sup_{t\in\mathbb{Z}} \; \alpha\left(\sigma(\{X_s\}_{s \leq t}), \; \sigma(\{X_s\}_{s \geq t + k})\right).
\end{align*}
Here $\sigma(Y)$ is the $\sigma$-algebra generated by $Y$ and the strong mixing, or $\alpha$-mixing, coefficient between two $\sigma$-algebras $\mathcal{A}, \mathcal{B} \subset \mathcal{F}$ is defined as:
\begin{align*}
    \alpha(\mathcal{A}, \mathcal{B}) := \sup_{A\in\mathcal{A}, B\in\mathcal{B}} |\Pr(A \cap B) - \Pr(A)\Pr(B)|.
\end{align*}
To simplify notation, we will often write $\alpha_k$ in place of $\alpha_k(\{X_t\}_{t\in\mathbb{Z}})$.
\end{definition}

\subsubsection{Event Probability Bounds}

% Lemma 1 %
\begin{lemma}[Lemma 3 in \citealp{Padilla23}]\label{lemma:Padilla23}
Let $\nu > 0$ be given. Suppose that $\{X_t\}_{t=1}^\infty$ is a stationary $\alpha$-mixing time-series with mixing coefficients $\{\alpha_k\}_{k=0}^K$. Suppose that $\E[X_t] = 0$ and that there exists constants $\delta, \Delta, D_1, D_2 > 0$ such that:
\begin{align*}
    \sup_{t \geq 1}\; \E\left[\left|X_t\right|^{2+\delta+\Delta}\right] \leq D_1 
\end{align*}
and:
\begin{align*}
    \sum_{k=0}^\infty (k+1)^{\frac{\delta}{2}} \alpha_k^{\frac{\Delta}{2+\delta+\Delta}} \leq D_1.
\end{align*}
Then there exists some constant $C > 0$ such that for any $a \in (0,1)$:
\begin{align*}
    \Pr \left(\left|\sum_{t'=1}^t X_{t'}\right| \leq a^{-1}C\sqrt{t}\left[\log (\nu t) + 1\right], \;\sforall t\geq \nu^{-1}\right) 
    \geq 1 - a^2.
\end{align*}
\end{lemma}

% Lemma 2 %
\begin{lemma}\label{lemma:2}
Suppose the conditions of Lemma \ref{lemma:Padilla23} hold for the process $\{X_t\}_{t \geq 1}$. Let $\varphi > 0$ be given, then for $T > 0$ and $c \in (0,1/2)$, there exists a constant $C_1 > 0$ so that for any $t_0$ that satisfies $\min\{t_0,T-t_0\} > cT$, we have:
\begin{align*}
    \Pr \left(\bigcup_{t=t_0 + 1}^{(1-c)T}\left\{ \left|\sum_{t'=t_0}^{t-1} X_{t'}\right| > C_1\sqrt{t-t_0}\log^{1+\varphi}T\right\}\right) 
    &\leq \frac{1}{\log^{2\varphi} T}
\end{align*}
and:
\begin{align*}
    \Pr \left(\bigcup_{t=cT}^{t_0-1}\left\{ \left|\sum_{t'=t}^{t_0-1} X_{t'}\right| > C_1\sqrt{t_0-t}\log^{1+\varphi}T\right\}\right) 
    &\leq \frac{1}{\log^{2\varphi} T}
\end{align*}
and:
\begin{align*}
    \Pr \left(\bigcup_{t=cT}^{(1-c)T}\left\{ \left|\sum_{t'=t}^{T} X_{t'}\right| > C_1\sqrt{T-t+1}\log^{1+\varphi}T\right\}\right) 
    &\leq \frac{1}{\log^{2\varphi} T}.
\end{align*}
\end{lemma}

\begin{proof}
By Lemma \ref{lemma:Padilla23}, there is some constant $C_1 > 0$ so that if we let $\nu = 1$ and $a = \log^{-\varphi} T$, then:
\begin{align*}
    \Pr \left(\bigcup_{t=1}^\infty\left\{\left|\sum_{t'=1}^t X_{t'}\right| > C_1\log^{\varphi}T\sqrt{t}\left[\log t + 1\right]\right\}\right) 
    \leq \frac{1}{\log^{2\varphi} T}.
\end{align*}
Note that since $\{X_t\}_{t=1}^\infty$ is a stationary process, we can shift the time indices of the sum by $t_0-1$ to get:
\scriptsize
\begin{align*}
    \frac{1}{\log^{2\varphi} T} &\geq \Pr \left(\bigcup_{t=1}^\infty\left\{\left|\sum_{t'=1}^t X_{t'}\right| > C_1\log^{\varphi}T\sqrt{t}\left[\log t + 1\right]\right\}\right) \\
    &= \Pr \left(\bigcup_{t=t_0+1}^\infty\left\{\left|\sum_{t'=t_0}^{t-1} X_{t'}\right| > C_1\log^{\varphi}T\sqrt{t-t_0}\left[\log (t-t_0) + 1\right]\right\}\right) \tag{stationarity} \\
    &\geq \Pr \left(\bigcup_{t=t_0+1}^{(1-c)T}\left\{\left|\sum_{t'=t_0}^{t-1} X_{t'}\right| > C_1\log^{\varphi}T\sqrt{t-t_0}\log (t-t_0)\right\}\right) \textcolor{red}{\text{we might need to change the constant } \,\,C_1 \,\,\text{to have the inequality here} } \\
    &\geq \Pr \left(\bigcup_{t=t_0+1}^{(1-c)T}\left\{\left|\sum_{t'=t_0}^{t-1} X_{t'}\right| > C_1\sqrt{t-t_0}\log^{1+\varphi}T\right\}\right). \tag{$T > t-t_0$}
\end{align*}
\normalsize
Similarly:
\scriptsize
\begin{align*}
     \frac{1}{\log^{2\varphi} T} &\geq \Pr \left(\bigcup_{t=1}^\infty\left\{\left|\sum_{t'=1}^t X_{t'}\right| > C_1\log^{\varphi}T\sqrt{t} \left[\log t + 1\right]\right\}\right) \\
     &\geq \Pr \left(\bigcup_{t=1}^{t_0-cT}\left\{\left|\sum_{t'=1}^t X_{t'}\right| > C_1\log^{\varphi}T\sqrt{t}\log t\right\}\right)  \textcolor{red}{\text{we might need to change the constant } \,\,C_1 \,\,\text{to have the inequality here} }\\
    &= \Pr \left(\bigcup_{t=1}^{t_0-cT}\left\{\left|\sum_{t'=1}^{t_0-cT + 1 - t} X_{t'}\right| > C_1\log^{\varphi}T\sqrt{t_0-cT + 1 - t}\log (t_0-cT + 1 - t)\right\}\right)  \\
    &= \Pr \left(\bigcup_{t=cT}^{t_0-1}\left\{\left|\sum_{t'=t}^{t_0-1} X_{t'}\right| > C_1\log^{\varphi}T\sqrt{t_0 - t}\log (t_0 - t)\right\}\right) \tag{stationarity} \\
    &\geq \Pr \left(\bigcup_{t=cT}^{t_0-1}\left\{\left|\sum_{t'=t}^{t_0-1} X_{t'}\right| > C_1\sqrt{t_0-t}\log^{1+\varphi} T\right\}\right). \tag{$T > t_0-t$}
\end{align*}
\normalsize
Lastly:
\scriptsize
\begin{align*}
     \frac{1}{\log^{2\varphi} T} &\geq \Pr \left(\bigcup_{t=1}^\infty\left\{\left|\sum_{t'=1}^t X_{t'}\right| > C_1\log^{\varphi}T\sqrt{t} \left[\log t + 1\right]\right\}\right) \\
     &\geq \Pr \left(\bigcup_{t=cT}^{(1-c)T}\left\{\left|\sum_{t'=1}^{t+1} X_{t'}\right| > C_1\log^{\varphi}T\sqrt{t+1}\log (t+1)\right\}\right) \\
    &= \Pr \left(\bigcup_{t=cT}^{(1-c)T}\left\{\left|\sum_{t'=1}^{T - t + 1} X_{t'}\right| > C_1\log^{\varphi}T\sqrt{T - t + 1}\log (T - t+1)\right\}\right)  \\
    &= \Pr \left(\bigcup_{t=cT}^{(1-c)T}\left\{\left|\sum_{t'=t}^{T} X_{t'}\right| > C_1\log^{\varphi}T\sqrt{T - t+1}\log (T - t+1)\right\}\right)  \tag{stationarity} \\
    &\geq \Pr \left(\bigcup_{t=cT}^{(1-c)T}\left\{\left|\sum_{t'=t}^{T} X_{t'}\right| > C_1\sqrt{T-t+1}\log^{1 + \varphi}T\right\}\right). \tag{$T > T-t+1$}
\end{align*}
\end{proof}

% Lemma 3


% Lemma 4
\begin{lemma}[Theorem \ref{theorem:smscp} Event Bounds]\label{lemma:thm3-event-bound}
Suppose that Assumption \ref{assumption:1} (i) holds so that $t_0$ is the location of the true change-point in the mean or variance of $\mathbf{y}$ and $\min\{t_0,T-t_0\} > cT$ for some $c \in (0,1/2)$.

Define $z_t$ as in (\ref{eq:normalized}) and for $s> t$ define: 
\begin{align*}
    \overline{z}_{t:s} := \frac{1}{s - t} \sum_{t'=t}^s z_{t'}.
\end{align*}
Assume that:
\begin{enumerate}[label=(\roman*)]
    \item $\{y_t\}_{t=1}^T$ is an $\alpha$-mixing process with coefficients that satisfy $\alpha_k \leq e^{-Ck}$ for some $C > 0$.
    \item There exists constants $\delta_1, \; D_1 > 0$ such that $\max_{1\leq t \leq T}\; \E\left[|z_t|^{2+\delta_1}\right]\leq D_1.$ 
    \item There exists constants $\delta_2, \; D_2 > 0$ such that $\max_{1\leq t \leq T}\; \E\left[|z_t^2 - 1|^{2+\delta_2}\right]\leq D_2.$ 
\end{enumerate}
Let $\varphi > 0$ be given, then there exists a constant $C_1$ so that if define the events:
\begin{align*}
    \Omega_1 &:= \bigcap_{t \;:\; cT < t < (1-c)T,\; t \neq t_0} \left\{\left|\sum_{t'=\min\{t_0,t\}}^{\max\{t_0,t\}-1} z_{t'}\right| < C_1\sqrt{|t_0-t|}\log^{1+\varphi} T \right\} \\
    \Omega_2 &:=  \bigcap_{t \;:\; cT < t < (1-c)T,\; t \neq t_0}  \left\{\left|\sum_{t'=\min\{t_0,t\}}^{\max\{t_0,t\}-1} z_{t'}^2 - |t_0-t| \right| < C_1\sqrt{|t_0 -t|}\log^{1+\varphi} T \right\} \\
    \Omega_4 &:=  \bigcap_{t \;:\; cT < t < (1-c)T }  \left\{\left|\sum_{t'=t}^{T} z_{t'}\right| < C_1\sqrt{T-t+1}\log^{1+\varphi} T \right\} \\
    \Omega_5 &:=  \bigcap_{t \;:\; cT < t < (1-c)T }  \left\{\left|\sum_{t'=t}^{T} z_{t'}^2 - (T-t+1) \right| < C_1\sqrt{T}\log^{1+\varphi} T \right\}
\end{align*} 
and the joint event $\Omega := \bigcap_{i=1}^4 \Omega_i $, then we have $\lim_{T\to\infty} \Pr(\Omega) = 1$. 

If we further assume that $y_t \sim \mathcal{N}(0,1)$ for $t < t_0$ and $y_t \sim \mathcal{N}(b_0,s^2_0)$ for $t \geq t_0$ and define the events: 
\begin{align*}
    \mathcal{E}_1 &:= \bigcap_{t \;:\; cT < t < (1-c)T,\; t \neq t_0} \left\{\left|\sum_{t'=\min\{t_0,t\}}^{\max\{t_0,t\}-1} z_{t'}\right| < 2\sqrt{T \log T} \right\} \\
    \mathcal{E}_2 &:=  \bigcap_{t \;:\; cT < t < (1-c)T,\; t \neq t_0}  \left\{\left|\sum_{t'=\min\{t_0,t\}}^{\max\{t_0,t\}-1} z_{t'}^2 - |t_0-t| \right| < 8\sqrt{T \log T} \right\}  \\
    \mathcal{E}_3 &:= \bigcap_{t \;:\; cT < t < (1-c)T} \left\{\left|\sum_{t'=t}^T z_{t'}\right| < 2\sqrt{T \log T} \right\} \\
    \mathcal{E}_4 &:= \bigcap_{cT < t < t_0} \left\{\left|\sum_{t'=t}^{t_0-1} (z_{t'} - \overline{z}_{t:(t_0-1)})^2- (t_0 -t - 1)\right| < 8\sqrt{T \log T}\right\} \\
    \mathcal{E}_5 &:= \bigcap_{cT < t < (1-c) T} \left\{\left|\sum_{t'=t}^T (z_{t'} - \overline{z}_{t:T})^2 - (T-t)\right| < 8\sqrt{T \log T}\right\}.
\end{align*}
and the joint event $\mathcal{E} := \bigcap_{i=1}^5 \mathcal{E}_i$, then we have $\lim_{T\to\infty} \Pr(\mathcal{E}) = 1$. 
\end{lemma}

\begin{proof}
From the definition in (\ref{eq:normalized}), $z_t$ is a continuous and therefore measurable function of $y_t$, as is $z_t^2 - 1$. Therefore, for any $t \geq 1$ and $k \geq 0$, we have:
\begin{align*}
    \sigma(\{z_{t'}\}_{t'\leq t}) &\subset \sigma(\{y_{t'}\}_{t'\leq t}) \\
    \sigma(\{z_{t'}\}_{t'\geq t+k}) &\subset \sigma(\{y_{t'}\}_{t'\geq t+k})
\end{align*}
and:
\begin{align*}
    \sigma(\{z^2_{t'} - 1\}_{t'\leq t}) &\subset \sigma(\{y_{t'}\}_{t'\leq t}) \\
    \sigma(\{z^2_{t'} - 1\}_{t'\geq t+k}) &\subset \sigma(\{y_{t'}\}_{t'\geq t+k})
\end{align*}
which implies:
\begin{align*}
    \max\left\{\alpha_k(\{z_t\}_{t\geq 1}),\; \alpha_k(\{z^2_t - 1\}_{t\geq 1})\right\} \leq \alpha_k(\{y_t\}_{t\geq 1}) \leq e^{-Ck}
\end{align*}
i.e. $\{z_t\}_{t=1}^\infty$ and $\{z^2_t-1\}_{t=1}^\infty$ are both $\alpha$-mixing processes with coefficients that satisfy the inequality above. Combined with assumptions (ii) and (iii) in the statement of Lemma \ref{lemma:thm3-event-bound}, the conditions of Lemma \ref{lemma:Padilla23} are met for both  $\{z_t\}_{t=1}^\infty$ and $\{z^2_t-1\}_{t=1}^\infty$ \textcolor{red}{we might need to be careful here because stationary only holds before the change point, and after the change point, so we have to apply the result separately to each case}. Therefore, Lemma \ref{lemma:2} implies there exists some $K_1$ so that:
\small
\begin{align*}
    \Pr \left(\bigcup_{t=cT}^{t_0-1}\left\{ \left|\sum_{t'=t}^{t_0-1} z_{t'}\right| \geq K_1\sqrt{t_0-t}\log^{1+\varphi}T\right\}\right) + \Pr \left(\bigcup_{t=t_0+1}^{(1-c)T}\left\{ \left|\sum_{t'=t_0}^{t-1} z_{t'}\right| \geq K_1\sqrt{t_0-t}\log^{1+\varphi}T\right\}\right)
    &\leq \frac{2}{\log^{2\varphi} T}
\end{align*}
\normalsize
some $K_2 > 0$ so that:
\scriptsize
\begin{align*}
    \Pr \left(\bigcup_{t=cT}^{t_0-1}\left\{ \left|\sum_{t'=t}^{t_0-1} (z^2_{t'}-1)\right| \geq K_1\sqrt{t_0-t}\log^{1+\varphi}T\right\}\right) + \Pr \left(\bigcup_{t=t_0+1}^{(1-c)T}\left\{ \left|\sum_{t'=t_0}^{t-1} (z^2_{t'}-1)\right| \geq K_1\sqrt{t_0-t}\log^{1+\varphi}T\right\}\right)
    &\leq \frac{2}{\log^{2\varphi} T} 
\end{align*}
\normalsize
\textcolor{red}{the stationarity might be issue in the next two displays:} some $K_3 > 0$ so that:
\small
\begin{align*}
    \Pr \left(\bigcup_{t=cT}^{(1-c)T}\left\{ \left|\sum_{t'=t}^T z_{t'}\right| \geq K_3\sqrt{T-t+1}\log^{1+\varphi}T\right\}\right) &\leq \frac{1}{\log^{2\varphi} T} 
\end{align*}
\normalsize
and some $K_4 > 0$ so that:
\small
\begin{align*}
    \Pr \left(\bigcup_{t=cT}^{(1-c)T}\left\{ \left|\sum_{t'=t}^T (z^2_{t'}-1)\right| \geq K_4\sqrt{T}\log^{1+\varphi}T\right\}\right)&\leq \Pr \left(\bigcup_{t=cT}^{(1-c)T}\left\{ \left|\sum_{t'=t}^T (z^2_{t'}-1)\right| \geq K_4\sqrt{T-t+1}\log^{1+\varphi}T\right\}\right) \tag{$T>T-t+1$}\\
    &\leq \frac{1}{\log^{2\varphi} T}. 
\end{align*}
\normalsize
Let $C_1 = \max\{K_1,K_2,K_3,K_4\}$, then:
\small
\begin{align*}
    \Pr(\Omega_1^c) &=\Pr\left(\bigcup_{t \;:\; cT < t < (1-c)T,\; t \neq t_0} \left\{\left|\sum_{t'=\min\{t_0,t\}}^{\max\{t_0,t\}-1} z_{t'}\right| > C_1\sqrt{|t_0-t|}\log^{1+\varphi} T \right\} \right) \\
    &\leq \Pr \left(\bigcup_{t=cT}^{t_0-1}\left\{ \left|\sum_{t'=t}^{t_0-1} z_{t'}\right| > K_1\sqrt{t_0-t}\log^{1+\varphi}T\right\}\right) + \Pr \left(\bigcup_{t=t_0+1}^{(1-c)T}\left\{ \left|\sum_{t'=t_0}^{t-1} z_{t'}\right| > K_1\sqrt{t-t_0}\log^{1+\varphi}T\right\}\right) \tag{union bound and $C_1 \geq K_1$} \\
    &\leq \frac{2}{\log^{2\varphi} T}
\end{align*}
\normalsize
Noting that:
\begin{align*}
    \sum_{t'=\min\{t_0,t\}}^{\max\{t_0,t\}-1} (z_{t'}^2 - 1)  = \sum_{t'=\min\{t_0,t\}}^{\max\{t_0,t\}-1} z_{t'}^2 - |t_0-t| 
\end{align*}
and:
\begin{align*}
    \sum_{t'=t}^{T} (z_{t'}^2 - 1)  = \sum_{t'=t}^{T} z_{t'}^2 - (T-t+1) 
\end{align*}
then an identical argument gives:
\begin{align*}
    \Pr(\Omega_2^c) \leq  \frac{2}{\log^{2\varphi} T}.
\end{align*}
and:
\begin{align*}
    \max\{\Pr(\Omega_3^c), \Pr(\Omega_4^c)\} \leq  \frac{1}{\log^{2\varphi} T}.
\end{align*}
so:
\begin{align*}
    \lim_{T\to\infty}\Pr(\Omega) &= 1 - \lim_{T\to\infty}\Pr(\cup_{i=1}^3 \Omega_i^c) \\
    &\geq 1 - \lim_{T\to\infty}\sum_{i=1}^4  \Pr(\Omega_i^c) \tag{union bound} \\
    &\geq 1 - \lim_{T\to\infty} \frac{6}{\log^{2\varphi} T} \\
    &= 1.
\end{align*}

If $y_t \sim \mathcal{N}(0,1)$ for $t < t_0$ and $y_t \sim \mathcal{N}(b_0,s^2_0)$ for $t \geq t_0$, then $z_t \overset{\text{i.i.d.}}{\sim} \mathcal{N}(0,1)$. First we have:
\small
\begin{align*}
    \Pr(\mathcal{E}^c_1) &= \Pr\left(\bigcup_{t \;:\; cT < t < (1-c)T\; t \neq t_0} \left\{\left|\sum_{t'=\min\{t_0,t\}}^{\max\{t_0,t\}-1} z_{t'}\right| \geq 2\sqrt{T \log T} \right\} \right) \\
    &\leq \sum_{t = cT}^{t_0 - 1} \Pr\left(\left|\sum_{t'=t}^{t_0-1} z_{t'}\right| \geq 2\sqrt{T \log T} \right) + \sum_{t = t_0+1}^{(1-c)T} \Pr\left(\left|\sum_{t'=t_0}^{t-1} z_{t'}\right| \geq 2\sqrt{T \log T} \right) \tag{union bound} \\
    &\leq \sum_{t = cT}^{t_0 - 1} 2\exp\left[-\frac{2 T\log T}{t_0-t}\right] + \sum_{t = t_0+1}^{(1-c)T} 2\exp\left[-\frac{2 T\log T}{t-t_0}\right] \tag{Chernoff bound (\ref{eq:chernoff})} \\
    &\leq \sum_{t = cT}^{t_0 - 1} \frac{2}{T^2}  + \sum_{t = t_0+1}^{(1-c)T} \frac{2}{T^2} \tag{$T > |t_0-t|$} \\
    &\leq \frac{2(1-2c)}{T}
\end{align*}
\normalsize
and:
\small
\begin{align*}
    \Pr(\mathcal{E}^c_3) &= \Pr\left(\bigcup_{t= cT}^{(1-c)T} \left\{\left|\sum_{t'=t}^T z_{t'}\right| \geq 2\sqrt{T \log T} \right\} \right) \\
    &\leq \sum_{t = cT}^{(1-c)T} \Pr\left(\left|\sum_{t'=t}^T z_{t'}\right| \geq 2\sqrt{T \log T} \right)\tag{union bound} \\
    &\leq \sum_{t = cT}^{(1-c)T}  2\exp\left[-\frac{2 T\log T}{T-t+1}\right] \tag{Chernoff bound (\ref{eq:chernoff})} \\
    &\leq \sum_{t = cT}^{(1-c)T}  \frac{2}{T^2}  \tag{$T > T-t+1$} \\
    &\leq \frac{2(1-2c)}{T}.
\end{align*}
\normalsize
Next, we have:
\begin{align*}
    \sum_{t'=\min\{t_0,t\}}^{\max\{t_0,t\}-1} z_{t'}^2 \sim \chi^2_{|t_0-t|} \\
\end{align*}
and for $t < t_0$:
\begin{align*}
    \sum_{t'=t}^{t_0-1} (z_{t'} - \overline{z}_{t:(t_0-1)})^2 \sim \chi^2_{t_0-t-1}
\end{align*}
so if $|t-t_0| > 8\sqrt{T\log T}$ we get:
\small
\begin{align*}
    \Pr(\mathcal{E}^c_2) &= \Pr\left(\bigcup_{t \;:\; cT < t < (1-c)T,\; t \neq t_0}  \left\{\left|\sum_{t'=\min\{t_0,t\}}^{\max\{t_0,t\}-1} z_{t'}^2 - |t_0-t| \right| \geq 8\sqrt{T \log T} \right\} \right) \\
    &\leq \sum_{t = cT}^{t_0 - 1} \Pr\left(\left|\sum_{t'=t}^{t_0-1} z^2_{t'} - (t_0-t)\right| \geq 8\sqrt{T \log T} \right) + \sum_{t = t_0+1}^{(1-c)T} \Pr\left(\left|\sum_{t'=t_0}^{t-1} z^2_{t'} - (t-t_0)\right| \geq 8\sqrt{T \log T} \right) \tag{union bound} \\
    &\leq \sum_{t = cT}^{t_0 - 1} 2 \exp\left[- \frac{8 T \log T}{t_0-t} \right] + \sum_{t = t_0+1}^{(1-c)T} 2 \exp\left[-  \frac{8 T \log T}{t-t_0} \right] \tag{by (\ref{eq:chi2-ineq})} \\
    &\leq \sum_{t = cT}^{t_0 - 1} \frac{2}{T^8} + \sum_{t = t_0+1}^{(1-c)T} \frac{2}{T^8} \tag{$T > |t_0 - t|$} \\
    &\leq \frac{2(1-2c)}{T^7}
\end{align*}
\normalsize
and:
\small
\begin{align*}
    \Pr(\mathcal{E}^c_4) &= \Pr\left(\bigcup_{t \;:\; cT < t < t_0}  \left\{\left|\sum_{t'=t}^{t_0-1} (z_{t'} - \overline{z}_{t:(t_0-1)})^2- (t_0 -t - 1)\right| \geq 8\sqrt{T \log T}\right\}\right) \\
    &\leq \sum_{t = cT}^{t_0 - 1} \Pr\left( \left|\sum_{t'=t}^{t_0-1} (z_{t'} - \overline{z}_{t:(t_0-1)})^2- (t_0 -t - 1)\right| \geq 8\sqrt{T \log T} \right) \tag{union bound} \\
    &\leq \sum_{t = cT}^{t_0 - 1} 2 \exp\left[- \frac{8 T \log T}{t_0-t-1} \right] \tag{by (\ref{eq:chi2-ineq})} \\
    &\leq \sum_{t = cT}^{t_0 - 1} \frac{2}{T^8} \tag{$T > t_0 - t - 1$} \\
    &\leq \frac{2(1-2c)}{T^7}. \tag{$t_0 < (1-c)T$}
\end{align*}
\normalsize
Similarly:
\begin{align*}
    \sum_{t'=t}^T (z_{t'} - \overline{z}_{t:T})^2 &\sim \chi^2_{T-t}
\end{align*}
so if $T$ is large enough so that $cT > 8\sqrt{T\log T}$:
\small
\begin{align*}
    \Pr(\mathcal{E}^c_5) &= \Pr\left(\bigcup_{cT < t < (1-c) T} \left\{\left|\sum_{t'=t}^T (z_{t'} - \overline{z}_{t:T})^2 - (T-t)\right| \geq 8\sqrt{T \log T}\right\}\right) \\
    &\leq \sum_{t=cT}^{(1-c)T} \Pr\left(\left|\sum_{t'=t}^T (z_{t'} - \overline{z}_{t:T})^2 - (T-t)\right| \geq 8\sqrt{T \log T}\right) \tag{union bound} \\
    &\leq \sum_{t=cT}^{(1-c)T} 2 \exp\left[-\frac{8 T \log T}{T-t} \right] \tag{by (\ref{eq:chi2-ineq})} \\
    &\leq \sum_{t=cT}^{(1-c)T} \frac{2}{T^8} \tag{$T > T-t$} \\
    &= \frac{2(1-2c)}{T^7}.
\end{align*}
\normalsize
Altogether we have:
\begin{align*}
    \lim_{T\to\infty}\Pr(\mathcal{E}) &= 1 - \lim_{T\to\infty}P(\cup_{i=1}^5\mathcal{E}_i^c) \\
    &\geq 1 - \lim_{T\to\infty}\sum_{i=1}^5 P(\mathcal{E}_i^c) \tag{union bound} \\
    &\geq 1 - \lim_{T\to\infty}\frac{4(1-2c)}{T} - \frac{6(1-2c)}{T^7} \\
    &=1.
\end{align*}
\end{proof}