\subsection{Notation and Definitions}
\label{app:notation}

\subsubsection{Asymptotic Analysis} 

In the following proofs, for some functions $f$ and $g$ we use the notation $f(T) = \mathcal{O}(g(T))$ to mean that there exists some constant $M > 0$ and some $T^* > 0$ so that for all $T > T^*$ we have: $$|f(T)| \leq M g(T).$$ Similarly, we use $g(T) \gtrsim f(T)$ to indicate that there exists $M > 0$ so that for any $T$ we have $f(T) \leq M g(T).$ We also use $f(T) = o(g(T))$ to mean:$$\lim_{T\to\infty} \frac{f(T)}{g(T)} = 0$$ and $f(T) \sim g(T)$ to mean: $$\lim_{T\to\infty} \frac{f(T)}{g(T)} = 1.$$ 

\subsubsection{Random Variables} 

For a generic real-valued random variable $X$, we use $p(x)$ to denote the density of $X$ when it exists. If $\sigma(X)$ is the $\sigma$-algebra generated by $X$, then we use $\Pr(\cdot)$ to indicate the measure induced by $X$ on the measurable space $(\mathbb{R},\sigma(X))$.

\subsubsection{Sub-Gaussian Distributions}

We write $X\in\mathcal{SG}(\sigma)$ to denote that the random variable $X$ has a sub-Gaussian distribution with parameter $\sigma$, i.e. for all $\lambda \in \mathbb{R}$: 
\begin{align} 
    \E[\exp(\lambda X)] \leq \exp\left[\frac{\lambda^2\sigma^2}{2}\right]. \label{def:sub-gaussian}
\end{align}
For $t, \lambda > 0$, we have:
\begin{align}
     \Pr(|X| > t) &\leq \Pr(X > t) + \Pr(-X > t) \tag{union bound} \\
     &= \Pr(e^{\lambda X} > e^{\lambda t}) + \Pr(e^{-\lambda X} > e^{\lambda t}) \notag \\
     &\leq \E[\exp(\lambda X - \lambda t)] + \E[\exp(-\lambda X - \lambda t)] \tag{Markov inequality} \\
     &\leq 2\exp\left(\frac{\lambda^2\sigma^2}{2} - \lambda t\right). \tag{by (\ref{def:sub-gaussian})} 
\end{align}
The last bound is minimized be setting $\lambda = \frac{t}{\sigma^2}$, which yields the Chernoff bound:
\begin{align} \label{eq:chernoff}
    \Pr(|X| > t) \leq 2\exp\left[-\frac{t^2}{2\sigma^2}\right]
\end{align}
The sub-Gaussian norm of a random variable is defined as $\lVert X \rVert_{\psi_2} := \inf\{t > 0 \::\: \E[\exp(X^2 /t^2)] \leq 2\}$. By Proposition 2.5.2 of \cite{Vershynin18}, $X$ is sub-Gaussian if and only if $\lVert X \rVert_{\psi_2} < \infty$.

\subsubsection{Sub-Exponential Distributions}

We write $X\in\mathcal{SE}(\nu, \alpha)$ to denote that the random variable $X$ has a sub-Exponential distribution with parameters $\nu$ and $\alpha$, i.e. for all $|\lambda| \leq \frac{1}{\alpha}$: 
\begin{align*}
    \E[\exp(\lambda X)] \leq \exp\left[\frac{\lambda^2\nu^2}{2}\right].
\end{align*}
If $X\in\mathcal{SE}(\nu, \alpha)$, then by Proposition 2.9 of \cite{Wainwright19}:
\begin{align}\label{eq:wainwright_prop_2.9}
    \Pr(|X - \E[x]| \geq t) \leq 
    \begin{cases}
        2 \exp\left[-\frac{t^2}{2\nu^2}\right], & \text{if } t \in \left(0, \frac{\nu^2}{\alpha}\right], \\
        2 \exp\left[-\frac{t}{2\alpha}\right], & \text{if } t \in \left(\frac{\nu^2}{\alpha}, \infty\right).
    \end{cases}
\end{align}
As per Example 2.11 of \cite{Wainwright19}, if $X \sim \chi^2_n$, then $X\in\mathcal{SE}(2\sqrt{n}, 4)$, and thus: 
\begin{align} \label{eq:chi2-ineq}
    \Pr\left(\frac{|X - n|}{n} \geq t\right) \leq 
    \begin{cases}
        2 \exp\left[-\frac{n t^2}{8}\right], & \text{if } t \in \left(0, 1\right], \\
        2 \exp\left[-\frac{n t}{8}\right], & \text{if } t \in \left(1, \infty\right).
    \end{cases}
\end{align}
The sub-exponential norm of a random variable is defined as $\lVert X \rVert_{\psi_1} := \inf\{t > 0 \::\: \E[\exp(X /t)] \leq 2\}$. By Proposition 2.7.1 of \cite{Vershynin18}, $X$ is sub-exponential if and only if $\lVert X \rVert_{\psi_1} < \infty$.

\subsubsection{$\alpha$-mixing}

\begin{definition}\label{def:alpha-mixing}
Let $\{X_t\}_{t\in \mathbb{Z}}$ be a stochastic process on the probability space $(\Omega, \mathcal{F}, \Pr)$, then $\{X_t\}_{t\in \mathbb{Z}}$ is said to be $\alpha$-mixing if:
\begin{align*}
    \lim_{k\to\infty} \alpha_k(\{X_t\}_{t\in\mathbb{Z}}) = 0,
\end{align*}
where:
\begin{align*}
    \alpha_k(\{X_t\}_{t\in\mathbb{Z}}) := \sup_{t\in\mathbb{Z}} \; \alpha\left(\sigma(\{X_s\}_{s \leq t}), \; \sigma(\{X_s\}_{s \geq t + k})\right).
\end{align*}
Here $\sigma(Y)$ is the $\sigma$-algebra generated by $Y$ and the strong mixing, or $\alpha$-mixing, coefficient between two $\sigma$-algebras $\mathcal{A}, \mathcal{B} \subseteq \mathcal{F}$ is defined as:
\begin{align*}
    \alpha(\mathcal{A}, \mathcal{B}) := \sup_{A\in\mathcal{A}, B\in\mathcal{B}} |\Pr(A \cap B) - \Pr(A)\Pr(B)|.
\end{align*}
To simplify notation, we will often write $\alpha_k$ in place of $\alpha_k(\{X_t\}_{t\in\mathbb{Z}})$.
\end{definition}
