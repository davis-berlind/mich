\subsection{Notation and Preliminary Results}
\label{app:localization-smcp}

\textbf{Assymptotic Characterizations:} In the following proofs, for some functions $f$ and $g$ we use the notation $f(T) = \mathcal{O}(g(T))$ to mean that there exists some constant $M > 0$ and some $T^* > 0$ so that for all $T > T^*$ we have: $$f(T) \leq M g(T).$$ Similarly, we use $g(T) \gtrsim f(T)$ to indicate that there exists $M > 0$ so that for any $T$ we have $f(T) \leq M g(T).$ We use the notation $f(T) \sim g(T)$ to mean: $$\lim_{T\to\infty} \frac{f(T)}{g(T)} = 1.$$ 

\textbf{Random Variables:} For a generic real-valued random variable $X$, we use $p(x)$ to denote the density of $X$ when it exists. If $\sigma(X)$ is the $\sigma$-algebra generated by $X$, then we use $\mathbb{P}(\cdot)$ to indicate the measure induced by $X$ on the measurable space $(\sigma(X), \mathbb{R})$.

\textbf{$\chi^2$-Concentration:} A random variable $X$ is said to be sub-exponential with parameters $\nu$ and $\alpha$ if:
\begin{align*}
    \mathbb{P}(|X - \E[x]| \geq t) \leq 
    \begin{cases}
        2 \exp\left[-\frac{t^2}{2\nu^2}\right], & \text{if } t \in \left(0, \frac{\nu^2}{\alpha}\right], \\
        2 \exp\left[-\frac{t}{2\alpha}\right], & \text{if } t \in \left(\frac{\nu^2}{\alpha}, \infty\right).
    \end{cases}
\end{align*}
As per Example 2.11 of \cite{Wainwright19}, if $X \sim \chi^2_n$, then $X$ is sub-exponential with $\nu = 2\sqrt{n}$ and $\alpha = 4$, and thus: 
\begin{align} \label{eq:chi2-ineq}
    \mathbb{P}\left(\frac{1}{n}|X - n| \geq t\right) \leq 
    \begin{cases}
        2 \exp\left[-\frac{n t^2}{8}\right], & \text{if } t \in \left(0, 1\right], \\
        2 \exp\left[-\frac{n t}{8}\right], & \text{if } t \in \left(1, \infty\right).
    \end{cases}
\end{align}