\subsection{Estimating \texorpdfstring{$\mu_0$}{mu0} and \texorpdfstring{$\lambda_0$}{lambda0} with Empirical Bayes}
\label{app:empirical-bayes}

In most cases, the intercept $\mu_0$ and initial scale $\lambda_0$ terms in (\ref{eq:mu_t}) and (\ref{eq:lambda_t}) are unknown nuisance parameters that must be estimated. One approach is to just set $\mu_0 = 0$ and $\lambda_0 = 1$ in Algorithm \ref{alg:mich}. Then, if $\mu_0 \neq 0$ or $\lambda_0 \neq 1$, Algorithm \ref{alg:mich} will attempt to detect a change-point at $t=1$ and center $\mu_{t}$ and $\lambda_t$ around $\mu_0$ and $\lambda_0$ respectively for all $t$ up until the first real change-point. The problem with fitting $\mu_0$ and $\lambda_0$ this way is that it requires MICH to search for a change-point whose location is known. Searching for an additional change-point requires MICH to estimate  $\mathcal{O}(T)$ additional parameters, when we only actually needed to fit two parameters $\mu_0$ and $\lambda_0$.\footnote{E.g. in the case of $\mu_0 \neq 0$ and $\lambda_0 \neq 1$, we end up having to estimate $5T$ additional parameters $\{\overline{b}_t, \overline{\tau}_t, \overline{u}_t, \overline{v}_t, \overline{\pi}_t\}_{t=1}^T$.} Another problem with this approach is that we risk missing a real change-point, e.g. if the are $J^*>0$ true changes and we set $J=J^*$ to fit MICH, we will end up using one component of $J$ to fit $\mu_0$ and $\lambda_0$ leaving only $J^*-1$ components to fit the remaining $J^*$ true change-points. 

To circumvent these challenges, we adopt an empirical Bayes (EB) approach for estimating $\mu_0$ and $\lambda_0$. We seek the values $\mu_0$ and $\lambda_0$ that maximize the evidence $\log p(\mathbf{y}_{1:T};\mu_0,\lambda_0)$. Note that in the expression (\ref{eq:elbo}), $\text{ELBO}(q;\mu_0,\lambda_0)$ is equivalent to $\log p(\mathbf{y}_{1:T};\mu_0,\lambda_0)$ plus a term that does not depend on $\mu_0$ or $\lambda_0$. Therefore, for fixed $q$, the $\mu_0$ and $\lambda_0$ that maximize the evidence are equivalent to: 
\begin{align}
    \argmax{\{\mu_0,\lambda_0\}\in\mathbb{R}\times\mathbb{R}_{++}} \; \text{ELBO}(q\:;\mu_0,\lambda_0),
\end{align}
which in turn is equivalent to:
\begin{align}\label{eq:EB-max}
    \argmax{\{\mu_0,\lambda_0\}\in\mathbb{R}\times\mathbb{R}_{++}} \; \E_{q} \left[\log p\left(\mathbf{y}_{1:T} \:|\: \boldsymbol{\Theta};\mu_0,\lambda_0\right)\right]. 
\end{align}
\cite{Wang20} use this same approach, and as they note, optimizing the ELBO over $\mu_0$ and $\lambda_0$ constitutes the M-step of an EM algorithm where the E-step is approximate (\citealp{Dempster77, Heskes03, Neal98}) do to the use of $q$. Define $\tilde{r}_t$, $\overline{\lambda}_t$ and $\delta_t$ as in (\ref{eq:mod-resid})-(\ref{eq:delta}) along with: 
\begin{align}
    \tilde{r}_{-0t} &:= \tilde{r}_t + \mu_0 \\
    \overline{\lambda}_{-0t} &:= \lambda_0^{-1}\overline{\lambda}_t
\end{align}
then (\ref{eq:EB-max}) can be rewritten as:
\begin{align}\label{eq:EB-max-simple}
    \argmax{\{\mu_0,\lambda_0\}\in\mathbb{R}\times\mathbb{R}_{++}} \; \frac{T }{2}\log\lambda_0 - \frac{\lambda_0}{2}\sum_{t=1}^{T} \overline{\lambda}_{-0t}\left[\left(\tilde{r}_{-0t}-\mu_0\right)^2 + \delta_t\right].
\end{align}
The solutions to (\ref{eq:EB-max-simple}) is given by:
\begin{align}
    \hat{\mu}_0 &= \frac{\sum_{t=1}^{T} \overline{\lambda}_{-0t}\tilde{r}_{-0t}}{\sum_{t=1}^{T} \overline{\lambda}_{-0t}} \label{eq:EB-max-solution-mu} \\
    \hat{\lambda}_0 &= \left[\frac{\sum_{t=1}^{T} \overline{\lambda}_{-0t}[\left(\tilde{r}_{-0t}-\hat{\mu}_0\right)^2 + \delta_t]}{T}\right]^{-1}. \label{eq:EB-max-solution-lambda}
\end{align}
We incorporate the estimation step for $\mu_0$ and $\lambda_0$ into Algorithm \ref{alg:mich-eb}.

\begin{algorithm}[!h]
\label{alg:mich-eb}
\caption{Empirical Bayes Variational Approximation to MICH Posterior.}

\small
\SetAlgoLined
  Inputs: $\mathbf{y}_{1:T},\:L,\:K,\:J,\:\{\omega_\ell,\boldsymbol{\pi}_\ell\}_{\ell=1}^L,\:\{u_k,v_k,\boldsymbol{\pi}_k\}_{k=1}^K,\:\{\omega_j,u_j,v_j,\boldsymbol{\pi}_j\}_{j=1}^J;$ \\
  Initialize: $\mu_0,\: \lambda_0,\: \overline{\boldsymbol{\Theta}} := \{\{\overline{\boldsymbol{\theta}}_\ell\}_{\ell=1}^L$, $\{\overline{\boldsymbol{\theta}}_k\}_{k=1}^K$, $\{\overline{\boldsymbol{\theta}}_j\}_{j=1}^J\}$;
  
  \Repeat {Convergence} {
    \For{$\ell=1$ \KwTo $L$} {
      $\tilde{r}_{-\ell t} := y_t - \mu_0- \sum_{\ell' \neq \ell}^L \E[\mu_{\ell' t}] - \sum_{j =1}^J \E[\lambda_{jt} \mu_{jt}] / \E[\lambda_{jt}]$ \tcp*{l\textsuperscript{th} partial mean residual}
      $\overline{\lambda}_{t} := \lambda_0\prod_{k=1}^K \E[\lambda_{kt}]\prod_{j=1}\E[\lambda_{jt}]$ \tcp*{precision of residual} 
      $\overline{\boldsymbol{\theta}}_\ell := \texttt{mean-scp}(\tilde{\mathbf{r}}_{-\ell} \:;\: \overline{\boldsymbol{\lambda}}_{1:T}, \omega_{\ell}, \boldsymbol{\boldsymbol{\pi}}_{\ell})$ \tcp*{update mean-scp parameters}
    }
    \For{$k=1$ \KwTo $K$} {
      $\tilde{r}_{t} := y_t - \mu_0 - \sum_{\ell = 1}^L \E[\mu_{\ell t}] - \sum_{j=1}^J \E[\lambda_{jt} \mu_{jt}] / \E[\lambda_{jt}]$ \tcp*{mean residual} 
      $\overline{\lambda}_{-kt} := \lambda_0\prod_{k' \neq k} \E[\lambda_{k't}]\prod_{j=1}^J\E[\lambda_{jt}] $ \tcp*{k\textsuperscript{th} partial scale residual} 
      %% $\delta_{t} := \sum_{\ell=1}^L  \Var(\mu_{\ell t} ) + \sum_{j=1}^J[\E[\lambda_{jt} \mu^2_{jt}] / \E[\lambda_{jt}] - (\E[\lambda_{jt} \mu_{jt}] / \E[\lambda_{jt}])^2 ];$ \tcp*{variance correction term}\\
      Compute $\boldsymbol{\delta}_{1:T}$, $\tilde{\mathbf{v}}_{k}$, and $\tilde{\boldsymbol{\boldsymbol{\pi}}}_{k}$ by (\ref{eq:delta}), (\ref{eq:v-k-corrected}), and (\ref{eq:pi-k-corrected}) \tcp*{variance corrected priors} 
      $\overline{\boldsymbol{\theta}}_k := \texttt{var-scp}(\tilde{\mathbf{r}}_{1:T} 
      \:;\:\overline{\boldsymbol{\lambda}}_{-k}, u_k, \tilde{\mathbf{v}}_{k}, \tilde{\boldsymbol{\boldsymbol{\pi}}}_k)$ \tcp*{update var-scp parameters}
    }
    \For{$j=1$ \KwTo $J$} {
      $\tilde{r}_{-jt} := y_t - \mu_0 - \sum_{\ell = 1}^L \E[\mu_{\ell t}] - \sum_{j' \neq j} \E[\lambda_{j't} \mu_{j't}] / \E[\lambda_{j't}]$ \tcp*{j\textsuperscript{th} partial mean residual} 
      $\overline{\lambda}_{-jt} := \lambda_0\prod_{k=1}^K \E[\lambda_{kt}]\prod_{j' \neq j}\E[\lambda_{j't}] $ \tcp*{j\textsuperscript{th} partial scale residual}
      %%$\delta_{-jt} := \sum_{\ell=1}^L  \Var(\mu_{\ell t} ) + \sum_{j' \neq j}[\E[\lambda_{j't} \mu^2_{j't}] / \E[\lambda_{j't}] - (\E[\lambda_{j't} \mu_{j't}] / \E[\lambda_{j't}])^2 ]$ \tcp*{j\textsuperscript{th} variance correction term} 
      Compute $\boldsymbol{\delta}_{-j}$, $\tilde{\mathbf{v}}_{j}$, and $\tilde{\boldsymbol{\boldsymbol{\pi}}}_{j}$ by (\ref{eq:delta-j}), (\ref{eq:v-j-corrected}), and (\ref{eq:pi-j-corrected}) \tcp*{variance corrected priors} 
      $\overline{\boldsymbol{\theta}}_j := \texttt{meanvar-scp}(\tilde{\mathbf{r}}_{-j} \:;\:\overline{\boldsymbol{\lambda}}_{-j}, \omega_j, u_j, \tilde{\mathbf{v}}_{j}, \tilde{\boldsymbol{\boldsymbol{\pi}}}_j)$ \tcp*{update meanvar-scp parameters}
    }
    $\tilde{r}_{-0t} := y_t - \sum_{\ell = 1}^L \E[\mu_{\ell t}] - \sum_{j=1}^J \E[\lambda_{j't} \mu_{j't}] / \E[\lambda_{j't}]\;$; \\
    $\overline{\lambda}_{-0t} := \prod_{k=1}^K \E[\lambda_{kt}]\prod_{j=1}^J\E[\lambda_{j't}]\;$; \\
    Compute $\mu_0$ and $\lambda_0$ by (\ref{eq:EB-max-solution-mu}) and (\ref{eq:EB-max-solution-lambda}) \tcp*{EB step} 
  }
  \Return{Model Parameters: $\mu_0,\:\lambda_0,\:\overline{\boldsymbol{\Theta}}$}.
\end{algorithm}


Note that Algorithm \ref{alg:mich-eb} embeds Algorithm \ref{alg:mich} with the additional step of maximizing the ELBO with respect to $\mu_0$ and $\lambda_0$ while holding the approximate posterior $q$ fixed. Therefore, Algorithm \ref{alg:mich-eb} also defines a coordinate ascent procedure. The objective function in (\ref{eq:EB-max-simple}) is continuously differentiable with respect to $\mu_0$ and $\lambda_0$, so if we can show that $\hat{\mu}_0$ and $\hat{\lambda}_0$ are the unique maximizers of (\ref{eq:EB-max-simple}), then the conditions of Proposition 2.7.1 of \cite{Bertsekas97} will still hold even with the added coordinate ascent step of maximizing the ELBO with respect to $\mu_0$ and $\lambda_0$. 

To prove the uniqueness of $\hat{\mu}_0$ and $\hat{\lambda}_0$, first note that the term:
\begin{align*}
    -\frac{1}{2}\sum_{t=1}^{T} \overline{\lambda}_{-0t}\left(\tilde{r}_{-0t}-\mu_0\right)^2
\end{align*}
is strictly concave as a function of $\mu_0$ and is maximized at $\hat{\mu}_0$. Therefore, for any $\mu_0\in\mathbb{R}\setminus\{\hat{\mu}_0\}$ we have:
\begin{align*}
    -\frac{1}{2}\sum_{t=1}^{T} \overline{\lambda}_{-0t}\left(\tilde{r}_{-0t}-\hat{\mu}_0\right)^2 > -\frac{1}{2}\sum_{t=1}^{T} \overline{\lambda}_{-0t}\left(\tilde{r}_{-0t}-\mu_0\right)^2
\end{align*}
and thus for any $\lambda_0>0$,
\begin{align*}
    \frac{T }{2}\log\lambda_0-\frac{\lambda_0}{2}\sum_{t=1}^{T} \overline{\lambda}_{-0t}\left(\tilde{r}_{-0t}-\hat{\mu}_0\right)^2 > \frac{T }{2}\log\lambda_0-\frac{\lambda_0}{2}\sum_{t=1}^{T} \overline{\lambda}_{-0t}\left(\tilde{r}_{-0t}-\mu_0\right)^2.
\end{align*}
Since the $\log$ is strictly concave, the left-hand side of the above inequality is uniquely maximized by $\hat{\lambda}_0$. So for any $\lambda_0 > 0$, we have 
\begin{align*}
    \frac{T }{2}\log\hat{\lambda}_0-\frac{\hat{\lambda}_0}{2}\sum_{t=1}^{T} \overline{\lambda}_{-0t}\left(\tilde{r}_{-0t}-\hat{\mu}_0\right)^2 &> \frac{T }{2}\log\lambda_0-\frac{\lambda_0}{2}\sum_{t=1}^{T} \overline{\lambda}_{-0t}\left(\tilde{r}_{-0t}-\hat{\mu}_0\right)^2 \\
    &> \frac{T }{2}\log\lambda_0-\frac{\lambda_0}{2}\sum_{t=1}^{T} \overline{\lambda}_{-0t}\left(\tilde{r}_{-0t}-\mu_0\right)^2.
\end{align*}
Proving that $\hat{\mu}_0$ and $\hat{\lambda}_0$ are the global maximizers of (\ref{eq:EB-max-simple}). Therefore, Proposition \ref{prop:coord-ascent} still holds with the included maximization step for $\mu_0$ and $\lambda_0$ and Algorithm \ref{alg:mich-eb} will also converge to a stationary point. 

%\begin{algorithm}
\label{alg:2}
\small
\SetAlgoLined
  Inputs: $J;\:K;\:L;\:B_l;\:B_r;\:\{\tau_j,u_j,v_j,\pi_j\}_{j=1}^J;\:\{\tau_\ell,\pi_\ell\}_{\ell=1}^L;\:\{u_k,v_k,\pi_k\}_{k=1}^K$; \\
  Initialize: $\tilde{\mathbf{r}} := \mathbf{y}$; $\overline{\pmb{\lambda}}:= \mathbf{1}$; $\pmb{\delta}:=\mathbf{0}$; $\mu_0$; $\lambda_0$; $\{\{\overline{b}_{jt}, \overline{\tau}_{jt}, \overline{u}_{jt}, \overline{v}_{jt}, \overline{\pi}_{jt}\}_{t=1}^T\}_{j=1}^J$; $\{\{\overline{b}_{\ell t}, \overline{\tau}_{\ell t}, \{\overline{\pi}_{\ell t}\}_{t=1}^T\}_{\ell=1}^L$; $\{\{\overline{u}_{kt}, \overline{v}_{kt}, \overline{\pi}_{kt}\}_{t=1}^T\}_{k=1}^K$;\\
  
  \Repeat {Convergence} {
    $j:=1;\;\ell:=1;\;k:=1;$ \\
    \While{$j \leq J$} {
      $\tilde{r}_t := \tilde{r}_t + \E_q[\lambda_{jt}]^{-1}\E_q[\lambda_{jt} \mu_{jt}],\; \sforall t$; \texttt{// calculate partial residuals} \\
      $\overline{\lambda}_t := \E_q[\lambda_{jt}]^{-1}\overline{\lambda}_t, \; \sforall t$; \\
      $\delta_t := \delta_t - \E_q[\lambda_{jt}]^{-1}\E_q[\lambda_{jt} \mu^2_{jt}] + (\E_q[\lambda_{jt}]^{-1}\E_q[\lambda_{jt} \mu_{jt}])^2,\; \sforall t;$ \\
      Calculate: $\tilde{v}_{jt}$ and $\tilde{\pi}_{jt}$ by (\ref{eq:mod-v_j}) and (\ref{eq:mod-pi_j}), $\sforall t$; \texttt{// calculate corrected priors} \\
      $\{\overline{b}_{jt}, \overline{\tau}_{jt}, \overline{u}_{jt}, \overline{v}_{jt}, \overline{\pi}_{jt}\}_{t=1}^T := \texttt{SMSCP}\left(\tilde{\mathbf{r}} \:;\:\overline{\pmb{\lambda}}, \tau_j, u_j, \{\tilde{v}_{jt}\}_{t=1}^T, \tilde{\pmb{\pi}}_j, B_l,B_r\right);$ \texttt{// fit model}\\
      Calculate: $\E_q[\lambda_{jt}]$, $\E_q[\lambda_{jt} \mu_{jt}]$, and $\E_q[\lambda_{jt} \mu^2_{jt}]$; \texttt{// update residuals}\\
      $\tilde{r}_t := \tilde{r}_t - \E_q[\lambda_{jt}]^{-1}\E_q[\lambda_{jt} \mu_{jt}],\; \sforall t$; \\
      $\overline{\lambda}_t := \E_q[\lambda_{jt}]\overline{\lambda}_t, \; \sforall t$; \\
      $\delta_t := \delta_t + \E_q[\lambda_{jt}]^{-1}\E_q[\lambda_{jt} \mu^2_{jt}] - (\E_q[\lambda_{jt}]^{-1}\E_q[\lambda_{jt} \mu_{jt}])^2,\; \sforall t;$ \\
      $j := j + 1$; \\
    }
    \While{$\ell \leq L$} {
      $\tilde{r}_t := \tilde{r}_t + \E_q[\mu_{\ell t}],\; \sforall t$; \texttt{// calculate partial residuals} \\
      $\delta_t := \delta_t -  \Var_q(\mu_{\ell t}),\; \sforall t;$ \\
      $\{\overline{b}_{\ell t}, \overline{\tau}_{\ell t}, \overline{\pi}_{\ell t}\}_{t=1}^T := \texttt{SMCP}\left(\tilde{\mathbf{r}} \:;\: \overline{\pmb{\lambda}}, \tau_{\ell}, \pmb{\pi}_{\ell}, B_l,B_r\right);$ \texttt{// fit model} \\
      Calculate: $\E_q[\mu_{\ell t}]$ and $\Var_q(\mu_{\ell t})$; \texttt{// update residuals} \\
      $\tilde{r}_t := \tilde{r}_t - \E_q[\mu_{\ell t}],\; \sforall t$; \\
      $\delta_t := \delta_t + \Var_q(\mu_{\ell t}),\; \sforall t;$ \\
      $\ell := \ell + 1$; \\
    }
    \While{$k \leq K$} {
      $\overline{\lambda}_t := \E_q[\lambda_{k t}]^{-1}\overline{\lambda}_t, \; \sforall t$; \texttt{// calculate partial residuals}\\
      Calculate: $\tilde{v}_{kt}$ and $\tilde{\pi}_{kt}$ by (\ref{eq:mod-v_k}) and (\ref{eq:mod-pi_k}), $\sforall t$; \texttt{// calculate corrected priors} \\
      $\{\overline{u}_{kt}, \overline{v}_{kt}, \overline{\pi}_{kt}\}_{t=1}^T := \texttt{SSCP}\left(\tilde{\mathbf{r}} 
      \:;\:\overline{\pmb{\lambda}}, u_k, \{\tilde{v}_{kt}\}_{t=1}^T, \tilde{\pmb{\pi}}_k, B_l,B_r\right)$; \texttt{// fit model} \\
      Calculate: $\E_q[\lambda_{k t}]$; \texttt{// update residuals} \\
      $\overline{\lambda}_t := \E_q[\lambda_{kt}]\overline{\lambda}_t, \; \sforall t$; \\
      $k := k + 1$; \\
    }
    $\tilde{r}_t := \tilde{r}_t + \mu_0\; \sforall t$; \\
    $\overline{\lambda}_t := \lambda_0{-1}\overline{\lambda}_t, \; \sforall t$; \\
    Calculate: $\mu_0$ and $\lambda_0$ by (\ref{eq:EB-max-solution-mu}) and (\ref{eq:EB-max-solution-lambda});\\ 
    $\tilde{r}_t := \tilde{r}_t - \mu_0\; \sforall t$; \\
    $\overline{\lambda}_t := \lambda_0\overline{\lambda}_t, \; \sforall t$; \\
  }
  \Return{Posterior Parameters: $\mu_0;\lambda_0;\{\{\overline{b}_{jt}, \overline{\tau}_{jt}, \overline{u}_{jt}, \overline{v}_{jt}, \overline{\pi}_{jt}\}_{j=1}^J, \{\overline{b}_{\ell t}, \overline{\tau}_{\ell t}, \overline{\pi}_{\ell t}\}_{\ell=1}^L, \{\overline{u}_{kt}, \overline{v}_{kt}, \overline{\pi}_{kt}\}_{k=1}^K\}_{t=1}^T$}.
  \caption{Variational Bayes Approximation to MICH Posterior with $\mu_0$ and $\lambda_0$ Unknown.}
\end{algorithm}