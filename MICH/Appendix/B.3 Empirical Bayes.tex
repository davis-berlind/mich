\subsection{Estimating \texorpdfstring{$\mu_0$}{mu0} and \texorpdfstring{$\lambda_0$}{lambda0} with Empirical Bayes}
\label{app:empirical-bayes}

The intercept $\mu_0$ and initial scale $\lambda_0$ are taken as known constants in Algorithm \ref{alg:1}; however, in most cases $\mu_0$ and $\lambda_0$ are unknown parameters and must be estimated. One approach to estimating $\mu_0$ and $\lambda_0$ is to just set $\mu_0 = 0$ and $\lambda_0 = 1$ in Algorithm \ref{alg:1} and fit MICH. Under this approach, if $\mu_0 \neq 0$ or $\lambda_0 \neq 1$, then MICH will attempt to detect a change-point at $t=1$ and center $\mu_{t}$ and $\lambda_t$ around $\mu_0$ and $\lambda_0$ respectively for all $t$ up until the first real change-point.

The problem with fitting $\mu_0$ and $\lambda_0$ this way is that it requires MICH to search for a change-point whose location is known. Searching for an additional change-point requires MICH to estimate additional parameters on the order of $\mathcal{O}(T)$, when we only actually needed to fit two nuisance parameters $\mu_0$ and $\lambda_0$.\footnote{E.g. in the case of $\mu_0 \neq 0$ and $\lambda_0 \neq 1$, we end up having to estimate $5T$ additional parameters $\{\overline{b}_t, \overline{\tau}_t, \overline{u}_t, \overline{v}_t, \overline{\pi}_t\}_{t=1}^T$.} Another problem with this approach is that we risk missing a real change-point, e.g. if the are $J^*$ true changes and we set $J=J^*$ to fit MICH, we will end up using one component of $J$ to fit $\mu_0$ and $\lambda_0$ leaving only $J^*-1$ components to fit the remaining $J^*$ true change-points. 

To circumvent these difficulties, we adopt an empirical Bayes (EB) approach to estimate $\mu_0$ and $\lambda_0$. For fixed $q$ and $p$, we can maximize the ELBO with respect to $\mu_0$ and $\lambda_0$. \cite{Wang20} use the same approach to estimate nuisance parameters in the SuSiE model. As these authors note, optimizing over $\mu_0$ and $\lambda_0$ in the ELBO constitutes the M-step of an EM algorithm where the E-step is approximate (\citealp{Dempster77, Heskes03, Neal98}). Note that in the expression for the ELBO, only the marginal log-likelihood depends on $\mu_0$ and $\lambda_0$. So for a fixed approximate posterior $q$, solving:
\begin{align}
    \argmax{\{\mu_0,\lambda_0\}\in\mathbb{R}\times\mathbb{R}_{++}} \; \text{ELBO}(q\:;\mu_0,\lambda_0)
\end{align}
is equivalent to solving:
\begin{align}\label{eq:EB-max}
    \argmax{\{\mu_0,\lambda_0\}\in\mathbb{R}\times\mathbb{R}_{++}} \; \E_{q} \left[\log p\left(\mathbf{y} \:|\: \theta;\mu_0,\lambda_0\right)\right].
\end{align}
If we define:
\begin{align}
    r_{-0t} &:= y_t - \sum_{j=1}^J \mu_{jt} - \sum_{\ell=1}^L \mu_{\ell t} \\
    \lambda_{-0t} &:= \prod_{j=1}^J \lambda_{jt} \prod_{\ell=1}^L \lambda_{\ell t}
\end{align}
and correspondingly,
\begin{align}
    \tilde{r}_{-0t} &:= \tilde{r}_t + \mu_0 \\
    \overline{\lambda}_{-0t} &:= \lambda_0^{-1}\overline{\lambda}_t
\end{align}
then (\ref{eq:EB-max}) can be rewritten as:
\small
\begin{align}\label{eq:EB-max-simple}
    \argmax{\{\mu_0,\lambda_0\}\in\mathbb{R}\times\mathbb{R}_{++}} \; \frac{T+B_r +B_l}{2}\log\lambda_0 - \frac{\lambda_0}{2}\sum_{t=1-B_l}^{T+B_r} \overline{\lambda}_{-0t}\left[\left(\tilde{r}_{-0t}-\mu_0\right)^2 + \delta_t\right].
\end{align}
\normalsize
The solutions to (\ref{eq:EB-max-simple}) is given by:
\begin{align}
    \hat{\mu}_0 &= \frac{\sum_{t=1-B_l}^{T+B_r} \overline{\lambda}_{-0t}\tilde{r}_{-0t}}{\sum_{t=1-B_l}^{T+B_r} \overline{\lambda}_{-0t}} \label{eq:EB-max-solution-mu} \\
    \hat{\lambda}_0 &= \left[\frac{\sum_{t=1-B_l}^{T+B_r} \overline{\lambda}_{-0t}[\left(\tilde{r}_{-0t}-\hat{\mu}_0\right)^2 + \delta_t]}{T+B_l+B_r}\right]^{-1}. \label{eq:EB-max-solution-lambda}
\end{align}
We incorporate the estimation step for $\mu_0$ and $\lambda_0$ into Algorithm \ref{alg:2}. Note that Algorithm \ref{alg:2} embeds Algorithm \ref{alg:1} with the additional step of maximizing the ELBO with respect to $\mu_0$ and $\lambda_0$ while holding the approximate posterior $q$ fixed. Therefore, Algorithm \ref{alg:2} also defines a coordinate ascent procedure. The objective function in (\ref{eq:EB-max-simple}) is continuously differentiable in $\mu_0$ and $\lambda_0$, so if we can show that $\hat{\mu}_0$ and $\hat{\lambda}_0$ are the unique maximizers of (\ref{eq:EB-max-simple}), then the conditions of Proposition 2.7.1 of \cite{Bertsekas97} will still hold even with the added coordinate ascent step of maximizing the ELBO with respect to $\{\mu_0,\lambda_0\}$. 

To prove the uniqueness of $\hat{\mu}_0$ and $\hat{\lambda}_0$, first note that the term:
\begin{align*}
    -\frac{1}{2}\sum_{t=1-B_l}^{T+B_r} \overline{\lambda}_{-0t}\left(\tilde{r}_{-0t}-\mu_0\right)^2
\end{align*}
is strictly concave as a function of $\mu_0$ and is maximized at $\hat{\mu}_0$. Therefore, for any $\mu_0\in\mathbb{R}\setminus\{\hat{\mu}_0\}$ we have:
\begin{align*}
    -\frac{1}{2}\sum_{t=1-B_l}^{T+B_r} \overline{\lambda}_{-0t}\left(\tilde{r}_{-0t}-\hat{\mu}_0\right)^2 > -\frac{1}{2}\sum_{t=1-B_l}^{T+B_r} \overline{\lambda}_{-0t}\left(\tilde{r}_{-0t}-\mu_0\right)^2
\end{align*}
and thus for any $\lambda_0>0$,
\begin{align*}
    \frac{T+B_r +B_l}{2}\log\lambda_0-\frac{\lambda_0}{2}\sum_{t=1-B_l}^{T+B_r} \overline{\lambda}_{-0t}\left(\tilde{r}_{-0t}-\hat{\mu}_0\right)^2 > \frac{T+B_r +B_l}{2}\log\lambda_0-\frac{\lambda_0}{2}\sum_{t=1-B_l}^{T+B_r} \overline{\lambda}_{-0t}\left(\tilde{r}_{-0t}-\mu_0\right)^2.
\end{align*}
Since the $\log$ is strictly concave, the left-hand side of the above inequality is uniquely maximized by $\hat{\lambda}_0$. So for any $\lambda_0 > 0$, we have 
\begin{align*}
    \frac{T+B_r +B_l}{2}\log\hat{\lambda}_0-\frac{\hat{\lambda}_0}{2}\sum_{t=1-B_l}^{T+B_r} \overline{\lambda}_{-0t}\left(\tilde{r}_{-0t}-\hat{\mu}_0\right)^2 &> \frac{T+B_r +B_l}{2}\log\lambda_0-\frac{\lambda_0}{2}\sum_{t=1-B_l}^{T+B_r} \overline{\lambda}_{-0t}\left(\tilde{r}_{-0t}-\hat{\mu}_0\right)^2 \\
    &> \frac{T+B_r +B_l}{2}\log\lambda_0-\frac{\lambda_0}{2}\sum_{t=1-B_l}^{T+B_r} \overline{\lambda}_{-0t}\left(\tilde{r}_{-0t}-\mu_0\right)^2.
\end{align*}
Proving that $\hat{\mu}_0$ and $\hat{\lambda}_0$ are the global maximizers of (\ref{eq:EB-max-simple}). Therefore, Proposition \ref{prop:2} still holds with the included maximization step for $\mu_0$ and $\lambda_0$ and Algorithm \ref{alg:2} will also converge to a stationary point. 

\begin{algorithm}
\label{alg:2}
\small
\SetAlgoLined
  Inputs: $J;\:K;\:L;\:B_l;\:B_r;\:\{\tau_j,u_j,v_j,\pi_j\}_{j=1}^J;\:\{\tau_\ell,\pi_\ell\}_{\ell=1}^L;\:\{u_k,v_k,\pi_k\}_{k=1}^K$; \\
  Initialize: $\tilde{\mathbf{r}} := \mathbf{y}$; $\overline{\pmb{\lambda}}:= \mathbf{1}$; $\pmb{\delta}:=\mathbf{0}$; $\mu_0$; $\lambda_0$; $\{\{\overline{b}_{jt}, \overline{\tau}_{jt}, \overline{u}_{jt}, \overline{v}_{jt}, \overline{\pi}_{jt}\}_{t=1}^T\}_{j=1}^J$; $\{\{\overline{b}_{\ell t}, \overline{\tau}_{\ell t}, \{\overline{\pi}_{\ell t}\}_{t=1}^T\}_{\ell=1}^L$; $\{\{\overline{u}_{kt}, \overline{v}_{kt}, \overline{\pi}_{kt}\}_{t=1}^T\}_{k=1}^K$;\\
  
  \Repeat {Convergence} {
    $j:=1;\;\ell:=1;\;k:=1;$ \\
    \While{$j \leq J$} {
      $\tilde{r}_t := \tilde{r}_t + \E_q[\lambda_{jt}]^{-1}\E_q[\lambda_{jt} \mu_{jt}],\; \sforall t$; \texttt{// calculate partial residuals} \\
      $\overline{\lambda}_t := \E_q[\lambda_{jt}]^{-1}\overline{\lambda}_t, \; \sforall t$; \\
      $\delta_t := \delta_t - \E_q[\lambda_{jt}]^{-1}\E_q[\lambda_{jt} \mu^2_{jt}] + (\E_q[\lambda_{jt}]^{-1}\E_q[\lambda_{jt} \mu_{jt}])^2,\; \sforall t;$ \\
      Calculate: $\tilde{v}_{jt}$ and $\tilde{\pi}_{jt}$ by (\ref{eq:mod-v_j}) and (\ref{eq:mod-pi_j}), $\sforall t$; \texttt{// calculate corrected priors} \\
      $\{\overline{b}_{jt}, \overline{\tau}_{jt}, \overline{u}_{jt}, \overline{v}_{jt}, \overline{\pi}_{jt}\}_{t=1}^T := \texttt{SMSCP}\left(\tilde{\mathbf{r}} \:;\:\overline{\pmb{\lambda}}, \tau_j, u_j, \{\tilde{v}_{jt}\}_{t=1}^T, \tilde{\pmb{\pi}}_j, B_l,B_r\right);$ \texttt{// fit model}\\
      Calculate: $\E_q[\lambda_{jt}]$, $\E_q[\lambda_{jt} \mu_{jt}]$, and $\E_q[\lambda_{jt} \mu^2_{jt}]$; \texttt{// update residuals}\\
      $\tilde{r}_t := \tilde{r}_t - \E_q[\lambda_{jt}]^{-1}\E_q[\lambda_{jt} \mu_{jt}],\; \sforall t$; \\
      $\overline{\lambda}_t := \E_q[\lambda_{jt}]\overline{\lambda}_t, \; \sforall t$; \\
      $\delta_t := \delta_t + \E_q[\lambda_{jt}]^{-1}\E_q[\lambda_{jt} \mu^2_{jt}] - (\E_q[\lambda_{jt}]^{-1}\E_q[\lambda_{jt} \mu_{jt}])^2,\; \sforall t;$ \\
      $j := j + 1$; \\
    }
    \While{$\ell \leq L$} {
      $\tilde{r}_t := \tilde{r}_t + \E_q[\mu_{\ell t}],\; \sforall t$; \texttt{// calculate partial residuals} \\
      $\delta_t := \delta_t -  \Var_q(\mu_{\ell t}),\; \sforall t;$ \\
      $\{\overline{b}_{\ell t}, \overline{\tau}_{\ell t}, \overline{\pi}_{\ell t}\}_{t=1}^T := \texttt{SMCP}\left(\tilde{\mathbf{r}} \:;\: \overline{\pmb{\lambda}}, \tau_{\ell}, \pmb{\pi}_{\ell}, B_l,B_r\right);$ \texttt{// fit model} \\
      Calculate: $\E_q[\mu_{\ell t}]$ and $\Var_q(\mu_{\ell t})$; \texttt{// update residuals} \\
      $\tilde{r}_t := \tilde{r}_t - \E_q[\mu_{\ell t}],\; \sforall t$; \\
      $\delta_t := \delta_t + \Var_q(\mu_{\ell t}),\; \sforall t;$ \\
      $\ell := \ell + 1$; \\
    }
    \While{$k \leq K$} {
      $\overline{\lambda}_t := \E_q[\lambda_{k t}]^{-1}\overline{\lambda}_t, \; \sforall t$; \texttt{// calculate partial residuals}\\
      Calculate: $\tilde{v}_{kt}$ and $\tilde{\pi}_{kt}$ by (\ref{eq:mod-v_k}) and (\ref{eq:mod-pi_k}), $\sforall t$; \texttt{// calculate corrected priors} \\
      $\{\overline{u}_{kt}, \overline{v}_{kt}, \overline{\pi}_{kt}\}_{t=1}^T := \texttt{SSCP}\left(\tilde{\mathbf{r}} 
      \:;\:\overline{\pmb{\lambda}}, u_k, \{\tilde{v}_{kt}\}_{t=1}^T, \tilde{\pmb{\pi}}_k, B_l,B_r\right)$; \texttt{// fit model} \\
      Calculate: $\E_q[\lambda_{k t}]$; \texttt{// update residuals} \\
      $\overline{\lambda}_t := \E_q[\lambda_{kt}]\overline{\lambda}_t, \; \sforall t$; \\
      $k := k + 1$; \\
    }
    $\tilde{r}_t := \tilde{r}_t + \mu_0\; \sforall t$; \\
    $\overline{\lambda}_t := \lambda_0{-1}\overline{\lambda}_t, \; \sforall t$; \\
    Calculate: $\mu_0$ and $\lambda_0$ by (\ref{eq:EB-max-solution-mu}) and (\ref{eq:EB-max-solution-lambda});\\ 
    $\tilde{r}_t := \tilde{r}_t - \mu_0\; \sforall t$; \\
    $\overline{\lambda}_t := \lambda_0\overline{\lambda}_t, \; \sforall t$; \\
  }
  \Return{Posterior Parameters: $\mu_0;\lambda_0;\{\{\overline{b}_{jt}, \overline{\tau}_{jt}, \overline{u}_{jt}, \overline{v}_{jt}, \overline{\pi}_{jt}\}_{j=1}^J, \{\overline{b}_{\ell t}, \overline{\tau}_{\ell t}, \overline{\pi}_{\ell t}\}_{\ell=1}^L, \{\overline{u}_{kt}, \overline{v}_{kt}, \overline{\pi}_{kt}\}_{k=1}^K\}_{t=1}^T$}.
  \caption{Variational Bayes Approximation to MICH Posterior with $\mu_0$ and $\lambda_0$ Unknown.}
\end{algorithm}