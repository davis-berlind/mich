\subsection{Merging Duplicate Components}
\label{app:merge-procedure}

As discussed in Section \ref{sec:merge-procedure}, Algorithm \ref{alg:1} occasionally reaches a stationary point where a single change-point is split across multiple $\tau_i$'s. We propose a modification that helps Algorithm \ref{alg:1} move out of theses stationary points. If $\tau_i$ and $\tau_{i'}$ correspond to change-points of the same class, then by (\ref{eq:mean-field}) we have:
\begin{align}
    q(\tau_i = \tau_{i'}) = \sum_{t=1}^T q(\tau_{i'} = t | \tau_i = t ) q(\tau_i = t) = \sum_{t=1}^T q_{i'}(\tau_{i'} = t)q_i(\tau_i = t) = \langle\overline{\boldsymbol{\pi}}_{i'}, \overline{\boldsymbol{\pi}}_i \rangle.
\end{align}
We therefore propose merging components $i$ and $i'$ if $\langle\overline{\boldsymbol{\pi}}_{i'}, \overline{\boldsymbol{\pi}}_i \rangle$ exceeds some threshold $\beta > 0$. In the event that $\min\{\langle\overline{\boldsymbol{\pi}}_{i'}, \overline{\boldsymbol{\pi}}_i \rangle,\langle\overline{\boldsymbol{\pi}}_{i'}, \overline{\boldsymbol{\pi}}_{i''} \rangle\} \geq \beta$, but $\langle\overline{\boldsymbol{\pi}}_{i}, \overline{\boldsymbol{\pi}}_{i''} \rangle \leq \beta$, we enforce transitivity and merge components $i$, $i'$, and $i''$. When components $i$ and $i'$ both capture true change-points, then $\overline{\boldsymbol{\pi}}_{i}$ and $\overline{\boldsymbol{\pi}}_{i'}$ tend to be sparse. Thus, when components $i$ and $i'$ correspond to distinct change-points, $\langle\overline{\boldsymbol{\pi}}_{i}, \overline{\boldsymbol{\pi}}_{i'} \rangle$ tends to be vanishingly small. When components $i$ and $i'$ capture the same change-point, $\langle\overline{\boldsymbol{\pi}}_{i}, \overline{\boldsymbol{\pi}}_{i'}\rangle$ will clearly be nonzero in most cases, so any small value of $\beta$ will correctly merge the true duplicates. After merging the duplicate components, we then restart Algorithm \ref{alg:1}. To account for the fact that Algorithm \ref{alg:1} may just split the merged components again, we iteratively increase $\beta$ until no more merges are proposed, which prevents the procedure from entering an infinite loop. 

To select a default for $\beta$, we consider the worst case scenario where $t^*_i$ and $t^*_{i+1}$ are consecutive change-points separated by the minimum spacing condition from Assumption \ref{assumption:1}, i.e. $|t^*_i - t^*_{i+1}| = \log^{1+\varepsilon} T$ for some $\varepsilon > 0$. Suppose components $i$ and $i+1$ of MICH identify $t^*_i$ and $t^*_{i+1}$ respectively. Given a localization rate $\epsilon_T = \mathcal{O}(\log T)$, for large enough $T$ we have $\mathbb{B}_{\epsilon_T}^{1}(t^*_i) \cap \mathbb{B}_{\epsilon_T}^{1}(t^*_{i+1}) = \emptyset$. In the proof of Corollary \ref{cor:cred-sets}, we show that $\overline{\pi}_{it} \leq T^{-2}$ with high probability for $t \not\in \mathbb{B}_{\epsilon_T}^{1}(t^*_i)$. Though this result is for the single change-point setting, in empirical results for the multiple change-point setting, we observe a similar rate of decay for the posterior probabilities that fall outside of the window defined by the localization rate. Thus, with high probability:
\begin{align}
    \sum_{t \in \mathbb{B}_{\epsilon_T}^{1}(t^*_{i+1})} \overline{\pi}_{it} \overline{\pi}_{(i+1)t} \leq \frac{2\epsilon_T}{T^2}
\end{align}
A symmetric argument gives an identical bound for $t \in \mathbb{B}_{\epsilon_T}^{1}(t^*_{i})$, and for the remaining indices we have $\overline{\pi}_{it} \overline{\pi}_{(i+1)t} \leq T^{-4}$; therefore, the merge probability in this case is at most $\mathcal{O}(T^{-2}\log T)$. If we set $\beta = \frac{\log^{2+\delta} T}{T^2}$, where $\delta > 0$ is the same value used in the detection rule (\ref{eq:LKJ-estimator}), then we should avoid merging any truly distinct components with high probability as $T \to \infty$.

The merge procedure we have proposed may still run into issues when the model includes redundant components. As previously noted, when component $i$ does not capture a change-point, then $\overline{\boldsymbol{\pi}}_{i}$ tends to be diffuse with $\overline{\pi}_{it} \approx T^{-1}$. If component $i'$ does correspond to a change and $\overline{\boldsymbol{\pi}}_{i'}$ is sparse, then we may end up with a situation where $\langle\overline{\boldsymbol{\pi}}_{i}, \overline{\boldsymbol{\pi}}_{i'}\rangle \approx T^{-1}$. Under the default value of $\beta$ we have proposed, components $i$ and $i'$ will be erroneously merged. To avoid this outcome, we only merge components that we believe detect true change-points, i.e. we restrict merge candidates to only include the model components with $\alpha$-level credible sets that satisfy the detection rule (\ref{eq:LKJ-estimator}) with $\alpha = 0.9$ by default. The procedure tends to be insensitive to the choice of $\alpha$, so long as it is moderately large. For example, when $T = 100$, we have $\log^2 T \approx 25$, so for a diffuse $\overline{\boldsymbol{\pi}}_i$, we would rule component $i$ out as a merge candidate for any $\alpha \geq 0.25$.