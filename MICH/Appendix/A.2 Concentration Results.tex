\subsection{Concentration Results and Event Bounds}
\label{app:concentration}

% Sum of sub-Gaussians/Exponentials Lemma
\begin{lemma}\label{lemma:sum-sub-gaussian}
If $\{X_i\}_{i=1}^n$ is a collection of random variables such that for each $i \in [n]$, $\E[X_i] = 0$ and $X_i \in \mathcal{SG}(\sigma_i)$ for some $\sigma_i > 0$, then letting $\boldsymbol{\sigma} = \{\sigma_i\}_{i=1}^n$, we have:
\begin{align*}
    \sum_{i=1}^N X_i \in 
    \begin{cases}
        \mathcal{SG}(\lVert \boldsymbol{\sigma}\rVert_2), &\text{if $\{X_i\}_{i=1}^n$ are mutually independent,} \\
        \mathcal{SG}(\lVert \boldsymbol{\sigma}\rVert_1), &\text{otherwise.}
    \end{cases}
\end{align*}
Similarly, if $X_i \in \mathcal{SE}(\nu_i, \alpha_i)$ for some $\sigma_i,\:\alpha_i > 0$, then letting $\boldsymbol{\nu} = \{\nu_i\}_{i=1}^n$ and $\boldsymbol{\alpha} = \{\alpha_i\}_{i=1}^n$ , we have:
\begin{align*}
    \sum_{i=1}^N X_i \in 
    \begin{cases}
        \mathcal{SE}(\lVert \boldsymbol{\nu}\rVert_2, \lVert \boldsymbol{\alpha}\rVert_\infty), &\text{if $\{X_i\}_{i=1}^n$ are mutually independent,} \\
        \mathcal{SE}(\lVert \boldsymbol{\nu}\rVert_1,  \lVert \boldsymbol{\alpha}\rVert_\infty), &\text{otherwise.}
    \end{cases}
\end{align*}
\end{lemma}

\begin{proof}
First suppose that $X_i \in \mathcal{SE}(\nu_i, \alpha_i)$ and that $\{X_i\}_{i=1}^n$ is a collection of independent random variables, then by independence:
\begin{align*}
    \E\left[\exp\left(\lambda \sum_{i=1}^n X_i\right)\right] &=  \prod_{i=1}^n \E\left[\exp\left(\lambda X_i\right)\right].
\end{align*}
Next, note that if $|\lambda| < \lVert \boldsymbol{\alpha}\rVert^{-1}_{\infty}$, then $|\lambda| < \alpha_i^{-1}$ for each $i \in [n]$ and thus:
\begin{align*}
    \E\left[\exp\left(\lambda X_i\right)\right] \leq \exp\left[\frac{\lambda^2\nu_{i}^2}{2}\right]
\end{align*}
which implies:
\begin{align*}
    \E\left[\exp\left(\lambda \sum_{i=1}^n X_i\right)\right] &\leq \exp\left[\frac{\lambda\sum_{i=1}^n\nu^2_i}{2}\right].
\end{align*}
So $\sum_{i=1}^n X_i \in \mathcal{SE}(\lVert \boldsymbol{\nu}\rVert_2, \lVert \boldsymbol{\alpha}\rVert_\infty)$. When $\{X_i\}_{i=1}^n$ are not independent, then for any $|\lambda| \leq \lVert \boldsymbol{\alpha}\rVert^{-1}_{\infty}$ and any collection of parameters $\{p_i\}_{i=1}^n$ such that $p_i > 0$ and $\sum_{i=1}^n p_i^{-1} = 1$, we have:
\begin{align*}
    \E\left[\exp\left(\lambda \sum_{i=1}^n X_i\right)\right] &= \E\left[ \prod_{i=1}^n \exp\left(\lambda X_i\right)\right] \\
    &\leq \prod_{i=1}^n \left(\E[\exp\left(p_i \lambda X_i\right)]\right)^{1/p_i} \tag{H\"older's inequality} \\
    &\leq \prod_{i=1}^n \left(\E\left[\exp\left(\frac{p^2_i\nu_i^2 \lambda^2}{2}\right)\right]\right)^{1/p_i} \tag{$X_i \in \mathcal{SE}(\nu_i,\alpha_i)$} \\
    &= \exp\left[\frac{\lambda^2\sum_{i=1}^n p_i \nu_i^2}{2}\right].
\end{align*}
Since our choice of $\{p_i\}_{i=1}^n$ was arbitrary so long as $p_i > 0$ and $\sum_{i=1}^n p_i^{-1} = 1$, we can set each $p_i = \frac{\lVert\boldsymbol{\nu}\rVert_1}{\nu_i},$ then we get:
\begin{align*}
    \E\left[\exp\left(\lambda \sum_{i=1}^n X_i\right)\right] &\leq \exp\left[\frac{\lambda \lVert \boldsymbol{\nu}\rVert^2_1}{2}\right]
\end{align*}
showing that $\sum_{i=1}^n X_i \in \mathcal{SE}(\lVert\boldsymbol{\nu}\rVert_1, \lVert \boldsymbol{\alpha}\rVert_\infty).$ The proof for the case where $X_i \in \mathcal{SG}(\sigma_i)$ follows an identical argument as above with $\sigma_i$ replacing $\nu_i$ and ignoring any constraints on $\lambda$. 
\end{proof}

% Theorem 1 Events 

\subsubsection{Theorem \ref{theorem:smcp} Event Bounds}

In Lemma \ref{lemma:thm1-event-bound}, we assume $\lVert \mathbf{y}_t\rVert_{\psi_2}$ is uniformly bounded over all $t, d \geq 1$. As per Remark \ref{rmk:sub-g}, this assumption holds when either: i) each entry of $\mathbf{y}_t$ is independent and $\mathcal{SG}(\sigma)$ for some $\sigma < \infty$, ii) $\mathbf{y}_t \sim \mathcal{N}_d(\boldsymbol{\mu}_t, \boldsymbol{\Lambda}^{-1})$ and $\inf_{d\geq 1}  \lambda_{\min} > 0$, where $\lambda_{\min}$ is the smallest eigen-value of $\Lambda$, or iii) $y_{t,j} \in \mathcal{SG}(\sigma_j)$ for each $j \in \{1,\ldots, d\}$ with $\sup_{d \geq 1} \lVert\boldsymbol{\sigma}\rVert_2 < \infty$, $\lVert\boldsymbol{\sigma}\rVert_1 = \mathcal{O}\left(\frac{1}{\sqrt{d}}\right)$, or $\lVert\boldsymbol{\sigma}\rVert_\infty = \mathcal{O}\left(\frac{1}{\sqrt{d}}\right)$. 

\begin{enumerate}[label=\roman*)]
    \item When each entry of $\mathbf{y}_t$ is independent and $y_{t,j} \in \mathcal{SG}(\sigma)$ for each $j \in \{1, \ldots, d\}$, then for any $\mathbf{v} \in \mathbb{B}^d_2(1)$, by Lemma \ref{lemma:sum-sub-gaussian} we have $\langle \mathbf{v}, \mathbf{y}_t \rangle \in \mathcal{SG}(\sigma)$.
    \item For any $\mathbf{v} \in \mathbb{B}^d_2(1)$, we have:
    \begin{align*}
        \langle \mathbf{v}, \mathbf{y}_t \rangle &= \langle \boldsymbol{\Lambda}^{-\frac{1}{2}}\mathbf{v}, \boldsymbol{\Lambda}^{\frac{1}{2}}\mathbf{y}_t \rangle \\
        &= \lVert\boldsymbol{\Lambda}^{-\frac{1}{2}}\mathbf{v}\rVert_{2} \left\langle \frac{\boldsymbol{\Lambda}^{-\frac{1}{2}}\mathbf{v}}{\lVert\boldsymbol{\Lambda}^{-\frac{1}{2}}\mathbf{v}\rVert_{2}}, \boldsymbol{\Lambda}^{\frac{1}{2}}\mathbf{y}_t \right\rangle.
    \end{align*}
    Since $\mathbf{y}_t$ is Gaussian, the coordinates of $\boldsymbol{\Lambda}^{\frac{1}{2}}\mathbf{y}_t$ are independent and $\mathcal{SG}(1)$, so by Lemma \ref{lemma:sum-sub-gaussian}, the inner-prodcut above is $\mathcal{SG}(1)$. Also, $\inf_{d\geq 1}  \lambda_{\min} > 0$ by assumption so:
    \begin{align*}
        \lVert\boldsymbol{\Lambda}^{-\frac{1}{2}}\mathbf{v}\rVert_{2} \leq \sup_{\mathbf{v} \in \mathbb{B}^d_2(1)} \lVert\boldsymbol{\Lambda}^{-\frac{1}{2}}\mathbf{v}\rVert_{2} = \frac{1}{\sqrt{\lambda_{\min}}} < \infty.
    \end{align*}
    \item If $y_{t,j} \in \mathcal{SG}(\sigma_j)$, then for any $\mathbf{v} \in \mathbb{B}^d_2(1)$: 
    \begin{align*}
        \langle \mathbf{v}, \mathbf{y}_t \rangle &\in \mathcal{SG}(\langle \mathbf{v}, \boldsymbol{\sigma} \rangle) \tag{Lemma \ref{lemma:sum-sub-gaussian}}\\
        &\subset \mathcal{SG}(\lVert\mathbf{v}\rVert_2 \lVert\boldsymbol{\sigma}\rVert_2). \tag{Cauchy-Schwarz inequality} \\
        &\subset \mathcal{SG}(\sqrt{d} \lVert \boldsymbol{\sigma}\rVert_1). \tag{$\lVert \mathbf{v}\rVert_2 = 1$ and $\lVert \boldsymbol{\sigma}\rVert_2 \leq \sqrt{d} \lVert \boldsymbol{\sigma}\rVert_1 $}
    \end{align*}    
    In the last line we can replace $\lVert \boldsymbol{\sigma}\rVert_1$ with $\lVert \boldsymbol{\sigma}\rVert_\infty$.
\end{enumerate}


\begin{lemma}\label{lemma:thm1-event-bound}

Let $\{\mathbf{y}_t\}_{t=1}^T$ be a sequence of independent random vectors with $\mathbf{y}_t \in \mathbb{R}^d$, $\normalfont{\Var}(\mathbf{y}_t) = \boldsymbol{\Lambda}^{-1}$ and $\sup_{t\geq 1, d\geq 1} \lVert \mathbf{y}_t\rVert_{\psi_2} < \infty$. In addition to Assumptions \ref{assumption:1} (i), (ii), (iii) (a) and Assumption \ref{assumption:mean}, let $\lambda_{\max}$ and $\lambda_{\min}$ be the largest and smallest eigenvalues of $\boldsymbol{\Lambda}$ respectively and assume that $\sup_{d\geq 1} \lambda_{\min} < \infty$ and $\inf_{d\geq 1} \lambda_{\min} > 0$. If we define the normalized terms $\mathbf{z}_t := \boldsymbol{\Lambda}^{\frac{1}{2}}(\mathbf{y}_t - \mathbf{b}_0\mathbbm{1}\{t\geq t_0\})$, then there exist positive constants $\kappa$ and $C_1$ such that we can define the events:
\begin{align*}
    \Omega_1 &:= \bigcap_{t \;:\; |t_0 - t| > \frac{\kappa \log T}{\lVert \boldsymbol{\Lambda}^{1/2}\mathbf{b}_0\rVert_2^2}} \left\{ 
    \left|\left\langle\boldsymbol{\Lambda}^{\frac{1}{2}}\mathbf{b}_0, \sum_{t'=\min\{t_0,t\}}^{\max\{t_0,t\}-1}\mathbf{z}_{t'} \right\rangle\right| < \frac{|t_0-t| \lVert\boldsymbol{\Lambda}^{\frac{1}{2}} \mathbf{b}_0\rVert_2^2}{8}\right\} \\
    \Omega_2 &:= \left\{\left|\left\langle \boldsymbol{\Lambda}^{\frac{1}{2}}\mathbf{b}_0, \sum_{t'=t_0}^T\mathbf{z}_{t'}\right\rangle\right| < \frac{(T-t_0+1) \lVert\boldsymbol{\Lambda}^{\frac{1}{2}} \mathbf{b}_0\rVert_2^2}{8} \right\} \\
    \Omega_3 &:=  \bigcap_{t \;:\; |t_0 - t| > \frac{\kappa \log T}{\lVert \boldsymbol{\Lambda}^{1/2}\mathbf{b}_0\rVert_2^2}} \left\{\left|\left\langle \sum_{t'=\min\{t_0,t\}}^{\max\{t_0,t\}-1}\frac{\mathbf{z}_{t'}}{\sqrt{|t_0-t|}}, \frac{\sum_{t'=\max\{t_0,t\}}^{T}\mathbf{z}_{t'}}{\lVert\sum_{t'=\max\{t_0,t\}}^{T}\mathbf{z}_{t'}\rVert_2}\right\rangle\right| < 2\sqrt{C_1\log T}\right\} \\
    \Omega_4 &:= \bigcap_{t \geq t_0} \left\{\left|\frac{1}{T-t+1}\left\lVert\sum_{t'=t}^T \mathbf{z}_{t'}\right\rVert_2^2 -d \right| < 4 C_1 d \log T \right\} \\
    \Omega_5 &:= \bigcap_{t\neq t_0} \left\{\left|\frac{1}{|t_0 - t|}\left\lVert\sum_{t'=\min\{t_0,t\}}^{\max\{t_0,t\}-1} \mathbf{z}_{t'}\right\rVert_2^2 -d \right| < 4 C_1 d \log T \right\} \\
    \Omega_6 &:= \bigcap_{t\geq t_0} \left\{\left|\frac{1}{\sqrt{T-t+1}}\left\lVert\sum_{t'=t}^T \mathbf{z}_{t'}\right\rVert_2 -\sqrt{d} \right| < 2\sqrt{C_1d \log T} \right\} 
\end{align*}
and we will have $\lim_{T\to\infty} \Pr(\Omega) = 1$, where $\Omega := \cap_{i=1}^6 \Omega_i$.

\end{lemma}

\begin{proof}

By the union bound $\Pr(\Omega^c) = 1 - P(\cup_{i=1}^6 \Omega^c_i) < 1 - 1 - \sum_{i=1}^6 \Pr(\Omega^c_i)$, so it is sufficient to show that $\Pr(\Omega^c_i) \to 0$ for each $i$. By assumption, there is exists some $K < \infty$ such that $\sup_{t\geq 1, d\geq 1} \lVert \mathbf{y}_t\rVert_{\psi_2} \leq K$, so for any $\mathbf{v} \in \mathbb{B}^d_2(1)$ we have:
\begin{align*}
    \left\lVert\left\langle\mathbf{v}, \mathbf{z}_t \right\rangle\right\rVert_{\psi_2} &= \lVert \boldsymbol{\Lambda}^{\frac{1}{2}}\mathbf{v}\rVert_{2}\left\lVert\left\langle\frac{\boldsymbol{\Lambda}^{\frac{1}{2}}\mathbf{v}}{\lVert \boldsymbol{\Lambda}^{\frac{1}{2}}\mathbf{v} \rVert_2}, \mathbf{y}_t - \E[\mathbf{y}_t] \right\rangle\right\rVert_{\psi_2} \\
    &\leq \sup_{\mathbf{v} \in  \mathbb{B}^d_2(1)}\lVert\boldsymbol{\Lambda}^{\frac{1}{2}}\mathbf{v}\rVert_{2} \sup_{\mathbf{v} \in  \mathbb{B}^d_2(1)} \left\lVert\left\langle\mathbf{v}, \mathbf{y}_t - \E[\mathbf{y}_t] \right\rangle\right\rVert_{\psi_2}\\
    &= \sqrt{\lambda_{\max}} \left\lVert\mathbf{y}_t - \E[\mathbf{y}_t] \right\rVert_{\psi_2} \\
    &\lesssim \sqrt{\lambda_{\max}} \left\lVert\mathbf{y}_t \right\rVert_{\psi_2}. \tag{Lemma 2.6.8 of \cite{Vershynin18}} \\
    &\lesssim \sqrt{\lambda_{\max}} K
\end{align*}
Since $\lambda_{\max}$ is finite by assumption, and $\{\mathbf{z}_t\}_{t=1}^T$ is a independent sequence, then by Propositions 2.5.2 and 2.6.1 of \cite{Vershynin18} there exists some constant $C$ such that for any index set $\mathcal{T} \subset [T]$ with cardinality $|\mathcal{T}|$ we have: 
\begin{align*}
    \sum_{t \in \mathcal{T}} \left\langle\mathbf{v}, \mathbf{z}_t \right\rangle \in \mathcal{SG}\left(C\sqrt{\lambda_{\max}|\mathcal{T}|} K\right).
\end{align*}
So letting $\mathbf{v} = \frac{\boldsymbol{\Lambda}^{\frac{1}{2}}\mathbf{b}_0}{\lVert \boldsymbol{\Lambda}^{\frac{1}{2}}\mathbf{b}_0 \rVert_2}$, then by the Chernoff bound (\ref{eq:chernoff}): 
\begin{align*}
    \Pr\left(\left|\left\langle\boldsymbol{\Lambda}^{\frac{1}{2}}\mathbf{b}_0, \sum_{t'=\min\{t_0,t\}}^{\max\{t_0,t\}-1}\mathbf{z}_{t'} \right\rangle\right| \geq \frac{|t_0-t| \lVert\boldsymbol{\Lambda}^{\frac{1}{2}} \mathbf{b}_0\rVert_2^2}{8}\right) &\leq 2\exp\left[-\frac{|t_0-t|\lVert\boldsymbol{\Lambda}^{\frac{1}{2}} \mathbf{b}_0\rVert^2_2}{128\lambda_{\max}C^2K^2}\right].
\end{align*}
For all $t$ such that $|t_0 - t| > \frac{\kappa \log T}{\lVert \boldsymbol{\Lambda}^{1/2}\mathbf{b}_0\rVert_2^2}$ and $\kappa \geq 256\lambda_{\max}C^2K^2$, we then have: 
\begin{align*}
    \Pr\left(\left|\left\langle\boldsymbol{\Lambda}^{\frac{1}{2}}\mathbf{b}_0, \sum_{t'=\min\{t_0,t\}}^{\max\{t_0,t\}-1}\mathbf{z}_{t'} \right\rangle\right| \geq \frac{|t_0-t| \lVert\boldsymbol{\Lambda}^{\frac{1}{2}} \mathbf{b}_0\rVert_2^2}{8}\right) &\leq \frac{2}{T^2}.
\end{align*}
Therefore:
\begin{align*}
    \Pr(\Omega^c_1) &= \Pr\left(\bigcup_{t \;:\; |t_0 - t| > \frac{\kappa \log T}{\lVert \boldsymbol{\Lambda}^{1/2}\mathbf{b}_0\rVert_2^2}} \left\{\left|\left\langle\boldsymbol{\Lambda}^{\frac{1}{2}}\mathbf{b}_0, \sum_{t'=\min\{t_0,t\}}^{\max\{t_0,t\}-1}\mathbf{z}_{t'} \right\rangle\right| \geq \frac{|t_0-t| \lVert\boldsymbol{\Lambda}^{\frac{1}{2}} \mathbf{b}_0\rVert_2^2}{8}\right\} \right) \\
    &\leq \sum_{t \;:\; |t_0 - t| > \frac{\kappa \log T}{\lVert \boldsymbol{\Lambda}^{1/2}\mathbf{b}_0\rVert_2^2}}\Pr\left(\left|\left\langle\boldsymbol{\Lambda}^{\frac{1}{2}}\mathbf{b}_0, \sum_{t'=\min\{t_0,t\}}^{\max\{t_0,t\}-1}\mathbf{z}_{t'} \right\rangle\right| \geq \frac{|t_0-t| \lVert\boldsymbol{\Lambda}^{\frac{1}{2}} \mathbf{b}_0\rVert_2^2}{8}\right)\tag{union bound} \\
    &\leq \sum_{t \;:\; |t_0 - t| > \frac{\kappa \log T}{\lVert \boldsymbol{\Lambda}^{1/2}\mathbf{b}_0\rVert_2^2}} \frac{2}{T^2} \\
    &\leq \sum_{t=1}^T \frac{2}{T^2} \\
    &= \frac{2}{T}.
\end{align*}
We can again use the Chernoff bound (\ref{eq:chernoff}) to get: 
\begin{align*}
    \Pr\left(\left|\left\langle\boldsymbol{\Lambda}^{\frac{1}{2}}\mathbf{b}_0, \sum_{t'=t_0}^T \mathbf{z}_{t'} \right\rangle\right| \geq \frac{(T-t_0+1) \lVert\boldsymbol{\Lambda}^{\frac{1}{2}} \mathbf{b}_0\rVert_2^2}{8}\right) &\leq 2\exp\left[-\frac{(T-t_0+1)\lVert\boldsymbol{\Lambda}^{\frac{1}{2}} \mathbf{b}_0\rVert^2_2}{128\lambda_{\max}C^2K^2}\right].
\end{align*}
We have $\lVert\boldsymbol{\Lambda}^{\frac{1}{2}} \mathbf{b}_0\rVert^2_2 \geq \lambda_{\min} \lVert \mathbf{b}_0\rVert^2_2$ and by Assumption \ref{assumption:mean}, there is some finite $\underline{b} > 0$ such that $\lVert\mathbf{b}_0\rVert_2^2 \geq \underline{b}^2 d$. At the same time, by Assumption \ref{assumption:1} (i) we have $T-t_0 +1 > \log^{1+\varepsilon} T$, and thus:
\begin{align*}
    \Pr\left(\left|\left\langle\boldsymbol{\Lambda}^{\frac{1}{2}}\mathbf{b}_0, \sum_{t'=t_0}^T \mathbf{z}_{t'} \right\rangle\right| \geq \frac{(T-t_0+1) \lVert\boldsymbol{\Lambda}^{\frac{1}{2}} \mathbf{b}_0\rVert_2^2}{8}\right) &\leq 2\exp\left[-\frac{\lambda_{\min}\underline{b}^2d }{128\lambda_{\max}C^2K^2}\log^{1+\varepsilon} T\right].
\end{align*}
So $\Pr(\Omega_2^c) \to 0.$ Again, since $\frac{\sum_{t'=\max\{t_0,t\}}^{T}\mathbf{z}_{t'}}{\lVert\sum_{t'=\max\{t_0,t\}}^{T}\mathbf{z}_{t'}\rVert_2} \in \mathbb{B}_2^d(1)$ for each $t$, so by the Chernoff bound (\ref{eq:chernoff}) we have:
\begin{align*}
    \Pr\left(\left|\left\langle \sum_{t'=\min\{t_0,t\}}^{\max\{t_0,t\}-1}\frac{\mathbf{z}_{t'}}{\sqrt{|t_0-t|}}, \frac{\sum_{t'=\max\{t_0,t\}}^{T}\mathbf{z}_{t'}}{\lVert\sum_{t'=\max\{t_0,t\}}^{T}\mathbf{z}_{t'}\rVert_2}\right\rangle\right| \geq \sqrt{2C_1\log T}\right) \leq 2\exp\left[-\frac{2C_1\log T\log T}{C^2K^2\lambda_{\max}}\right] 
\end{align*}
So for $C_2 \geq C^2K^2\lambda_{\max}$, we can use the same argument as we did for $\Omega_1$ above to get $\Pr(\Omega_3^c) \leq 2T^{-1}$.


Next, since our observations are independent across $t$, we have $\E[\mathbf{z}'_{t'}\mathbf{z}_{t}] = \E[\mathbf{z}'_{t'}]\E[\mathbf{z}_{t}] = 0$ for $t \geq t'$, and since $\E[z_{t,j}^2] = 1$ by construction, then:
\begin{align*}
    \E\left[ \frac{1}{T-t+1}\left\lVert\sum_{t'=t}^T\mathbf{z}_{t'}\right\rVert_2^2\right] &=  \frac{1}{T-t+1}\sum_{t'=t}^T\E[\mathbf{z}'_{t'}\mathbf{z}_{t'}] = d
\end{align*}
Since:
\begin{align*}
    \left\lVert\frac{1}{\sqrt{|\mathcal{T}|}}\sum_{t'\in\mathcal{T}}\mathbf{z}_{t'}\right\rVert_{\psi_2} \lesssim K\sqrt{\lambda_{\max}}
\end{align*}
then by the Hanson–Wright inequality (see e.g. exercise 6.2.5 of \citealp{Vershynin18}), there are universal constants $K_1,K_2>0$ so that for any $t \geq t_0$:
\begin{align*}
    \frac{1}{T-t+1}\left\lVert\sum_{t'=t}^T\mathbf{z}_{t'}\right\rVert_2^2 &\in \mathcal{SE}(K_1K\sqrt{\lambda_{\max}d},K_2) 
\end{align*}
For $T$ large enough so that $\log T > \frac{K_1^2K^2\lambda_{\max}}{K_2}$, by (\ref{eq:wainwright_prop_2.9}) we have: 
\begin{align*}
    \Pr\left(\left| \frac{1}{T-t+1}\left\lVert\sum_{t'=t}^T\mathbf{z}_{t'}\right\rVert_2^2 - d\right| \geq 4K_2d\log T\right) &\leq \frac{2}{T^{2d}}
\end{align*}
and thus for any $C_1 > K_2$:
\begin{align*}
    \Pr(\Omega^c_3) &= \Pr\left(\bigcup_{t= t_0}^T \left\{\left| \frac{1}{T-t+1}\left\lVert\sum_{t'=t}^T\mathbf{z}_{t'}\right\rVert_2^2 - d\right| \geq 4 C_1 d\log T \right\} \right) \\
    &\leq \sum_{t = t_0}^{T} \Pr\left(\left| \frac{1}{T-t+1}\left\lVert\sum_{t'=t}^T\mathbf{z}_{t'}\right\rVert_2^2 - d\right| \geq  4 K_2 d \log T \right) \tag{union bound and $C_1 > K_2$} \\
    &\leq \sum_{t = t_0}^{T} \frac{2}{T^{2d}} \\
    &\leq \frac{2}{T^{2d - 1}}
\end{align*}
The last line is converging to zero since $d \geq 1$. An identical argument gives $\Pr(\Omega^c_4) \to 0$. Next, for any $x,\delta \geq 0$ we have $|x - 1| \geq \delta \implies |x^2 - 1| \geq \max\{\delta, \delta^2\}$ (see 3.2 in \citealp{Vershynin18}). Using this fact and that $2\sqrt{C_1 \log T} \leq 4C_1 \log T$ for large $T$, we have:
\begin{align*}
    \left\{\left|\frac{1}{\sqrt{d(T-t+1)}}\left\lVert\sum_{t'=t}^T \mathbf{z}_{t'}\right\rVert_2 -1 \right| \geq  2\sqrt{ C_1 \log T}\right\} &\subset \left\{\left|\frac{1}{d(T-t+1)}\left\lVert\sum_{t'=t}^T \mathbf{z}_{t'}\right\rVert^2_2 -1 \right| \geq 4 C_1 \log T \right\} 
\end{align*}
and thus $\Omega_5^c \subset \Omega_3^c \implies \Pr(\Omega_5^c) \leq 2T^{-(2d-1)}$.

\end{proof}
