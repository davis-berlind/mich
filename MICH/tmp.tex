Change-point detection (CPD) has been a perennial topic of interest in statistical inference since the introduction of the CUSUM algorithm by \cite{Page54}. Broadly, CPD involves identifying the locations of structural breaks in an ordered sequence of data $\mathbf{y}_{1:T}:= \{\mathbf{y}_t\}_{t=1}^T$ where $\mathbf{y}_t \in \mathbb{R}^d$. Suppose there are $L$ indices $\boldsymbol{\tau}_{1:L}:=\{\tau_\ell\}_{\ell=1}^{L} \subset \{1,\ldots,T\}$, with $\tau_0:=0 < \tau_1 < \ldots< \tau_L < \tau_{L+1} := T+1$, and a collection of $L+1$ distributions $\mathbf{F} := \{F_\ell\}_{\ell=0}^L$ such that:
\begin{align}\label{eq:cdp-def}
    \mathbf{y}_t \sim F_\ell, \;\sforall t \in [\tau_\ell,\tau_{\ell+1}).
\end{align}
The aim of any CPD method is to consistently estimate the number of changes $L$ and their locations $\boldsymbol{\tau}_{1:L}$. Due to the generality of this construction, CPD problems appear in a variety of scientific fields, including detecting: adjustments to the real interest rate (\citealp{Bai03}), the location of changes in the structure of a genome (\citealp{Muggeo11}), the presence of radiological anomalies \citep{madrid2019sequential}, and the occurrence of deforestation events (\citealp{Wendelberger21}). In addition to estimating the number and locations of the changes underlying $\mathbf{y}_{1:T}$, a limited set of methods can also return localized regions of $[T]$ that contain change-points at a prescribed significance level. Such measures of uncertainty are essential in applications like the design of medical procedures. For example, \cite{Gao19} use a CPD method to determine whether the a human liver is viable for transplant. In this case, knowing whether the estimated change-point is likely to be off by minutes versus hours will materially affect the outcome of the procedure. 

Early attempts to provide confidence sets were either limited to the case of single mean change (\citealp{Siegmund86, Worsley86}), required knowledge of $L$ (\citealp{Bai03}), or could only produce approximate sets based on the limiting distribution of the change-point estimator (\citealp{Bai10}). Since the introduction of SMUCE by \cite{Frick14}, the state-of-the-art has rapidly advanced and there are now methods that return confidence sets under a variety of classes of changes, data structures, and dependence settings (see e.g. \citealp{Pein17, Eichinger18, Fang20, Chen22, Cho22, Fryzlewicz23, Fryzlewicz24}). Still, there remains room for improvement. As noted in \cite{chen2014discussion}, the sets returned by SMUCE exhibit coverage issues as $\alpha$ decreases. Methods like NSP attempt to address the coverage issues of SMUCE, but still return sets that tend to be very large and often overly conservative. Additionally, each new CPD method with confidence sets tends to focus narrowly on one kind of change, and methods for multivariate data and changes in the variance of $\mathbf{y}_{1:T}$ remain underdeveloped. We aim to address these remaining gaps in the literature.


\subsection{Related Work}
Frequentist CPD methods broadly fall into two classes: i) greedy algorithms that recursively partition $\mathbf{y}_{1:T}$, and ii) exact search methods that jointly estimate the locations of the changes by solving an optimization problem, often using dynamic programming. Binary Segmentation (BS; \citealp{Scott74, Sen75, Vostrikova81}) is the classical greedy algorithm for detecting changes in the mean and scale of $\mathbf{y}_{1:T}$. Despite its suboptimal statistical properties, BS remains popular due to its simplicity and $\mathcal{O}(\log T)$ computational complexity. Many variants of BS have been introduced to improve its performance and theoretical guarantees (\citealp{Olshen04, Fryzlewicz14, Kovacs22}). Examples of exact search methods include: the segment neighborhood method (SN; \citealp{Auger89}); the pruned exact linear time method (PELT; \citealp{Killick12}); and the narrowest-over-threshold method (NOT; \citealp{Baranowski19}). There are also specialized methods for detecting changes in the just the variance of $\mathbf{y}_{1:T}$ (\citealp{Inclan94, Chen97, Gao19, Padilla22}). 



As previously noted, the CPD literature is limited when it comes to uncertainty quantification. Neither BS nor any of the exact search methods mentioned above provide confidence sets or any other measure of uncertainty for the point estimates they return. A few early works sought to address this gap, but were limited to the case of a single change in the mean of a univariate sequence. \cite{Siegmund86} used an exact approach to find $\alpha$-level confidence sets for a mean change in finite samples and \cite{Worsley86} inverted a likelihood ratio test to generate confidence sets for a mean change in a exponential family. Later, \citealp{Bai03} generalized to the case of multiple change-points in the mean and variance of a univariate $\mathbf{y}_{1:T}$, but they still require knowledge of the number of changes $L$, and can only generate approximate confidences sets based on the limiting distribution of the change-point estimators. 

Since the introduction of SMUCE by \cite{Frick14}, the literature has seen renewed interest in quantifying uncertainty with certain theoretical guarantees. SMUCE was originally designed for the i.i.d. Gaussian case, but has since been extended for heterogeneous Gaussian noise (\citealp{Pein17}) and dependent data (\citealp{Dette20}). Beyond SMUCE, there are now methods that return confidence sets in the cases of: changes to a piece-wise constant mean (\citealp{Fang20}), changes to a piece-wise constant median (\citealp{Fryzlewicz24median}), changes to piece-wise parameters in regression model with auto-correlation (\citealp{fang2021detectionestimationlocalsignals, Fryzlewicz24}), simultaneous changes in a piece-wise constant mean and variance (\citealp{Eichinger18}), and nonparametric multivariate changes \cite{madrid2023change}. There are also many methods that approach uncertainty quantification from the view of post-selection inference (see e.g. ). These methods provide significance tests for individual change-point estimates, but do not return confidence sets or other measures of uncertainty that are more accessible to the practitioners for whom the method is intended. We therefore do not consider these methods in this article, but refer the reader to \cite{Fryzlewicz24} for a recent survey and discussion of the limitations of post-selection inference for CPD.  