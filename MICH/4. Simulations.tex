\section{Simulations}
\label{sec:simulations}

\textit{In Progress.}

In this section, we reproduce the simulation study from Section 4 of \cite{Pein17}, in which we generate $\mathbf{y}$ by:
\begin{enumerate}
    \item Fix the number of observations $T$, the number of change-points $L$, a constant $C=200$ and a minimum spacing condition.
    \item Draw the locations of the $L$ change-points uniformly from $\{2, \ldots, T-1\}$ subject to the minimum spacing condition. 
    \item Draw standard deviations $s_1, \ldots, s_L$ independently so that $s_\ell = 2^{U_\ell}$ where $U_\ell \sim \text{Uniform}(-2,2)$.
\end{enumerate}


\begin{table} \label{tab:hsmuce-sim}
    \footnotesize
    \centering
    \begin{tabular}{l| r r r r r r r r r r}
        Method & T & L & Min. Space & $|L - \hat{L}|$ & CI Len. & Cond. Cov. & $d(C^*,\hat{C})$ & FPSLE & FNSLE & Time \\ \hline\hline 
        H-SMUCE & 100 & 2 & 15 & 0.33 & 24.08 & 0.98 & 14.30 & 0.92 & 3.27 & 0.01 \\
        MICH(0,L,K) &  &  &  & 0.38 & 2.58 & 0.96 & 0.80 & 1.58 & 0.67 & 0.07 \\
        MICH(0,L,K) Oracle &  &  &  & 0.00 & 1.49 & 0.96 & 0.51 & 0.16 & 0.17 & 0.01 \\
        MICH(J,0,0) &  &  &  & 0.68 & 3.32 & 0.97 & 0.27 & 2.65 & 0.88 & 0.05 \\
        MICH(J,0,0) Oracle &  &  &  & 0.00 & 1.38 & 0.97 & 0.26 & 0.10 & 0.10 & 0.01 \\
        MOSUM BU &  &  &  & 0.13 & 9.26 & 0.97 & 2.84 & 0.99 & 0.99 & 0.01 \\
        MOSUM LP &  &  &  & 0.17 & 5.96 & 0.99 & 1.83 & 0.90 & 0.70 & 0.01 \\
        NOT &  &  &  & 0.16 & &  & 2.35 & 1.27 & 0.76 & 0.02 \\
        PELT &  &  &  & 0.12 & &  & 0.84 & 0.57 & 0.26 & 0.00\\ \hline
        
        H-SMUCE & 100 & 5 & 15 & 1.94 & 17.12 & 0.84 & 23.48 & 3.26 & 6.16 & 0.01 \\
        MICH(0,L,K) &  &  &  & 0.44 & 1.66 & 0.98 & 10.85 & 1.37 & 2.39 & 0.13 \\
        MICH(0,L,K) Oracle &  &  &  & 0.44 & 1.66 & 0.98 & 10.85 & 1.37 & 2.39 & 0.13 \\
        MICH(J,0,0) &  &  &  & 0.21 & 1.51 & 0.99 & 7.96 & 1.14 & 1.77 & 0.43 \\
        MICH(J,0,0) Oracle &  &  &  & 0.21 & 1.51 & 0.99 & 7.96 & 1.14 & 1.77 & 1.25 \\
        MOSUM BU &  &  &  & 0.78 & 24.12 & 1.00 & 13.41 & 1.12 & 2.23 & 0.02 \\
        MOSUM LP &  &  &  & 0.39 & 8.17 & 1.00 & 5.71 & 0.65 & 1.04 & 0.02 \\
        NOT &  &  &  & 0.32 &  &  & 6.45 & 0.58 & 0.76 & 0.02 \\
        PELT &  &  &  & 0.19 &  &  & 0.88 & 0.27 & 0.15 & 0.00\\ \hline

        H-SMUCE & 500 & 2 & 30 & 0.09 & 53.88 & 1.00 & 20.59 & 1.35 & 4.43 & 0.01 \\
        MICH(0,L,K) &  &  &  & 0.40 & 9.40 & 0.86 & 5.87 & 6.82 & 3.54 & 1.11 \\
        MICH(0,L,K) Oracle &  &  &  & 0.01 & 4.64 & 0.87 & 4.30 & 1.18 & 1.70 & 0.40 \\
        MICH(J,0,0) &  &  &  & 0.76 & 13.20 & 0.93 & 2.60 & 12.02 & 4.17 & 2.73 \\
        MICH(J,0,0) Oracle &  &  &  & 0.01 & 3.76 & 0.94 & 2.39 & 1.49 & 1.73 & 0.03 \\
        MOSUM BU &  &  &  & 0.40 & 24.00 & 0.98 & 4.78 & 11.64 & 5.26 & 0.02 \\
        MOSUM LP &  &  &  & 0.39 & 17.34 & 0.99 & 5.94 & 9.22 & 4.62 & 0.02 \\
        NOT &  &  &  & 0.04 &  &  & 4.25 & 1.42 & 1.15 & 0.04 \\
        PELT &  &  &  & 0.03 &  &  & 4.18 & 0.89 & 1.01 & 0.00\\ \hline 

        H-SMUCE & 500 & 5 & 15 & 0.60 & 40.17 & 0.92 & 25.92 & 1.96 & 6.37 & 0.01 \\
        MICH(0,L,K) &  &  &  & 1.02 & 6.33 & 0.91 & 16.89 & 6.67 & 7.03 & 8.07 \\
        MICH(0,L,K) Oracle &  &  &  & 0.51 & 5.25 & 0.91 & 44.65 & 7.14 & 14.71 & 0.64 \\
        MICH(J,0,0) &  &  &  & 1.70 & 8.93 & 0.95 & 4.72 & 9.79 & 3.43 & 4.35 \\
        MICH(J,0,0) Oracle &  &  &  & 0.18 & 3.88 & 0.95 & 24.17 & 5.18 & 8.33 & 3.83 \\
        MOSUM BU &  &  &  & 0.44 & 29.02 & 0.99 & 5.80 & 4.98 & 2.50 & 0.04 \\
        MOSUM LP &  &  &  & 0.63 & 16.04 & 1.00 & 15.51 & 5.66 & 5.05 & 0.03 \\
        NOT &  &  &  & 0.08 &  &  & 5.49 & 0.90 & 0.78 & 0.05 \\
        PELT &  &  &  & 0.06 &  &  & 3.53 & 0.61 & 0.53 & 0.00\\ \hline 

        H-SMUCE & 1000 & 2 & 50 & 0.04 & 90.55 & 1.00 & 12.11 & 2.51 & 3.09 & 0.01 \\
        MICH(0,L,K) &  &  &  & 0.34 & 17.15 & 0.86 & 8.74 & 12.15 & 5.59 & 7.38 \\
        MICH(0,L,K) Oracle &  &  &  & 0.00 & 9.21 & 0.86 & 6.32 & 2.60 & 2.60 & 0.06 \\
        MICH(J,0,0) &  &  &  & 0.92 & 26.87 & 0.93 & 2.70 & 26.44 & 6.87 & 5.23 \\
        MICH(J,0,0) Oracle &  &  &  & 0.00 & 5.34 & 0.93 & 3.21 & 2.05 & 1.93 & 0.06 \\
        MOSUM BU &  &  &  & 0.22 & 40.21 & 0.99 & 16.43 & 14.61 & 9.23 & 0.04 \\
        MOSUM LP &  &  &  & 0.62 & 25.74 & 0.98 & 17.49 & 28.09 & 13.03 & 0.03 \\
        NOT &  &  &  & 0.01 &  &  & 3.01 & 1.43 & 1.09 & 0.06 \\
        PELT &  &  &  & 0.02 &  &  & 3.27 & 1.63 & 1.09 & 0.00\\ \hline 

        H-SMUCE & 1000 & 10 & 50 & 0.44 & 50.16 & 0.98 & 31.72 & 1.52 & 3.34 & 0.01 \\
        MICH(0,L,K) &  &  &  & 0.92 & 6.75 & 0.88 & 69.86 & 9.30 & 13.81 & 27.41 \\
        MICH(0,L,K) Oracle &  &  &  & 0.89 & 6.31 & 0.88 & 101.28 & 11.75 & 21.92 & 15.08 \\
        MICH(J,0,0) &  &  &  & 1.25 & 7.87 & 0.94 & 43.60 & 8.91 & 7.90 & 26.32 \\
        MICH(J,0,0) Oracle &  &  &  & 0.43 & 5.05 & 0.94 & 76.32 & 9.68 & 15.04 & 19.06 \\
        MOSUM BU &  &  &  & 0.19 & 76.47 & 1.00 & 12.34 & 1.60 & 1.80 & 0.15 \\
        MOSUM LP &  &  &  & 0.99 & 21.92 & 1.00 & 50.82 & 7.12 & 9.29 & 0.07 \\
        NOT &  &  &  & 0.07 &  &  & 8.16 & 0.68 & 0.77 & 0.07 \\
        PELT &  &  &  & 0.07 &  &  & 5.72 & 0.66 & 0.62 & 0.54\\ \hline 
    \end{tabular}
    \caption{Simulation from \cite{Pein17} with results from MICH.}
    \label{tab:hsmuce-sim}
\end{table}


Table \ref{tab:hsmuce-sim} displays the results of the simulation study. The main take-away for our purposes is that among the methods that produce credible/confidence sets, the oracle version of MICH with $J$ set to the true number of change-points and $L=K=0$ dominates. Among the settings with uncertainty quantification, Oracle-MICH uniformly achieves the lowest bias $|L -\hat{L}|$ while still achieving the 90\% conditional coverage. Furthermore, every setting of MICH returns credible sets that are on average are an order of smaller than the confidence sets returned by H-SMUCE and MOSUM, while still achieving the nominal coverage. However, we do see that in terms of speed, MICH is still an order of magnitude slower than the other methods once $T = 1,000$. Furthermore, we see that PELT and NOT dominate the other methods as point estimators. Pelt in particular always achieves the lowest bias and has a much smaller Hausdorff statistic than the other methods, especially when the ratio of $L$ to $T$ is large. 

