% document set up
\documentclass{article}
\usepackage[utf8]{inputenc}
\usepackage{fullpage}
\setlength{\parindent}{0cm}
\setlength{\parskip}{1em}

% math packages
\usepackage{amsfonts}
\usepackage{amsmath}
\usepackage{amssymb}
\usepackage{amsthm}
\usepackage{enumitem}
\usepackage{bbm}

% general packages
\usepackage[colorlinks=true,linkcolor=blue,citecolor=blue]{hyperref}
\usepackage{natbib}

\newcommand\indep{\perp\!\!\!\perp}
\newcommand{\sforall}{\;\forall\;}
\newcommand{\E}{\mathbb{E}}
\renewcommand{\Pr}{\mathbb{P}}
\newcommand{\Var}{\text{Var}}
\newcommand{\Cov}{\text{Cov}}
\newcommand{\argmin}[1]{\underset{#1}{\text{arg min}}}
\newcommand{\argmax}[1]{\underset{#1}{\text{arg max}}}

\newtheorem{assumption}{Assumption}
\newtheorem{theorem}{Theorem}
\newtheorem{proposition}{Proposition}
\newtheorem{corollary}{Corollary}
\newtheorem{remark}{Remark}
\newtheorem{definition}{Definition}
\newtheorem{lemma}{Lemma}

\begin{document}

\section{Eigenvalue Change-Point Model}
Model:
\begin{align*}
    \mathbf{y}_t &\sim \overset{\text{ind.}}{\sim} \mathcal{N}_d \left(\mathbf{0}, [\mathbf{Q}_0 \boldsymbol{\Lambda}_t\mathbf{S}_t\mathbf{Q}_0^\intercal]^{-1}\right) \\
    \boldsymbol{\Lambda}_t &= \text{diag}(\{\lambda_{it} \}_{i=1}^d) \\
    \mathbf{S}_t &= \text{diag}\left(\left\{s_i^{\mathbbm{1}\{\gamma \geq t\}}\right\}_{i=1}^d\right) \\
    s_i &\overset{\text{i.i.d.}}{\sim} \text{Gamma}(u_0/2, v_0/2) \\
    \gamma &\sim \text{Categorical}(\boldsymbol{\pi}_{1:T}) \\
    \gamma &\indep \{s_i\}_{i=1}^d 
\end{align*}
Posterior:
\begin{align*}
    s_i \;|\; \gamma = t, \mathbf{y}_{1:T} &\overset{\text{ind.}}{\sim} \text{Gamma}(\overline{u}_t, \overline{v}_{it}) \\
    \gamma \;|\; \mathbf{y}_{1:T} &\sim \text{Categorical}(\overline{\boldsymbol{\pi}}_{1:T}) \\
    \overline{u}_t &= \frac{u_0 + T-t+1}{2} \\
    \overline{v}_{it} &= \frac{v_0 + \sum_{t'=t}^T \lambda_{t'i}[\mathbf{Q}^\intercal_0\mathbf{y}_{t'}]^2_{i}}{2} \\
    \overline{\pi}_t &\propto \pi_t \exp\left(-\frac{1}{2}\sum_{t'=1}^{t-1} \lVert\boldsymbol{\Lambda}^{\frac{1}{2}}_{t}\mathbf{Q}_0^\intercal\mathbf{y}_{t'}\rVert^2_2\right)\prod_{i=1}^d \Gamma(\overline{u}_t) \overline{v}_{it}^{-\overline{u}_t}
\end{align*}
\small
\begin{align*}
    p(\{s_i\}_{i=1}^d, \gamma = t \;|\; \mathbf{y}_{1:T}) &\propto  p(\mathbf{y}_{1:T} \;|\; \{s_i\}_{i=1}^d, \gamma = t) \pi_t \prod_{i=1}^T p(s_i) \\
    &\propto \pi_t \prod_{t'=1}^T |\mathbf{Q}_0 \boldsymbol{\Lambda}_{t'}\mathbf{S}_{t'}\mathbf{Q}_0^\intercal|^{\frac{1}{2}} \exp\left[-\frac{1}{2}\sum_{t'=1}^T \lVert \boldsymbol{\Lambda}^{\frac{1}{2}}_{t'}\mathbf{S}^{\frac{1}{2}}_{t'}\mathbf{Q}_0^\intercal\mathbf{y}_{t'}\rVert_2^2\right] \prod_{i=1}^d s_i^{\frac{u_0}{2}-1} \exp\left[-\frac{v_0s_i}{2}\right] \\
    &= \pi_t \exp\left[-\frac{1}{2}\sum_{t'=1}^{t-1} \lVert \boldsymbol{\Lambda}^{\frac{1}{2}}_{t'}\mathbf{Q}_0^\intercal\mathbf{y}_{t'}\rVert_2^2\right] \exp\left[-\frac{1}{2}\sum_{t'=t}^T \lVert \boldsymbol{\Lambda}^{\frac{1}{2}}_{t'}\mathbf{S}^{\frac{1}{2}}_{t'}\mathbf{Q}_0^\intercal\mathbf{y}_{t'}\rVert_2^2\right] \prod_{i=1}^d s_i^{\frac{u_0 + T-t+1}{2} - 1} \exp\left[-\frac{v_0s_i}{2}\right] \tag{$\gamma = t$} \\
    &= \pi_t \exp\left[-\frac{1}{2}\sum_{t'=1}^{t-1} \lVert \boldsymbol{\Lambda}^{\frac{1}{2}}_{t'}\mathbf{Q}_0^\intercal\mathbf{y}_{t'}\rVert_2^2\right] \prod_{i=1}^d s_i^{\frac{u_0 + T-t+1}{2} - 1} \exp\left[-s_0\left(\frac{v_0 + \sum_{t'=t}^T\lambda_{t'i}[\mathbf{Q}^\intercal_0\mathbf{y}_{t'}]^2_{i}}{2}\right)\right] \\
    &= \pi_t \exp\left[-\frac{1}{2}\sum_{t'=1}^{t-1} \lVert \boldsymbol{\Lambda}^{\frac{1}{2}}_{t'}\mathbf{Q}_0^\intercal\mathbf{y}_{t'}\rVert_2^2\right] \prod_{i=1}^d \frac{\Gamma(\frac{u_0 + T-t+1}{2})}{\left(\frac{v_0 + \sum_{t'=t}^T\lambda_{t'i}[\mathbf{Q}^\intercal_0\mathbf{y}_{t'}]^2_{i}}{2}\right)^{\frac{u_0 + T-t+1}{2}}} \\
    &\quad\times \prod_{i=1}^d \frac{\left(\frac{v_0 + \sum_{t'=t}^T\lambda_{t'i}[\mathbf{Q}^\intercal_0\mathbf{y}_{t'}]^2_{i}}{2}\right)^{\frac{u_0 + T-t+1}{2}}}{\Gamma(\frac{u_0 + T-t+1}{2})}s_i^{\frac{u_0 + T-t+1}{2} - 1} \exp\left[-s_0\left(\frac{v_0 + \sum_{t'=t}^T\lambda_{t'i}[\mathbf{Q}^\intercal_0\mathbf{y}_{t'}]^2_{i}}{2}\right)\right] 
\end{align*}
\normalsize

\subsection{Single Mean-Variance Change-Point (MeanVar-SCP) Model}

We now merge the mean- and cov-scp settings by allowing $\mathbf{y}_{1:T}$ to have a single change-point where both the mean and covariance shift simultaneously. We again assume that we have $T$ observations from (\ref{eq:dgp}), only now the concurrent change to the piece-wise constant structure of $\boldsymbol{\mu}_{1:T}$ and $\boldsymbol{\Lambda}_{1:T}$ is generated by the following Single Mean-Covariance Change-Point (meancov-scp) model:
\begin{align}
    \label{eq:smscp-start}
    \boldsymbol{\mu}_t &= \mathbf{b}\mathbbm{1}{\left\{t\geq \gamma \right\}} \\
    \boldsymbol{\Lambda}_t &= 
    \begin{cases}
        \boldsymbol{\Psi}_{0}, & t < \gamma  \\
        \boldsymbol{\Psi}^\frac{1} {2}_{0}\mathbf{S}\boldsymbol{\Psi}^\frac{1}{2}_{0}, & t \geq \gamma
    \end{cases} \label{eq:psi_0}
    \\
    \mathbf{b} \:|\: \mathbf{S} &\sim \mathcal{N}_d\left(\mathbf{0},\left[\lambda_0\boldsymbol{\Psi}^\frac{1}{2}_{0}\mathbf{S}\boldsymbol{\Psi}^\frac{1}{2}_{0}\right]^{-1}\right) \\
    \mathbf{S} &\sim \text{Wishart}(u_0, v_0^{-1}\mathbf{I}_d) \\    
    \gamma &\sim \text{Categorical}(\boldsymbol{\pi}_{1:T}) \\
    \gamma &\indep \{\mathbf{b},\mathbf{S}\}. 
    \label{eq:smscp-end}
\end{align}
Note that the construction of $\boldsymbol{\Lambda}_t$ in (\ref{eq:psi_0}) differs from (\ref{eq:sscp-start}) in the cov-scp model in that $\boldsymbol{\Psi}_t = \boldsymbol{\Psi}_0$ for all $t$. If the parameter $\boldsymbol{\Psi}_t$ is allowed to vary over time, then we lose conjugacy in the meancov-scp model. The posterior distribution for $p(\mathbf{b}, \mathbf{S}, \gamma \:|\: \mathbf{y}_{1:T})$ under the meancov-scp setting is given by:
\begin{align}
    \mathbf{b} \:|\: \mathbf{S}, \gamma = t, \mathbf{y}_{1:T} &\sim \mathcal{N}_d\left(\overline{\mathbf{b}}_t, \left[\overline{\boldsymbol{\Psi}}^{\frac{1}{2}}_t \mathbf{S} \overline{\boldsymbol{\Psi}}^{\frac{1}{2}}_t\right]^{-1} \right) \label{eq:b-smscp} \\
    \mathbf{S} \:|\: \gamma = t, \mathbf{y}_{1:T} &\sim \text{Wishart}(\overline{u}_t, \overline{\mathbf{V}}_t)  \\
    \gamma \:|\: \mathbf{y}_{1:T} &\sim \text{Categorical}(\overline{\boldsymbol{\pi}}_{1:T}) \label{eq:gamma-smscp} \\
    \overline{\boldsymbol{\Psi}}_t &= (\lambda_0 + T-t+1)\boldsymbol{\Psi}_0 \\
    \overline{\mathbf{b}}_t &= \overline{\boldsymbol{\Psi}}_t^{-1}\sum_{t'=t}^{T} \boldsymbol{\Psi}_0\mathbf{y}_{t'} \\
    \overline{u}_t &= u_0 + T - t + 1 \\
    \overline{\mathbf{V}}_t &= v_0\mathbf{I}_d - \left\lVert\overline{\boldsymbol{\Psi}}^{\frac{1}{2}}_t\overline{\mathbf{b}}_t\right\rVert_2^2 +  \sum_{t'=t}^{T} \boldsymbol{\Psi}_0^{\frac{1}{2}}\mathbf{y}_{t'} \mathbf{y}_{t'}^\intercal\boldsymbol{\Psi}_0^{\frac{1}{2}}\\
    \overline{\pi}_t &\propto \pi_t |\overline{\boldsymbol{\Psi}}_t |^{-\frac{1}{2}} \left(2^d |\overline{\mathbf{V}}_{t}|\right)^{\frac{\overline{u}_{t}}{2}}\Gamma_d\left(\frac{\overline{u}_{t}}{2}\right)\exp\left(- \frac{1}{2}\sum_{t'=1}^{t-1} \lVert\boldsymbol{\Psi}_0^{\frac{1}{2}}\mathbf{y}_{t'}\rVert_2^2\right).
\end{align}
As was the case with the cov-scp model, the meancov-scp model loses its utility in the multiple change-point setting. However, if we are again willing to make the assumption that  therefore, we We again define a function \texttt{meancov-scp} that takes $\mathbf{y}_{1:T}$ and the model parameters as inputs and returns the posterior parameters of $p(\mathbf{b}, \mathbf{S}, \gamma\:|\:\mathbf{y}_{1:T})$ as its output:
\begin{align}
    \texttt{meancov-scp}\left(\mathbf{y}_{1:T} \:;\: \lambda_0, \boldsymbol{\Psi}_0, u_0, v_0, \boldsymbol{\pi}_{1:T}\right) := \{\overline{\mathbf{b}}_t, \overline{\boldsymbol{\Psi}}_t, \overline{u}_t, \overline{\mathbf{V}}_t, \overline{\pi}_t\}_{t=1}^T
\end{align}
as well as a shorthand for the distribution (\ref{eq:b-smscp})-(\ref{eq:gamma-smscp})
\begin{align}
    \{\mathbf{b}, \mathbf{S}, \gamma\}\sim\text{meancov-scp}(\{\overline{\mathbf{b}}_t, \overline{\boldsymbol{\Psi}}_t, \overline{u}_t, \overline{\mathbf{V}}_t, \overline{\pi}_t\}_{t=1}^T).
\end{align}
As was the case with the cov-scp model, the conjugate meancov-scp model does not translate nicely to the multiple change-point setting, so we again replace $\boldsymbol{\Lambda}_t$ in (\ref{eq:psi_0}) with the same construction as in (\ref{eq:sscp-hadamard}). Using the same approximating technique from Section \ref{sec:sscp} we get an approximate posterior where $\{\mathbf{b}, \mathbf{S}, \gamma\}\sim\text{meancov-scp}(\{\overline{\mathbf{b}}_t, \overline{\boldsymbol{\Psi}}_t, \overline{u}_t, \overline{\mathbf{V}}_t, \overline{\pi}_t\}_{t=1}^T)$ with:
\end{document}