\documentclass{article}
\usepackage[utf8]{inputenc}
\usepackage{fullpage}
\setlength{\parindent}{0cm}
\setlength{\parskip}{1em}
\usepackage{amsmath}
\usepackage{amsfonts}
\usepackage{amssymb}
\usepackage{enumitem}
\usepackage{bbm}
\usepackage{hyperref}
\usepackage{qtree}

\newcommand{\sforall}{\;\forall\;}
\newcommand{\E}{\mathbb{E}}
\newcommand{\Var}{\text{Var}}
\newcommand{\Cov}{\text{Cov}}
\newcommand{\argmin}[1]{\underset{#1}{\text{arg min}}}
\newcommand{\argmax}[1]{\underset{#1}{\text{arg max}}}

\begin{document}

\section{Multiple Effect Lower Bound}

Let $\theta_\ell = \{b_\ell, s_\ell^2, \gamma_\ell\}$ and $\mathbf{D} = \{\mathbf{y}, \mathbf{X}\}$. With a slight abuse of notation, we let $\theta_{1:L} = \{\theta_1, \ldots, \theta_L\}$ and we want to find the distribution $q_{1:L}(\theta_{1:L})$ that maximizes the lower bound on the KL divergence between the true posterior distribution and $q_{1:L}$
\begin{align*}
    \text{LB}(q_{1:L}(\theta_{1:L})) &= \int q_{1:L}(\theta_{1:L}) \log \frac{p(\theta_{1:L}, \mathbf{D})}{q_{1:L}(\theta_{1:L})} \;d\theta_{1:L}.
\end{align*}
We make the following first Mean-Field assumption,
\begin{equation}
    q_{1:L}(\theta_{1:L}) = \prod_{\ell=1}^L q_\ell(\theta_\ell).
\end{equation}
Then, letting $q_{-\ell}(\theta_{-\ell}) = \prod_{j\neq\ell} q_j(\theta_j)$, we have
\begin{align*}
    \text{LB}(q_{1:L}(\theta_{1:L})) &= \int\int q_\ell(\theta_\ell) q_{-\ell}(\theta_{-\ell}) \log \frac{p(\theta_{1:L}, \mathbf{D})}{q_\ell(\theta_\ell) q_{-\ell}(\theta_{-\ell})} \;d\theta_{-\ell} d\theta_{\ell} \\
    &= \int q_\ell(\theta_\ell) \int q_{-\ell}(\theta_{-\ell}) \log p(\theta_{1:L}, \mathbf{D}) \;d\theta_{-\ell}d\theta_{\ell} \\
    &\quad - \int \int q_\ell(\theta_\ell) q_{-\ell}(\theta_{-\ell})\log q_\ell(\theta_\ell)\;d\theta_{-\ell}d\theta_{\ell} \\
    &\quad - \int \int q_\ell(\theta_\ell) q_{-\ell}(\theta_{-\ell})\log q_{-\ell}(\theta_{-\ell}) \;d\theta_{-\ell}d\theta_{\ell}  \\
    &= \int q_\ell(\theta_\ell) \E_{-\ell}\left[\log p(\theta_{1:L}, \mathbf{D})\right]\;d\theta_{\ell} \\
    &\quad - \int q_\ell(\theta_\ell) \log q_\ell(\theta_\ell)\;d\theta_{\ell} - \int q_{-\ell}(\theta_{-\ell})\log q_{-\ell}(\theta_{-\ell}) \;d\theta_{-\ell} \\ 
    &= \int q_\ell(\theta_\ell) \log \frac{\exp\left(\E_{-\ell}\left[\log p(\theta_{1:L}, \mathbf{D})\right]\right)}{q_\ell(\theta_\ell)}\;d\theta_{\ell} + C(q_{-\ell})
\end{align*}
where $C(q_{-\ell})$ is a constant that does not depend on $q_\ell$ and
\begin{align*}
    \E_{-\ell}\left[\log p(\theta_{1:L}, \mathbf{D})\right] =  \int q_{-\ell}(\theta_{-\ell}) \log p(\theta_{1:L}, \mathbf{D}) \;d\theta_{-\ell}.
\end{align*}
Note that if we define 
\begin{align*}
    q'_\ell(\theta_\ell) = \frac{\exp\left(\E_{-\ell}\left[\log p(\theta_{1:L}, \mathbf{D})\right]\right)}{\int \exp\left(\E_{-\ell}\left[\log p(\theta_{1:L}, \mathbf{D})\right]\right) \; d\theta_\ell}
\end{align*}
then the lower bound simplifies to 
\begin{align*}
     \text{LB}(q_{1:L}(\theta_{1:L})) = -\text{KL}(q_\ell \lVert q'_\ell) + \log \int \exp\left(\E_{-\ell}\left[\log p(\theta_{1:L}, \mathbf{D})\right]\right) \; d\theta_\ell + C(q_{-\ell}).
\end{align*}
The negative KL divergence achieves its zero upper-bound when $q_\ell$ is set equal to $q'_\ell$, and the other two terms are independent of $q_\ell$; therefore, we can maximize the lower bound using a coordinate ascent approach where at step $t$ we use $q^{(t)}_{-\ell}$ to find $q'_\ell$, store $q^{(t)}_\ell := q'_\ell$, and iterate until convergence.

\section{Single Effect Structure}

Without making any additional assumptions about the structure of $q_\ell$, we can always write
\begin{align*}
    q_\ell(\theta_\ell) = q_\ell(b_\ell\;|\;s^2_\ell,\gamma_\ell)q_\ell(s^2_\ell\;|\;\gamma_\ell)q_\ell(\gamma_\ell).
\end{align*}
Then we have 
\begin{align*}
    \text{LB}(q_{1:L}(\theta_{1:L})) &= \int q_\ell(\theta_\ell) \E_{-\ell}\left[\log p(\theta_{1:L}, \mathbf{D})\right]\;d\theta_{\ell} - \int q_\ell(\theta_\ell) \log q_\ell(\theta_\ell)\;d\theta_{\ell} + C(q_{-\ell}) \\
    &= \int\int q_\ell(s^2_\ell, \gamma_\ell) \int q_\ell(b_\ell\;|\;s^2_\ell,\gamma_\ell)\left(\E_{-\ell}\left[\log p(\theta_{1:L}, \mathbf{D})\right] - \log q_\ell(b_\ell\;|\;s^2_\ell,\gamma_\ell) \right)\; db_\ell \; ds_\ell^2\;d\gamma_\ell \\
    &\quad - \int\int q_\ell(s^2_\ell, \gamma_\ell) \log q_\ell(s^2_\ell, \gamma_\ell) \; ds_\ell^2\;d\gamma_\ell + C(q_{-\ell}) \\
    &= \int\int q_\ell(s^2_\ell, \gamma_\ell) \int q_\ell(b_\ell\;|\;s^2_\ell,\gamma_\ell) \log \frac{\exp\left(\E_{-\ell}\left[\log p(\theta_{1:L}, \mathbf{D})\right]\right)}{q_\ell(b_\ell\;|\;s^2_\ell,\gamma_\ell)}\; db_\ell \; ds_\ell^2\;d\gamma_\ell \\
    &\quad - \int\int q_\ell(s^2_\ell, \gamma_\ell) \log q_\ell(s^2_\ell, \gamma_\ell) \; ds_\ell^2\;d\gamma_\ell + C(q_{-\ell})
\end{align*}

$$q(b,s^2,\gamma) = q(b|\gamma)q(s|\gamma)q(\gamma)$$

$$\E[\E[\tau^2_{\ell,t} \beta_\ell |\gamma]] =\E[\E[\tau^2_{\ell,t}|\gamma]\E[\beta_\ell |\gamma]]  $$

\section{Conditionally Independent Location and Scale Effects}

Suppose now that the location and scale effects are independent conditional on the location of the effect, i.e.
\begin{align}
    q_\ell(\theta_\ell) = q_\ell(b_\ell\;|\;\gamma_\ell)q_\ell(s^2_\ell\;|\;\gamma_\ell)q_\ell(\gamma_\ell).
\end{align}
In this setting we have 
\begin{align*}
    \text{KL}(q_\ell \lVert q'_\ell) &=  \int q_\ell(\theta_\ell) \log \frac{\exp\left(\E_{-\ell}\left[\log p(\theta_{1:L}, \mathbf{D})\right]\right)}{q_\ell(\theta_\ell)}\;d\theta_{\ell} + C_{-\ell} \\
    &=  \int\int\int   q_\ell(b_\ell\;|\;\gamma_\ell)q_\ell(s^2_\ell\;|\;\gamma_\ell)q_\ell(\gamma_\ell)\E_{-\ell}\left[\log p(\theta_{1:L}, \mathbf{D})\right] \;db_\ell\;ds^2_\ell\;d\gamma_\ell \\
    &\quad - \int\int   q_\ell(b_\ell\;|\;\gamma_\ell)q_\ell(\gamma_\ell)\log q_\ell(b_\ell\;|\;\gamma_\ell)\;db_\ell\;d\gamma_\ell - \int\int q_\ell(s^2_\ell\;|\;\gamma_\ell)q_\ell(\gamma_\ell)\log q_\ell(s^2_\ell\;|\;\gamma_\ell)\;ds^2_\ell\;d\gamma_\ell \\
    &\quad - \int q_\ell(\gamma_\ell)\log q_\ell(\gamma_\ell)\;d\gamma_\ell + C_{-\ell}.
\end{align*}
Focusing on the sub-problem of finding the optimal $q_\ell(b_\ell\;|\;\gamma_\ell)$, we have
\begin{align*}
    \text{KL}(q_\ell \lVert q'_\ell) &= 
    \int q_\ell(\gamma_\ell)\int q_\ell(b_\ell\;|\;\gamma_\ell)\log\frac{\exp\left(\E_{s^2_\ell|\gamma_\ell}\left\{\E_{-\ell}\left[\log p(\theta_{1:L}, \mathbf{D})\right]\right\}\right)}{q_\ell(b_\ell\;|\;\gamma_\ell)} \;db_\ell\;d\gamma_\ell \\
    &\quad - \int\int q_\ell(s^2_\ell\;|\;\gamma_\ell)q_\ell(\gamma_\ell)\log q_\ell(s^2_\ell\;|\;\gamma_\ell)\;ds^2_\ell\;d\gamma_\ell - \int q_\ell(\gamma_\ell)\log q_\ell(\gamma_\ell)\;d\gamma_\ell + C_{-\ell}
\end{align*}
where 
\begin{align*}
    \E_{s^2_\ell|\gamma_\ell}\left\{\E_{-\ell}\left[\log p(\theta_{1:L}, \mathbf{D})\right]\right\} = \int \E_{-\ell}\left[\log p(\theta_{1:L}, \mathbf{D})\right] q_\ell(s^2_\ell\;|\;\gamma_\ell) \; ds^2_\ell.
\end{align*}
If we let $$C'_{\ell}(\gamma_\ell) = \left(\int \exp\left(\E_{s^2_\ell|\gamma_\ell}\left\{\E_{-\ell}\left[\log p(\theta_{1:L}, \mathbf{D})\right]\right\}\right)\;db_\ell\right)^{-1}$$ and define $$q'_\ell(b_\ell \;|\; \gamma_\ell) = C'_{\ell}(\gamma_\ell)\exp\left(\E_{s^2_\ell|\gamma_\ell}\left\{\E_{-\ell}\left[\log p(\theta_{1:L}, \mathbf{D})\right]\right\}\right),$$ then we have the following simplification 
\begin{align*}
    \text{KL}(q_\ell \lVert q'_\ell) &= 
    \int -\text{KL}(q_\ell(\cdot \;|\; \gamma_\ell)\;\lVert\; q'_\ell(\cdot \;|\; \gamma_\ell))q_\ell(\gamma_\ell)\;d\gamma_\ell- \int q_\ell(\gamma_\ell) \log C'_{\ell}(\gamma_\ell) \;d\gamma_\ell \\
    &\quad - \int\int q_\ell(s^2_\ell\;|\;\gamma_\ell)q_\ell(\gamma_\ell)\log q_\ell(s^2_\ell\;|\;\gamma_\ell)\;ds^2_\ell\;d\gamma_\ell - \int q_\ell(\gamma_\ell)\log q_\ell(\gamma_\ell)\;d\gamma_\ell + C_{-\ell}.
\end{align*}
Note that only the first term depends on $q_\ell(b_\ell\;|\;\gamma_\ell)$, and this term is maximized by setting $$q_\ell(b_\ell\;|\;\gamma_\ell) := q'_\ell(b_\ell\;|\;\gamma_\ell).$$ A symmetric argument shows that the solution to the sub-problem of finding $q_\ell(s^2_\ell\;|\;\gamma_\ell)$ is given by 
\begin{align*}
    q_\ell(s^2_\ell\;|\;\gamma_\ell) = \frac{\exp\left(\E_{b_\ell|\gamma_\ell}\left\{\E_{-\ell}\left[\log p(\theta_{1:L}, \mathbf{D})\right]\right\}\right)}{\int \exp\left(\E_{b_\ell|\gamma_\ell}\left\{\E_{-\ell}\left[\log p(\theta_{1:L}, \mathbf{D})\right]\right\}\right)\;ds^2_\ell}.
\end{align*}
Lastly, if we define 
\begin{align*}
    \E_{b_\ell,s^2_\ell|\gamma_\ell}\left\{\E_{-\ell}\left[\log p(\theta_{1:L}, \mathbf{D})\right]\right\} &= \int\int \E_{-\ell}\left[\log p(\theta_{1:L}, \mathbf{D})\right] q_\ell(b_\ell\;|\;\gamma_\ell) q_\ell(s^2_\ell\;|\;\gamma_\ell) \; db_\ell\; ds^2_\ell \\
    \E_{b_\ell|\gamma_\ell}[\log q_\ell(b_\ell\;|\;\gamma_\ell)] &= \int q_\ell(b_\ell\;|\;\gamma_\ell)\log q_\ell(b_\ell\;|\;\gamma_\ell) \; db_\ell \\
    \E_{s^2_\ell|\gamma_\ell}[\log q_\ell(s^2_\ell\;|\;\gamma_\ell)] &= \int q_\ell(s^2_\ell\;|\;\gamma_\ell)\log q_\ell(s^2_\ell\;|\;\gamma_\ell) \; ds^2_\ell.
\end{align*}
Then,
\small
\begin{align*}
    \text{KL}(q_\ell \lVert q'_\ell) &= 
    \int q_\ell(\gamma_\ell)\log \frac{\exp\left(\E_{b_\ell,s^2_\ell|\gamma_\ell}\left\{\E_{-\ell}\left[\log p(\theta_{1:L}, \mathbf{D})\right]\right\}- \E_{b_\ell|\gamma_\ell}[\log q_\ell(b_\ell\;|\;\gamma_\ell)]- \E_{s^2_\ell|\gamma_\ell}[\log q_\ell(s^2_\ell\;|\;\gamma_\ell)]\right)}{q_\ell(\gamma_\ell)} \;d\gamma_\ell \\
    &\quad + C_{-\ell}.
\end{align*}
\normalsize
So as above, we have 
\begin{align*}
    q_\ell(\gamma_\ell) \propto \exp\left(\E_{b_\ell,s^2_\ell|\gamma_\ell}\left\{\E_{-\ell}\left[\log p(\theta_{1:L}, \mathbf{D})\right]\right\}- \E_{b_\ell|\gamma_\ell}[\log q_\ell(b_\ell\;|\;\gamma_\ell)]- \E_{s^2_\ell|\gamma_\ell}[\log q_\ell(s^2_\ell\;|\;\gamma_\ell)]\right).
\end{align*}


\end{document}